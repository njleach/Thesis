\chapter{\label{resources}Resources}

This appendix lists and briefly describes additional non-research output that I have produced over the course of my thesis.

\section{Python packages}

  \paragraph*{\texttt{mystatsfunctions}}
    
    \texttt{mystatsfunctions} is a repository of scripts that perform basic statistical operations. It contains two modules: \texttt{OLSE}, for computing vectorised linear regression analyses; and \texttt{LMoments}, for fitting distributions using the method of L-Moments \citep{hosking_estimation_1985,hosking_parameter_1987,hosking_l-moments_1990,hosking_regional_1997}. The repository is publicly available from \url{https://github.com/njleach/mystatsfunctions}.

  \paragraph*{\texttt{moarpalettes}}
    
    \texttt{moarpalettes} is a repository of scripts that allow easy loading of numerous colour palettes for use in \texttt{seaborn} or \texttt{matplotlib}. I have made extensive use of it throughout the thesis. The repository is publicly available from \url{https://github.com/njleach/mystatsfunctions}.

  \paragraph*{\texttt{FaIRv2.0.0-alpha}}
    
    \texttt{FaIRv2.0.0-alpha} is a simple climate model for use in probabilistic future climate and scenario exploration, integrated assessment, policy analysis, and education. I led a model description paper about it during this PhD \citep{leach_fairv200_2021}. The model code alpha release is publicly available from \url{https://doi.org/10.5281/zenodo.4683173}.

\section{Code and data availability}

  \paragraph*{Chapter 2}
    
    Code used to carry out the analysis in \hyperref[ch2]{chapter 2} is publicly available from \url{https://github.com/njleach/Ch1_EU-heatwave-2018}.

  \paragraph*{Chapter 3}
    
    Code used to carry out the analysis in \hyperref[ch3]{chapter 3} is publicly available from \url{https://doi.org/10.5281/zenodo.5416058}. Data required to reproduce the analysis is available from \url{https://doi.org/10.5285/DD6A312C701F47778390DE50CD052071}. Further data produced by the study are available from ECMWF's MARS.

  \paragraph*{Chapter 4}
    
    Code used to carry out the analysis in \hyperref[ch4]{chapter 4} is currently available upon request from \url{https://github.com/njleach/PNW-attribution-manuscript}.

  \paragraph*{Chapter 5}
    
    Code used to carry out the analysis in \hyperref[ch5]{chapter 5} is publicly available from \url{https://doi.org/10.5281/zenodo.6327159}. Data required to reproduce the analysis is available from \url{https://doi.org/10.5281/zenodo.6327360}.

\section{Public outreach \& engagement}

  Over the course of my PhD I have been lucky enough to have the opportunity to perform a number of activities aiming to engage with an audience outside the immediate scientific community. These activities are listed below.

    \paragraph*{Article in CarbonBrief}
    
      I led the writing of an article in CarbonBrief, aiming to explain why we are particularly interested in forecast-based approaches, and how they differ from existing ones. The article is found at \url{https://www.carbonbrief.org/guest-post-how-weather-forecasts-can-spark-a-new-kind-of-extreme-event-attribution}.

    \paragraph*{Article in \emph{Science}}
    
      I was quoted in a news article in \emph{Science} following the Pacific Northwest Heatwave exploring the various ways in which a number of different groups are trying to understand extreme events better. The article is found at \url{https://www.science.org/content/article/record-shattering-events-spur-advances-in-tying-climate-change-to-extreme-weather}.

    \paragraph*{Public talk on extreme event attribution}
    
      I delivered a live-streamed talk about the attribution of extreme weather events as part of the Oxford@home COP Conversations series. In it, I try to briefly cover the current state of the science, as well as going into some detail about my own PhD research into the use of weather forecast models for attribution. The recording is found at \url{https://youtu.be/171HEr-6b6w}.

    \paragraph*{Public webinar on ExSamples}
    
      I delivered a talk on the underlying science as part of a webinar explaining the work described in \hyperref[ch5]{chapter 5} that I and several co-authors had carried out. The recording is found at \url{https://youtu.be/O--6xE_hgJI}.

    \paragraph*{Appearance on Radio Ecoshock}
    
      I was interviewed by Radio Ecoshock about the work described in \hyperref[ch3]{chapter 3} of this thesis. The edited recording of this interview is found at \url{https://www.ecoshock.org/2022/02/fixing-the-climate-hopes-and-hazards.html}.