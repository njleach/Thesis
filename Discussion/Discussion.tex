\begin{savequote}[8cm]
    Quote
      \qauthor{--- author}
\end{savequote}
    
\chapter{\label{discussion}Discussion} 

Chapter description.
% \small\paragraph{Author contributions:} This chapter is based on the the following publication \footnote{with the author contributions as follows.} \par\vspace{1em}
% \formatchref{Surname, I1. I2., Surname, I1. I2.}{year}{Title}{Journal}{vol}{issue}{pages}{DOI}

\minitoc

\clearpage

\section{An overview of this thesis}

  \blindtext

\section{This thesis in the context of previous work}

  References of relevant work. Hurricane forecast-based \citep{reed_attribution_2022,reed_forecasted_2020,patricola_anthropogenic_2018,lackmann_hurricane_2015}. Nested forecast modelling \citep{schaller_role_2020,meredith_crucial_2015}. Initialised (sub)seasonal \citep{hope_contributors_2015,hope_what_2016,hope_determining_2019,hope_subseasonal_2022,wang_initialized_2021,tradowsky_toward_2022,stone_effect_2022}. DADA \citep{hannart_dada_2016}. Probabilistic methodology ref \citep{pall_anthropogenic_2011}. Pseudo global warming \citep{schar_surrogate_1996}.

\section{Limitations}

  Although I have discussed various limitations of the individual studies that make up this thesis, in this section I consider some of the limitations of the forecast-based approach to attribution developed and explored here as a whole.

  \subsection*{Forecast adjustment}

    One key aspect of the counterfactual forecasts performed here is that the model --- or more specifically the model atmosphere --- adjusts continually to the imposed perturbations throughout the integration. This means that the further into the forecast the event of interest happens, the stronger the attributed impact of those perturbations is. At the same time, as the forecast evolves, this effect becomes more uncertain in general (though not necessarily always), due to the increasing dynamical noise arising from the chaotic nature of the weather system. This combination of this increasing strength and uncertainty can make analysing and interpreting the results of the counterfactual forecast experiments difficult. In chapter \ref{ch4} I alleviated this difficulty by making use of the fact that the attributable regional impacts of climate change were near-linearly related to the coincidental measured level of global warming. This linear relationship allowed me to benchmark the estimated impact at each forecast lead time to the same level of global warming, regardless of how adjusted they were at the time of the event in question. However, this linear relationship is not guaranteed for every extreme event, and therefore it would be valuable to find methodologies by which this adjustment could either be reduced or removed entirely. I consider a few ideas to achieve this below.

  \subsection*{Additional uncertainty dimensions}

    In the experiments performed here, I have only considered uncertainty arising from the chaotic nature of the weather system. However, there are additional uncertainties associated with the approach I have developed. 

  \subsection*{Single model}

  \subsection*{Single event class}

  \subsection*{Considering additional forcing agents}

  

\section{Future research directions}

  \subsection*{Assimilating the perturbations}

    \blindtext

  \subsection*{Alternative extremes}

    \blindtext

  \subsection*{Alternative forecast models}

    \blindtext

  \subsection*{Incorporating perturbation uncertainty}

    \blindtext

  \subsection*{Projections of future extremes}

    \blindtext

  \subsection*{Impact assessment}

    \blindtext

\section{Concluding remarks}