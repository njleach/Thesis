\begin{savequote}[8cm]
    Quote
      \qauthor{--- author}
\end{savequote}
    
\chapter{\label{intro}Introduction} 

In this chapter I introduce the problem of attribution of individual extreme weather events to anthropogenic climate change. I review the current methodologies and frameworks that address this problem, in particular the contrasting storyline and probabilistic approaches to attribution. Although these frameworks are gaining acceptance and maturity, I suggest that a weather forecast-based approach could further increase the trustworthiness of attribution studies. Finally, I provide a conceptual sketch of these various attribution frameworks within a simple non-linear dynamical system.
\small\paragraph{Author contributions:} This chapter is based on the the following publication \footnote{with the author contributing as follows.} \par\vspace{1em}
\formatchref{Surname, I1. I2., Surname, I1. I2.}{year}{Title}{Journal}{vol}{issue}{pages}{DOI}

\minitoc

\clearpage

\section{The question of extreme event attribution}

  % detection and attribution of climate change (hasselmann)
  % -> liability for climate change
  % Being specific about the question ("cause" vs influence, hannart)
  % scientific challenges posed by EEA
  \blindtext

\section{Motivating the question}

  % start off with liability
  % public interest
  % provoke action (extreme events are "visible" dangers from CC)
  \blindtext

\section{Answering the question}

\subsection{Conventional attribution}

  % typical "WWA" approach
  % trend-based observations
  % climate model runs
  % potential issues: specificity, model deficiency, short record
  \blindtext

\subsection{Attribution through storylines}

  % the aims of storyline approaches
  % Hoerling - hazelger - shepherd - benitez
  % issues: does not provide estimate of change in risk
  \blindtext

\section{Conceptualising attribution}

  % use the dynamicist's favourite "toy": Lorenz63
  % demonstrate "storyline / DADA", conventional attribution
  % propose forecast-based attribution
  \blindtext

\section{What to expect in this thesis}

  % exploratory conventional attribution study 
  % tangent study about climate change projections + link to forecast based approaches
  % first step to forecast-based attribution - partial attribution
  % "full" forecast-based attribution
  % conclusion discussing current limitations + future directions
  \blindtext