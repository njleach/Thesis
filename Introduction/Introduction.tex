\begin{savequote}[8cm]
    Quote
      \qauthor{--- author}
\end{savequote}
    
\chapter{\label{intro}Introduction} 

In this chapter I introduce the problem of attribution of individual extreme weather events to anthropogenic climate change. I review the current methodologies and frameworks that address this problem, in particular the contrasting storyline and probabilistic approaches to attribution. Although these frameworks are gaining acceptance and maturity, I suggest that a weather forecast-based approach could further increase the trustworthiness of attribution studies. Finally, I provide a conceptual sketch of these various attribution frameworks within a simple non-linear dynamical system.
\small\paragraph{Author contributions:} This chapter is based on the the following publication \footnote{with the author contributing as follows.} \par\vspace{1em}
\formatchref{Surname, I1. I2., Surname, I1. I2.}{year}{Title}{Journal}{vol}{issue}{pages}{DOI}

\minitoc

\clearpage

\section{The question of extreme event attribution}

  % detection and attribution of climate change (hasselmann->gillet)
  % Being specific about the question ("cause" vs influence, hannart)
  % scientific challenges posed by EEA
  \blindtext

\section{Motivating the question}

  Now that I have posed the question, before I move on to how we might answer it, I think it would be useful for me to discuss why we want to answer it. In short: \emph{what's the point of this thesis?}

  % start off with liability (allen03)
  In 2003, Myles Allen wrote \citetitle{allen_liability_2003} \citep{allen_liability_2003}. This commentary is widely acknowledged as the first time the idea that individual extreme events could be attributed to external drivers such as human influence was proposed \citep{otto_attribution_2017}. Although \citeauthor{allen_liability_2003} touched on both methodology and motivation for extreme event attribution, here I shall focus on the latter aspect. The motivation behind extreme event attribution as proposed in \citetitle{allen_liability_2003} is compensation for individual damages caused by climate change or, as \citeauthor{allen_liability_2003} puts it, 
  \begin{quote}
    Will it ever be possible to sue anyone for damaging the climate?
  \end{quote}
  Tort law \citep{allen_scientific_2007,lloyd_climate_2018,lloyd_environmental_2020,lloyd_climate_2021,lloyd_climate_2021-1,stuart-smith_filling_2021,stuart-smith_increased_2021,marjanac_acts_2017,marjanac_extreme_2018,burger_law_2020}. 
  Loss and damage \citep{clarke_inventories_2021,wehner_operational_2022}.

  % public interest
  % provoke action (extreme events are "visible" dangers from CC)
  \citep{ettinger_whats_2021,osaka_natural_2020,swain_attributing_2020}

  % improve models (high benchmark of test)
  \citep{sillmann_understanding_2017}

\section{Answering the question}

  At this point, I have discussed the question that this thesis is primarily concerned with, and the reasons why it is of broad importance. Now, I shall explore the ways in which previous studies have approached this question, in particular focusing on the probabilistic \citep[often ``conventional'',][]{stott_human_2004} and storyline \citep{hoerling_anatomy_2013} frameworks. 

  \subsection{Probabilistic attribution}

    % typical "WWA" approach
    Probabilistic attribution seeks, ultimately, to determine the change in probability of an extreme event arising due to some external driver. This was the approach to extreme event attributon proposed in \citet{allen_liability_2003} and first applied in \citet{stott_human_2004}. 

    % trend-based observations / simulations
    % climate model runs
    % potential issues: specificity, model deficiency, short record

  \subsection{Attribution through storylines}

    % the aims of storyline approaches
    % Hoerling - hazelger - shepherd - benitez
    % issues: does not provide estimate of change in risk
    \blindtext

  \subsection{Synthesising the two}

  \subsection{Operationalising attribution}

\section{Conceptualising attribution}

  % use the dynamicist's favourite "toy": Lorenz63
  % demonstrate "storyline / DADA", conventional attribution
  % propose forecast-based attribution
  \blindtext

\section{What to expect in this thesis}

  % exploratory conventional attribution study 
  % tangent study about climate change projections + link to forecast based approaches
  % first step to forecast-based attribution - partial attribution
  % "full" forecast-based attribution
  % conclusion discussing current limitations + future directions
  \blindtext