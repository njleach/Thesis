%%%%%%%%%%%%%%%%%%%%%%%%%%%%%%%%%%%%%%%%%%%%%%%%%%%%%%%%%%%%%%%
%% OXFORD THESIS TEMPLATE

% Use this template to produce a standard thesis that meets the Oxford University requirements for DPhil submission
%
% Originally by Keith A. Gillow (gillow@maths.ox.ac.uk), 1997
% Modified by Sam Evans (sam@samuelevansresearch.org), 2007
% Modified by John McManigle (john@oxfordechoes.com), 2015
%
% This version Copyright (c) 2015-2017 John McManigle
%
% Broad permissions are granted to use, modify, and distribute this software
% as specified in the MIT License included in this distribution's LICENSE file.
%

% I've (John) tried to comment this file extensively, so read through it to see how to use the various options.  Remember
% that in LaTeX, any line starting with a % is NOT executed.  Several places below, you have a choice of which line to use
% out of multiple options (eg draft vs final, for PDF vs for binding, etc.)  When you pick one, add a % to the beginning of
% the lines you don't want.


%%%%% CHOOSE PAGE LAYOUT
% The most common choices should be below.  You can also do other things, like replacing "a4paper" with "letterpaper", etc.

% This one will format for two-sided binding (ie left and right pages have mirror margins; blank pages inserted where needed):
\documentclass[a4paper,twoside]{ociamthesis}
% This one will format for one-sided binding (ie left margin > right margin; no extra blank pages):
%\documentclass[a4paper]{ociamthesis}
% This one will format for PDF output (ie equal margins, no extra blank pages):
%\documentclass[a4paper,nobind]{ociamthesis} 


%%%%% GENERIC TEXT
\usepackage{blindtext}


%%%%% SELECT YOUR DRAFT OPTIONS
% Three options going on here; use in any combination.  But remember to turn the first two off before
% generating a PDF to send to the printer!

% This adds a "DRAFT" footer to every normal page.  (The first page of each chapter is not a "normal" page.)
\fancyfoot[C]{\emph{DRAFT Printed on \today}}  

% This highlights (in blue) corrections marked with (for words) \mccorrect{blah} or (for whole
% paragraphs) \begin{mccorrection} . . . \end{mccorrection}.  This can be useful for sending a PDF of
% your corrected thesis to your examiners for review.  Turn it off, and the blue disappears.
\correctionstrue


%%%%% BIBLIOGRAPHY SETUP
% Note that your bibliography will require some tweaking depending on your department, preferred format, etc.
% The options included below are just very basic "sciencey" and "humanitiesey" options to get started.
% If you've not used LaTeX before, I recommend reading a little about biblatex/biber and getting started with it.
% If you're already a LaTeX pro and are used to natbib or something, modify as necessary.
% Either way, you'll have to choose and configure an appropriate bibliography format...

% The science-type option: numerical in-text citation with references in order of appearance.
\usepackage[style=phys, sorting=none, backend=biber, doi=false, isbn=false, natbib=true, citestyle=science, backref=true]{biblatex}
\newcommand*{\bibtitle}{References}
% defines the backreferencing string in the bibliography
\DefineBibliographyStrings{english}{%
  backrefpage = {p.},%
  backrefpages = {pp.}%
}

% The humanities-type option: author-year in-text citation with an alphabetical works cited.
%\usepackage[style=authoryear, sorting=nyt, backend=biber, maxcitenames=2, useprefix, doi=false, isbn=false]{biblatex}
%\newcommand*{\bibtitle}{Works Cited}

% This makes the bibliography left-aligned (not 'justified') and slightly smaller font.
\renewcommand*{\bibfont}{\raggedright\small}

% Change this to the name of your .bib file (usually exported from a citation manager like Zotero or EndNote).
\addbibresource{Introduction/Thesis-Introduction.bib}
\addbibresource{Ch2/Thesis-Ch2.bib}
\addbibresource{Ch3/Thesis-Ch3.bib}
\addbibresource{Ch4/Thesis-Ch4.bib}
\addbibresource{Ch5/Thesis-Ch5.bib}
\addbibresource{Discussion/Thesis-Discussion.bib}

% NJL: tidy up bibliography & allow custom notes (through the Zotero "note" interface) to be printed after entries

% NJL: these lines remove unwanted fields from the printed biliography
\AtEveryBibitem{\clearfield{issn}}
\AtEveryBibitem{\clearlist{issn}}
\AtEveryBibitem{\clearfield{language}}
\AtEveryBibitem{\clearlist{language}}
\AtEveryBibitem{\clearfield{note}}
\AtEveryBibitem{\clearlist{note}}

% NJL: adds in the annotation field
\DeclareFieldFormat{annotation}{\textbf{#1}}
\AtEveryBibitem{
\csappto{blx@bbx@\thefield{entrytype}}{\iffieldundef{annotation}{}{\par\printfield{annotation}}}
}

% Uncomment this if you want equation numbers per section (2.3.12), instead of per chapter (2.18):
%\numberwithin{equation}{subsection}



%%%%% THESIS / TITLE PAGE INFORMATION
% Everybody needs to complete the following:
\title{Reliable attribution and projection of extreme heat events altered by human influence on the climate}
\author{Nicholas J. Leach}
\college{St. Cross College}

% Master's candidates who require the alternate title page (with candidate number and word count)
% must also un-comment and complete the following three lines:
%\masterssubmissiontrue
%\candidateno{933516}
%\wordcount{28,815}

% Uncomment the following line if your degree also includes exams (eg most masters):
%\renewcommand{\submittedtext}{Submitted in partial completion of the}
% Your full degree name.  (But remember that DPhils aren't "in" anything.  They're just DPhils.)
\degree{Doctor of Philosophy}
% Term and year of submission, or date if your board requires (eg most masters)
\degreedate{Trinity 2022}

% Supervisors
\supervisorone{Antje Weisheimer}
\supervisortwo{Myles R. Allen}

%%%%% YOUR OWN PERSONAL MACROS
% This is a good place to dump your own LaTeX macros as they come up.

% To make text superscripts shortcuts
	\renewcommand{\th}{\textsuperscript{th}} % ex: I won 4\th place
	\newcommand{\nd}{\textsuperscript{nd}}
	\renewcommand{\st}{\textsuperscript{st}}
	\newcommand{\rd}{\textsuperscript{rd}}

% NJL: to format chapter references nicely
\usepackage{etoolbox}
\newcommand{\formatchref}[8]{%
	\noindent #1 %
	(#2). %
	\textbf{#3}. %
	\emph{#4}, %
	\ifstrempty{#5}{}{\textbf{#5}}%
	\ifstrempty{#6}{}{(#6)}%
	\ifstrempty{#7}{}{, #7}. %
	\ifstrempty{#8}{}{\url{#8}}%
	}

% NJL: to use tt font for urls
\urlstyle{tt}

% NJL: try to change spacing (works but messes up frontmatter)
\usepackage[]{setspace}
% \onehalfspacing

% NJL change font of "Appendices" title
% \let\appendixpagenameorig\appendixpagename
% \renewcommand{\appendixpagename}{\rmfamily\appendixpagenameorig}

% NJL: add in listings package for code snippets
\usepackage{listings}
\lstset{language=Python}

%%%%% THE ACTUAL DOCUMENT STARTS HERE
\begin{document}


%%%%% CHOOSE YOUR LINE SPACING HERE
% This is the official option.  Use it for your submission copy and library copy:
\setlength{\textbaselineskip}{22pt plus2pt}
% This is closer spacing (about 1.5-spaced) that you might prefer for your personal copies:
%\setlength{\textbaselineskip}{18pt plus2pt minus1pt}

% You can set the spacing here for the roman-numbered pages (acknowledgements, table of contents, etc.)
\setlength{\frontmatterbaselineskip}{17pt plus1pt minus1pt}

% Leave this line alone; it gets things started for the real document.
\setlength{\baselineskip}{\textbaselineskip}


%%%%% CHOOSE YOUR SECTION NUMBERING DEPTH HERE
% You have two choices.  First, how far down are sections numbered?  (Below that, they're named but
% don't get numbers.)  Second, what level of section appears in the table of contents?  These don't have
% to match: you can have numbered sections that don't show up in the ToC, or unnumbered sections that
% do.  Throughout, 0 = chapter; 1 = section; 2 = subsection; 3 = subsubsection, 4 = paragraph...

% The level that gets a number:
\setcounter{secnumdepth}{2}
% The level that shows up in the ToC:
\setcounter{tocdepth}{2}


%%%%% ABSTRACT SEPARATE
% This is used to create the separate, one-page abstract that you are required to hand into the Exam
% Schools.  You can comment it out to generate a PDF for printing or whatnot.
\begin{abstractseparate}
	Anthropogenic greenhouse gas emissions are now well-understood to be causing widespread and often damaging changes to the climate. One of the many ways in which the climate is changing is through the characteristics of extreme weather. Given the severe consequences that these devastating events can have, understanding how human influence on the climate is affecting them is of considerable importance. This is the aim of the relatively new field of `extreme event attribution'. Although there now exist a number of established methods for attributing individual weather events to climate change, ranging from probabilistic approaches with large climate model ensembles to very conditioned storyline approaches, questions still remain over the reliability of these approaches, especially when considering the most unprecedented events. We need to have confidence in this understanding in order to plan adaptation measures effectively to mitigate changes in risk into the future. In this thesis, I show how the models that we use to forecast the weather could provide us with such reliable information about how humanity is affecting extreme weather --- and in particular extreme heat. These models are state-of-the-art and can be shown to be unequivocally able to simulate the detailed physics of specific extreme weather through a successful prediction. Here I develop a perturbed initial and boundary condition approach using the European Center ensemble prediction system that aims to produce forecasts of individual events as if they had occurred in a world without human influence on the climate. These `counterfactual' forecasts can then be used to assess how not only the intensity, but also the probability of these events has changed. Although extreme weather attribution typically focuses on the past climate, the same approach can in theory be used to produce forecasts in possible warmer future worlds --- thus providing vital information about how the most damaging weather in the present may be expected to change in the future. In the final chapter, I depart from attribution but continue with this theme of extreme weather projection, examining a novel approach to producing large model ensembles that explore the range of uncertainty in future extreme weather. This work contrasts the very specific nature of the attribution of single weather events, but both approaches to extreme weather projection together could provide complementary and useful information for adaptation planning.  % Create an abstract.tex file in the 'text' folder for your abstract.
\end{abstractseparate}


% JEM: Pages are roman numbered from here, though page numbers are invisible until ToC.  This is in
% keeping with most typesetting conventions.
\begin{romanpages}

% Title page is created here
\maketitle

%%%%% DEDICATION -- If you'd like one, un-comment the following.
%\begin{dedication}
%This thesis is dedicated to\\
%someone\\
%for some special reason\\
%\end{dedication}

%%%%% ACKNOWLEDGEMENTS -- Nothing to do here except comment out if you don't want it.
\begin{acknowledgements}
 	I feel extremely privileged to have worked with an incredible group of co-authors and colleagues throughout my time at AOPP. I have learned a huge amount from all of you --- and possibly more importantly, I have had a fantastic time too. In particular, I would like to thank my supervisors, Myles and Antje, for supporting me throughout, ensuring that I never ran out of ideas, but most of all believing in me and pushing me to pursue the research that interested me the most.\\\\
\noindent Next I would like to recognise and thank all my co-authors:
\begin{itemize}[label={},leftmargin=*]
    \itemsep0em
    \item Tim Palmer, for many insightful discussions that ensured I understood the fundamentals and evolved my views over the past four years.
    \item David Sexton, for friendly supervision and enthusiasm during my work on Chapter 5 and beyond.
    \item Geert Jan van Oldenborgh, from whom I learnt an awful lot in a short space of time, and who is missed dearly by the whole community.
    \item Chris Roberts, for prompt and valuable feedback, and for knowing how to practically perturb and re-initialise IFS.
    \item Fraser Lott, for being incredibly helpful and friendly over email when I was getting started, and just as helpful and friendly in person.
    \item Peter Watson, for supportive and constructive advice on both science and career planning when I was stuck over what to do next.
    \item Sarah Sparrow, for enthusiastic and helpful discussions and feedback on the first and last science chapters.
    \item Dan Heathcote, for being a fantastic masters student.
    \item Dann Mitchell, for keen discussions and feedback on Chapter 4, and ensuring I think about real-world implications.
    \item Vikki Thompson, for interesting discussions on Chapter 4 and beyond.
    \item David Wallom, for technical support that ensured I had something to look at in Chapter 5.
\end{itemize}
Thanks also go to all the editors and reviewers that have improved both the science and writing in all the work that we carried out.\\\\
There are several others who have made this all possible that I would like to thank:
\begin{itemize}[label={},leftmargin=*]
    \itemsep0em
    \item Paul Dando, for priceless technical support and knowledge about IFS.
    \item Robin Hogan and Mat Chantry, for helping me to get IFS to do what I wanted.
    \item Richard Millar, for ensuring technical challenges didn't make me quit climate research in frustration before I'd really got started.
    \item Shirin Ermis, for helping me consider why attribution is important.
    \item Man-Suen Chan, for working tirelessly to make sure I (and everyone else in AOPP) have the computing resource to get any work done at all.
    \item Victoria Forth, Lucy Li, and Lewis Overs, for making sure my journey through both the DTP and AOPP was as smooth as possible.
    \item Heather Waller, for ensuring I met Myles regularly.
\end{itemize}
And all of the developers of the key software that I used throughout my work: \texttt{python}, \texttt{numpy}, \texttt{scipy}, \texttt{matplotlib}, \texttt{jupyter}, \texttt{xarray}, \texttt{pandas}, \texttt{seaborn} and \texttt{cdo}.\\\\
I am very grateful to the organisations that have funded my work over the last four years: the Natural Environmental Research Council for my stipend and research grant, and St Cross College for my fees and travel grants.\\\\
I consider myself immensely lucky to have such amazing and supportive friends. Stuart, Beth and Fraser, you have made sure that even when the science isn't going entirely to plan, I was still able to have a laugh. I will remember the time I spent doing my PhD incredibly fondly, and a lot of that is down to you. Davy, Fanners and Ellen, you've had to put up with me going on about my various PhD woes for the past four years, and yet still seem to want to spend time with me. There's no one else I would rather run for 9 hours straight with. And to all my other friends who have spent time with and supported me throughout this PhD: I am truly grateful. It would not have been anywhere near as enjoyable without all of you.\\\\
I would like to thank my family: mum, dad, and Matt; for always being there for me when I need you, and encouraging and helping me to do whatever it is that I want to do.\\\\
Finally, I would like to thank Charlotte for constantly supporting and inspiring me throughout the time we've been together: I couldn't have asked for more.
\end{acknowledgements}

%%%%% ABSTRACT -- Nothing to do here except comment out if you don't want it.
\begin{abstract}
	Anthropogenic greenhouse gas emissions are now well-understood to be causing widespread and often damaging changes to the climate. One of the many ways in which the climate is changing is through the characteristics of extreme weather. Given the severe consequences that these devastating events can have, understanding how human influence on the climate is affecting them is of considerable importance. This is the aim of the relatively new field of `extreme event attribution'. Although there now exist a number of established methods for attributing individual weather events to climate change, ranging from probabilistic approaches with large climate model ensembles to very conditioned storyline approaches, questions still remain over the reliability of these approaches, especially when considering the most unprecedented events. We need to have confidence in this understanding in order to plan adaptation measures effectively to mitigate changes in risk into the future. In this thesis, I show how the models that we use to forecast the weather could provide us with such reliable information about how humanity is affecting extreme weather --- and in particular extreme heat. These models are state-of-the-art and can be shown to be unequivocally able to simulate the detailed physics of specific extreme weather through a successful prediction. Here I develop a perturbed initial and boundary condition approach using the European Center ensemble prediction system that aims to produce forecasts of individual events as if they had occurred in a world without human influence on the climate. These `counterfactual' forecasts can then be used to assess how not only the intensity, but also the probability of these events has changed. Although extreme weather attribution typically focuses on the past climate, the same approach can in theory be used to produce forecasts in possible warmer future worlds --- thus providing vital information about how the most damaging weather in the present may be expected to change in the future. In the final chapter, I depart from attribution but continue with this theme of extreme weather projection, examining a novel approach to producing large model ensembles that explore the range of uncertainty in future extreme weather. This work contrasts the very specific nature of the attribution of single weather events, but both approaches to extreme weather projection together could provide complementary and useful information for adaptation planning. 
\end{abstract}

%%%%% MINI TABLES
% This lays the groundwork for per-chapter, mini tables of contents.  Comment the following line
% (and remove \minitoc from the chapter files) if you don't want this.  Un-comment either of the
% next two lines if you want a per-chapter list of figures or tables.
\dominitoc % include a mini table of contents
%\dominilof  % include a mini list of figures
%\dominilot  % include a mini list of tables

% This aligns the bottom of the text of each page.  It generally makes things look better.
\flushbottom

% This is where the whole-document ToC appears:
\tableofcontents

\listoffigures
	\mtcaddchapter
% \mtcaddchapter is needed when adding a non-chapter (but chapter-like) entity to avoid confusing minitoc

% Uncomment to generate a list of tables:
%\listoftables
%	\mtcaddchapter

%%%%% LIST OF ABBREVIATIONS
% This example includes a list of abbreviations.  Look at text/abbreviations.tex to see how that file is
% formatted.  The template can handle any kind of list though, so this might be a good place for a
% glossary, etc.
% First parameter can be changed eg to "Glossary" or something.
% Second parameter is the max length of bold terms.
\begin{mclistof}{List of Abbreviations}{3.2cm}

    \item[CDF] Cumulative distribution dunction
    \item[CI] Confidence interval 
    \item[CO$_2$] Carbon dioxide
    \item[DJF] December--January--February (meteorological winter)
    \item[ECMWF] European Centre for Medium-range Weather Forecasts
    \item[EEA] Extreme event attribution
    \item[FAR] Fraction of attributable risk
    \item[GEV] Generalised extreme value
    \item[GMST] Global mean surface temperature
    \item[GSAT] Global surface air temperature
    \item[JJA] June--July--August (meteorological summer)
    \item[PDF] Probability density function
    \item[PPE] Perturbed parameter ensemble
    \item[SIC] Sea ice concentration
    \item[SST] Sea surface temperature
    \item[UK] United Kingdom
    \item[UKCP] UK Climate Projections
    \item[UKMO] UK Met Office
    \item[WWA] World Weather Attribution

\end{mclistof}

% The Roman pages, like the Roman Empire, must come to its inevitable close.
\end{romanpages}

% NJL: this sets the spacing in the main text
% \onehalfspacing
\doublespacing

%%%%% CHAPTERS
% Add or remove any chapters you'd like here, by file name (excluding '.tex'):
\flushbottom
\begin{savequote}[8cm]
    Quote
      \qauthor{--- author}
\end{savequote}
    
\chapter{\label{intro}Introduction} 

In this chapter I introduce the problem of attribution of individual extreme weather events to anthropogenic climate change. I review the current methodologies and frameworks that address this problem, in particular the contrasting storyline and probabilistic approaches to attribution. Although these frameworks are gaining acceptance and maturity, I suggest that a weather forecast-based approach could further increase the trustworthiness of attribution studies. Finally, I provide a conceptual sketch of these various attribution frameworks within a simple non-linear dynamical system.
\small\paragraph{Author contributions:} This chapter is based on the the following publication \footnote{with the author contributing as follows.} \par\vspace{1em}
\formatchref{Surname, I1. I2., Surname, I1. I2.}{year}{Title}{Journal}{vol}{issue}{pages}{DOI}

\minitoc

\clearpage

\section{The problem of extreme event attribution}

  % detection and attribution of climate change (hasselmann->gillet)
  % Being specific about the question ("cause" vs influence, hannart)
  % scientific challenges posed by EEA
  Review papers: \citep{allen_scientific_2007,stott_attribution_2013,stott_attribution_2016,otto_attribution_2016,otto_attribution_2017,swain_attributing_2020,easterling_detection_2016,noauthor_attribution_2016}

\section{Motivating the question}

  Now that I have posed the question, before I move on to how we might answer it, I think it would be useful for me to discuss why we want to answer it. In short: \emph{what's the point of this thesis?}

  % start off with liability (allen03)
  In 2003, Myles Allen wrote \citetitle{allen_liability_2003} \citep{allen_liability_2003}. This commentary is widely acknowledged as the first time the idea that individual extreme events could be attributed to external drivers such as human influence was proposed \citep{otto_attribution_2017}. Although \citeauthor{allen_liability_2003} touched on both methodology and motivation for extreme event attribution, here I shall focus on the latter aspect. The motivation behind extreme event attribution as proposed in \citetitle{allen_liability_2003} is compensation for damage to individuals caused by climate change or, as \citeauthor{allen_liability_2003} puts it, 
  \begin{quote}
    Will it ever be possible to sue anyone for damaging the climate?
  \end{quote}
  \citeauthor{allen_liability_2003} suggests that in the future those affected by particular extreme weather may, given sufficient scientific certainty, be able to claim compensation from greehouse gas emitters for damages caused by the extreme weather. He proposes a framework, grounded in concepts from epidemiology \citep{stone_end--end_2005}, in which emitters pay for the `fraction' of an extreme weather event that they caused, even in the abscence of absolute causation. This fraction is estimated probabilistically based on the change in likelihood of the event in a world in which the emissions never happened (ie. if the event is half as likely to occur without the emissions, then the fraction of the event that is attributable to the emitters is 50\%).
  This specific application is therefore using extreme event attribution as evidence in environmental tort law. Since \citet{allen_liability_2003}, much has been written in relation to this application. \citet{allen_scientific_2007} presents an overview of the state of climate change detection and attribution aimed at legal professionals, concluding with a set of related questions for the legal community. More recently, \citet{stuart-smith_filling_2021} provided a set of suggestions for potential plaintiffs on how to best make use of the climate science available (noting that evidence used in previous cases `lags substantially behind the state of the art'). Coming from the other side of the coin, \citet{marjanac_acts_2017} provide suggestions for climate change scientists, emphasising that `clear and confident expression of science in a manner that can be applied by non-scientists, including lawyers' is key. Elisabeth A. Lloyd has authored a number of studies exploring various issues including the different standards of proof in scientific and legal contexts \citep{lloyd_climate_2021}; how different approaches to extreme event attribution can complement one another to provide the most useful picture of climate change impacts for a broad range of contexts \citep{lloyd_climate_2018,lloyd_environmental_2020}; and finally, examines a specific tort law case that made use of extreme event attribution \citep{lloyd_climate_2021-1}. For thorough review of both the science and legal context of extreme event attribution, written from a legal perspective, with reference to specific cases, I recommend \citet{burger_law_2020,marjanac_extreme_2018}.
  
  In addition to the original application in tort law suggested in \citet{allen_liability_2003}, more recently it has been suggested that extreme event attribution could `play a significant roll in quantifying loss and damage' \citep{wehner_operational_2022}. Loss and damage is generally understood as the unavoidable adverse impacts of climate change \citep{mace_loss_2016}, and has become a key piece of international climate change policy since the inclusion of Article 8 of the Paris Agreement. Several recent extreme event attribution based studies may have considerable influence in the future in this space, including \citet{clarke_inventories_2021}, who set out a framework for recording key details of high-impact weather events as a new source of evidence for global stocktakes on loss and damage; and \citet{otto_assigning_2017,lott_quantifying_2021}, who adapt conventional extreme event attribution approaches to estimate the contributions of \emph{specific} emitters to individual extreme weather events (as opposed to the more usual broad `human influence' considered). Perhaps we don't have long to wait before the question posed by \citeauthor{allen_liability_2003} quoted above is answered?

  The other key nonscientific motivating factor for extreme event attribution is the public engagement and interest in the research  \citep{swain_attributing_2020}. The `headline' number in climate science has for a long time been change in global mean temperature \citep{stocker_climate_2013,ipcc_global_2018,masson-delmotte_climate_2021}. While this is clearly a very important number as the primary metric of the impact that humanity is having on the climate, it is not a number that individuals can easy relate to due to the large scales involved and indirect nature of the associated impacts. On the other hand, extreme weather events are phenomena that are actually experienced in real time by individuals --- and regularly reported on by the media. Since extreme weather events can cause severe and direct socioeconomic impacts \citep{fouillet_excess_2006}, increases in their frequency would likely be a considerably more relatable and concerning consequence of climate change than the distant change in global mean quantities. Previous work has shown that extreme event attribution may be an exceptionally useful tool for climate change communication \citep{ettinger_whats_2021}, though can prove unhelpful if results are not clear and comprehensible for a general audience \citep[for example if different attribution studies regarding a single event appear to provide conflicting headline results][]{osaka_natural_2020}. A specific study investigating the experience-perception link of climate disaters in the context of Floridians that had experienced hurricane Irma found that this experience increased both their belief and concern in global warming \citep{bergquist_experiencing_2019}, though a more recent study looking at connections between local weather and climate change awareness in Germany did not find a link \citep{gartner_experiencing_2021}.

  % improve models (high benchmark of test) - decide whether to include
  \citep{sillmann_understanding_2017}

\section{Answering the question}

  At this point, I have discussed the question that this thesis is primarily concerned with, and the reasons why it is of broad importance. Now, I shall describe and explore the ways in which previous studies have approached this question, in particular focusing on the probabilistic \citep[often ``conventional'',][]{stott_human_2004} and storyline \citep{hoerling_anatomy_2013} frameworks. 

  \subsection{Probabilistic attribution}

    Probabilistic attribution seeks, ultimately, to determine the change in probability of an extreme event arising due to some external driver. This was the approach to extreme event attributon proposed by \citet{allen_liability_2003} and first applied by \citet{stott_human_2004} to the 2003 European heatwave. They applied an optimal fingerprinting technique \citep{hasselmann_optimal_1993,hasselmann_multi-pattern_1997} to transient climate model simulations. They used five simulations, one set of four with all forcings included, plus one with natural forcings only; generating scaling factors of the correspondence between the modelled response and observed changes by regressing each set onto observed central European summer temperature. The scaling factors could then be used to determine the 1990s temperature anomalies attributable to all forcings combined and natural forcings alone. A third control run at a fixed, pre-industrial level of forcing was used to estimate internal variability corresponding likelihood functions for these temperature anomalies. They finally used a peak-over-threshold extreme value analysis of the same control run to determine the probability of exceeding the temperature of the pre-2003 hottest summer in worlds with and without climate change by adjusting the mean summer temperature to the estimated 1990s temperature anomalies both with and without anthropogenic forcing. These probabilities (and associated uncertainty) could be used to determine the likelihood function of the change in risk of the heatwave attributable to human influence. This fairly involved statistical approach (in particular, the necessity for the use of a control run to estimate uncertainty due to internal variability) has been largely replaced by the use of much larger single- or multiple- model ensembles.

    The next advance in probabilistic extreme event attribution came with \citet{van_oldenborgh_how_2007}, who developed a methodology for estimating the change in risk of an extreme event using observations alone. \citeauthor{van_oldenborgh_how_2007} took observed timeseries of autumn temperatures measured by the De Bilt weather station, and removed the climate change signal by regression onto low-pass filtered global mean surface temperature (a reasonable proxy for anthropogenic influence on the climate, given the small contributions from natural forcing). Using this detrended series and extreme value analysis, he then computed the return period of the exceptional warm 2006 autumn in Europe, and compared it to the return period estimated using the original series. This method was later extended to use the return period to compute the return period of the extreme for both the present-day and pre-industrial period by shifting the detrended series by the attributable trend computed in the regession \citep[eg.][]{philip_protocol_2020,leach_anthropogenic_2020}. In this way, the change in risk between pre-industrial and present climates can be estimated. It is worth noting that this methodology does not formally attribute and changes in risk to anthropogenic influence, since trends in local climate may be influenced by other factors, and no use is made of a counterfactual world without human influence on the climate (since no observations of such a world exist). This method can also be applied to transient simulations from climate models.

    The final advance that I shall highlight is from \citet{pall_anthropogenic_2011}, and was the first instance where specific fixed forcing (as opposed to transient) factual and counterfactual simulations were used. \citeauthor{pall_anthropogenic_2011} generated very large (2000+ member) atmosphere-only climate model ensembles of autumn 2000. One ensemble was driven using observed sea surface temperatures and sea ice, and corresponding atmospheric conditions (greenhouse gas, aerosol and ozone concentrations) for that time. The other four were driven using atmospheric conditions for the year 1900, and subtracting four estimates of attributable twentieth-century SST warming from the observed sea surface temperatures; the four estimates of attributable warming were derived from four different coupled climate model simulations. River runoff for England and Wales in the factual ensemble was compared to runoff in the four `naturalised' counterfactual ensembles to determine the difference in risk of exceeding the value actually observed in autumn 2000 in the different climates. In this case, the ensembles are sufficiently large that extreme value analysis was not required (the -- SST conditional -- return period could be calculated by simply counting the number of members that exceeded the observed threshold and dividing by the total ensemble size in each case). This methodology is attractive due to the clear and straightforward statistical analysis of the large ensembles (with no reliance on extreme value analysis or optimal fingerprinting techniques). However, it does generally require very large ensembles of the time period when the event took place (ie. autumn 2000 in this case); and is conditional on the prescribed SST pattern (requiring either that anthropogenic influence isn't affecting interannual modes of SST variability, or that the extreme in question is independent of such modes).

    The `standard' approach to probabilistic attribution in the present draws upon each of these previous advances \citep{philip_protocol_2020}. Here I am taking the World Weather Attribution project (WWA) methodology as standard, since they are the most prolific group both in terms of number of events analysed and media coverage of their results \citep{van_oldenborgh_pathways_2021}. Their approach involves:
    \begin{enumerate}
      \item Defining the event. Extreme events are, by the nature of the weather, exceptionally high dimensional and could be described in a practially unlimited number of ways (ie. what variables to use? What spatial scale? What temporal scale?). However, to be able to analyse changes to such events, we must be able to define them quantatively. The WWA attempts to select a metric that most closely corresponds to the impacts of the extreme event in question, taking in account what questions are being asked by the various stakeholders. For example, if the key impact of interest is heatwave-associated mortality, then peak 3-day moving average daily maximum temperatures may be selected due to their close connection to health impacts \citep{dippoliti_impact_2010}. Once a metric has been chosen by which to define the extreme event, they use a `class-based' framing considering all events of a similar magnitude, often by choosing the annual maximum values of the metric. This class-based framing results in a largely unconditional analysis, which I will discuss further below.
      \item Analysing trends in observed data following \citet{van_oldenborgh_how_2007}. This is typically done by fitting an appropriate distribution whose parameters shift or scale with low-pass filtered global mean surface temperature (GMST). The shifting or scaling depends upon the chosen metric and its observed or expected response to climate change. For example, the temperature based heatwave metrics that are the primary concern of this thesis are typically assumed to simply shift with global warming. From this GMST-covarying distribution, the return period is then computed for present-day and pre-industrial values of GMST. From these returns periods the change in risk of the observed extreme can be calculated. As with \citet{van_oldenborgh_how_2007}, these changes in risk are not strictly attributable to human influence on the climate due to the lack of a no-human counterfactual.
      \item Analysing simulations from as many models as possible. Transient model simulations are analysed in an identical way to observations. Fixed forcing simulations are analysed following \citet{pall_anthropogenic_2011}. Only models which are able to closely represent the observed climate are considered -- evaluated on the basis of the trends and distribution parameters implied by the model data.
      \item Synthesising these various strands of evidence. The aim behind using as many lines of evidence as possible, combining statistical and numerical models, is to try and determine the most robust conclusion possible within the context of the associated uncertainties. 
    \end{enumerate}
    This `standard' approach has been refined over the past decade by the WWA team \citep{van_oldenborgh_pathways_2021}, learning through application to a wide range of locations and types of extreme. The widespread recognition and understanding of extreme event attribution by the general public is due, in no small part, to this approach and how rapidly the WWA team have been able to apply it to extreme events in the past few years. Their rapid response has meant that they are able to answer the questions people ask in the aftermath of such events when they are actually asking them -- rather than following a lengthy peer-review process. However, this standard probabilistic approach to attribution is not without issues of its own -- hence this thesis -- which I shall now discuss.

    The first issue arises due to the unconditional use of climate models. By their nature, extreme events are typically produced by exceptional physical processes, or combinations of processes. The type of models used in extreme event attribution are typically coarse (O(100 km)), and may well not be able to physically represent all the processes involved in the production of specific extreme events even if they can accurately represent the average climate \citep{sillmann_understanding_2017,trenberth_attribution_2015}. Such models still have serious known biases relevant to the simulation of extremes, including poor representation of Euro-Atlantic blocking \citep{schiemann_resolution_2017,dorrington_how_2021}, which is a key synoptic driver of heatwaves over the continent. These biases become even more important when considering not only the models' ability to simulate the present climate, but also their response to external forcings such as anthropogenic climate change \citep{palmer_nonlinear_1999,palmer_simple_2018}. The use of biased models can lead to potentially incorrect quantitative attribution statements \citep{bellprat_attribution_2016,bellprat_towards_2019}. 
    
    The second key issue derives from the treatment of individual extremes as one of an event class. For example, in the conventional probabilistic approach to attributing a particular heatwave, one might consider all the previous annual maximum temperatures (eg. in order to apply extreme value analysis as discussed above). However, the particular heatwave in question might have arisen from a very different -- possibly unique in the context of the relatively short historical record -- set of meteorological circumstances and drivers compared to all the previous heatwaves. This not only renders such extreme value analysis as is often performed potentially inappropriate (as the heatwave in question is drawn from a different underlying distribution to the others), but also any estimated climate change responses. What if the combination of the particular processes involved in producing the heatwave in question responds fundamentally differently to the processes that have generated past heatwaves?
    
    The final issue I shall discuss is the one that has been most often commented on in previous work: the risk of type II errors \citep{shepherd_common_2016,trenberth_attribution_2015}. This risk arises because some aspects of the climate system response to external forcing are much more certain and well-understood than others. For example, the thermodynamic aspects of climate change are broadly very well understood and certain: rising greenhouse gas concentrations lead to a thicker troposphere, thus raising surface temperatures and increasing the moisture capacity of the troposphere. On the other hand, the dynamic aspects of climate change are considerably less certain, with models often disagreeing over the direction of changes in atmospheric dynamics \citep{masato_winter_2013}. At this point, I note that this is not an entirely independent issue to the first issue discussed since much of this uncertainty arises due to model biases and imperfections. Since extreme events are often driven by a combinations of both thermodynamic and dynamic processes, the very certain thermodynamic climate change impacts can be masked to some extent by the much less certain dynamic climate change impacts. This uncertainty can lead to falsely asserting that there is no overall impact -- a type II error. \citet{trenberth_attribution_2015} argue that it is better to focus on the aspects of the event that are well understood, for example by conditioning analyses upon the large scale circulation of the event in question, thus removing the potentiall very uncertain dynamic aspects of climate change. This suggestion was extended and discussed at length by \citet{shepherd_common_2016}, and has become the basis for the new kid on the block in extreme weather attribution: the storyline approach.

  \subsection{Attribution through storylines}

    The storyline approach \citep[or `Boulder' approach,][]{otto_attribution_2017} aims to determine the contribution of various causal factors that to the extreme event, and considers how anthropogenic climate change has affected those factors (and thus the extreme) in a deterministic manner. This approach was first applied by \citet{hoerling_anatomy_2013} to the 2011 summer combined Texas heatwave and drought. They used a variety of simulations, including atmosphere-only and coupled climate model runs, and seasonal forecasts. They examined the influence of rainfall deficit in the months preceding the heatwave, SST influence on the drought, and the overall predictability of the extreme at the start of May. As such, their intended goal was much broader than just assessing the anthropogenic contribution to the heatwave, aiming to advance the overall predictability of such events by examining all causes, human and natural. 
    % the aims of storyline approaches
    % Hoerling - hazelger - shepherd - benitez
    \citep{hazeleger_tales_2015,shepherd_common_2016,benitez_july_2022}
    % issues: does not provide estimate of change in risk

  \subsection{Do we need new approaches?}

  \subsection{Operationalising attribution}

\section{Conceptualising different approaches}

  % use the dynamicist's favourite "toy": Lorenz63
  % demonstrate "storyline / DADA", conventional attribution
  % propose forecast-based attribution
  \blindtext

\section{What to expect in this thesis}

  % exploratory conventional attribution study 
  % tangent study about climate change projections + link to forecast based approaches
  % first step to forecast-based attribution - partial attribution
  % "full" forecast-based attribution
  % conclusion discussing current limitations + future directions
  \blindtext

\begin{savequote}[8cm]
    Quote
      \qauthor{--- author}
\end{savequote}
    
\chapter{\label{ch2}Attribution and projection} 

Chapter description.
\small\paragraph{Author contributions:} This chapter is based on the the following publication \footnote{with the author contributing as follows. Data curation, Formal analysis, Investigation, Methodology, Visualization and Writing -- original draft.} \par\vspace{1em}
\formatchref{Leach, N. J., Watson, P. A. G., Sparrow, S. N., Wallom, D. C. H., \& Sexton, D. M. H.}{2022}{Generating samples of extreme winters to support climate adaptation}{Weather and Climate Extremes}{36}{}{100419}{https://doi.org/10.1016/j.wace.2022.100419}

\minitoc

\clearpage

\section{Section}

    \blindtext

\begin{savequote}[8cm]
    Quote
      \qauthor{--- author}
\end{savequote}
    
\chapter{\label{ch3}Forecast-based attribution: perturbing the boundary conditions}

This chapter contains much of the conceptual description of, and motivation for, forecast-based attribution. Using the well-predicted February 2019 heatwave as a case study, I carry out forecasts with the operational medium-range ECMWF model in which I have instantaneously perturbed the CO$_2$ concentration at initialisation. These perturbed forecasts allow me to estimate the direct contribution of diabatic heating due to CO$_2$ to the heatwave. This partial attribution provides a proof-of-concept of the forecast-based approach, and I close with a discussussion of how I could perform a more complete estimate of anthropogenic influence on a specific extreme event in following work.
\small\paragraph{Author contributions:} This chapter is based on the the following publication \footnote{with the author contributing as follows. Conceptualisation, Data curation, Formal analysis, Investigation, Methodology, Resources, Visualisation and Writing -- original draft} \par\vspace{1em}
\formatchref{Leach, N. J., Weisheimer, A., Allen, M. R., \& Palmer, T.}{2021}{Forecast-based attribution of a winter heatwave within the limit of predictability}{Proceedings of the National Academy of Sciences}{118}{49}{}{https://doi.org/10.1073/pnas.2112087118}

\clearpage

\minitoc

\clearpage

\section{Chapter open}

\section{Abstract}

  Attribution of extreme weather events has expanded rapidly as a field over the past decade. However, deficiencies in climate model representation of key dynamical drivers of extreme events have led to some concerns over the robustness of climate model-based attribution studies. It has also been suggested that the unconditioned risk-based approach to event attribution may result in false negative results due to dynamical noise overwhelming any climate change signal. The “storyline” attribution framework, in which the impact of climate change on individual drivers of an extreme event is examined, aims to mitigate these concerns. Here we propose a novel methodology for attribution of extreme weather events using the operational ECMWF medium-range forecast model that successfully predicted the event. The use of a successful forecast ensures not only that the model is able to accurately represent the event in question; but also that the analysis is unequivocally an attribution of this specific event, rather than a mixture of multiple different events that share some characteristic. Since this attribution methodology is conditioned on the component of the event that was predictable at forecast initialisation, we show how adjusting the lead time of the forecast can flexibly set the level of conditioning desired. This flexible adjustment of the conditioning allows us to synthesise between a storyline (highly-conditioned) and a risk-based (relatively unconditioned) approach. We demonstrate this forecast-based methodology through a partial attribution of the direct radiative effect of increased CO$_2$ concentrations on the exceptional European winter heatwave of February 2019.

\section{Introduction}

  Attribution of extreme weather events is a relatively young field of research within climate science. However, it has expanded rapidly from its conceptual introduction \citep{allen_liability_2003} over the past twenty years; it now has an annual special issue in The Bulletin of the American Meteorological Society \citep{peterson_explaining_2012}. Extreme event attribution is of particular importance for communicating the impacts of climate change to the public \citep{hulme_attributing_2014,hassol_natural_2016}, since the changing frequency of extreme weather events due to climate change is an impact that is physically experienced by society. As a result of this rapid expansion, there now exists a large number of different methodologies for carrying out an event attribution \citep{herring_explaining_2021}. Many of these rely on large ensembles of climate model simulations, the credibility of which has been questioned by recent studies \citep{bellprat_attribution_2016,bellprat_towards_2019,palmer_simple_2018}. A particular issue is the dynamical response of the atmosphere to external forcing, which is highly uncertain within these models \citep{shepherd_common_2016}. As attribution studies try to provide quicker results, with an operational system a clear aim, it is vital that any such system provides trustworthy results. In this study we propose a “forecast-based” attribution methodology using medium-range weather forecasts which could provide several key advantages over traditional climate model-based approaches. Firstly, if an event is predictable within a forecasting system, we know that that system is capable of accurately representing the event. Secondly, we know that any attribution performed is unequivocally an attribution of the specific event that occurred; unlike in unconditioned climate model simulations. Finally, weather forecasts are run routinely by many different national and research centers. The models used are generally state-of-the-art and extensively verified. We propose that the attribution community could and should take advantage of the massive amount of resources that are put into these forecasts by developing methodologies that use the same type of simulation. Ideally, the experiments required for attribution with forecast models would be able to be run with little additional effort on top of the routine weather forecasts; in this way they might provide a rapid operational attribution system. We discuss these ideas further throughout the text.

  There have been several studies that propose or perform methodologies related to the forecast-based attribution demonstrated here. \citet{hoerling_anatomy_2013} used two seasonal forecast ensembles to examine the predictability of the 2011 Texas drought/heatwave within a comprehensive attribution analysis involving several different types of types of climate simulation. \citet{meredith_crucial_2015} used a triply nested convection-permitting regional forecast model to investigate the role of historical SST warming within an extreme precipitation event. They conditioned their analysis on the large-scale dynamics of the event through nudging in the outermost domain. More recently, \citet{van_garderen_methodology_2021} employed spectrally nudged simulations to assess the contribution of human influence on the climate over the 20th century on the 2003 European and 2010 Russian heatwaves. Possibly the most similar studies to the one presented here are a series of studies by Hope and colleagues \citep{hope_contributors_2015,hope_what_2016,hope_determining_2019}. They used a seasonal forecast model to assess anthropogenic CO$_2$ contributions to record-breaking heat and fire weather in Australia. Two more similar studies carried out forecast-based hurricane attribution studies \citep{reed_forecasted_2020,lackmann_hurricane_2015}. Tropical cyclones are a natural candidate for forecast-based methodologies due to the high model resolution required to represent them accurately, if at all. A final distinct, but related study is \citet{hannart_dada_2016}, which proposes the use of Data Assimilation for Detection and Attribution (DADA). They suggest that operational causal attribution statements could be made in a computationally efficient manner using the kind of data assimilation procedure carried out by weather centers (to initialise forecasts) to compute the likelihood of a particular weather event under different forcings (these would be observed and estimated pre-industrial forcings for conventional attribution). Our forecast-based framework differs from these other studies in several regards. Firstly, we use a state-of-the-art forecast model to perform the attribution analysis of the event in question; rather than to solely assess the predictability of the event. We use free-running coupled global integrations here, allowing the predictable component at initialisation to dynamically condition the ensemble; as opposed to nudging our simulations towards the dynamics of the event, using nested regional simulations, or using the highly observationally constrained output of data assimilation procedures. A final key difference is that here we present an attribution of the direct radiative effect of CO$_2$ in isolation, though we hope that our approach could be extended in the future to provide an estimate of the full anthropogenic contribution to extreme weather events as in these other studies. We argue that the relative simplicity in the validation, setup and conditioning of our simulations is desirable from an operational attribution perspective; and flexible across many different types of extreme event. 

  We begin by introducing the chosen case study, the 2019 February heatwave in Europe, describing its synoptic characteristics and formally defining the event quantitatively. We then demonstrate the predictability of the event within the ECMWF ensemble prediction system, showing that this operational weather forecast was able to capture both the dynamical and thermodynamical features of the event. In \hyperref[Ch4:experiments]{Perturbed CO$_2$ forecasts}, we outline the experiments we have performed in order to quantitatively determine the direct CO$_2$ contribution to the heatwave. We then provide quantitative results from these experiments, and finally conclude with a discussion of the strengths and potential issues of our forecast-based attribution methodology, including our proposed directions for further work.

\section{The 2019 February heatwave in Europe}\label{Ch4:heatwave}

  Between the 21st and 27th February 2019, climatologically exceptional warm temperature anomalies of 10-15 \degree C were experienced throughout Northern and Western Europe \citep{young_record-breaking_2020}, as shown in Fig. $1A$. In particular, the 25th - 27th February saw record-breaking temperatures measured at many weather stations and over wide areas of Iberia, France, the British Isles, the Netherlands, Germany and Southern Sweden, as shown in Fig. $1C$ \citep{cornes_ensemble_2018}. Fig. $1D$, comparing the regional mean maximum temperatures during the 2019 heatwave with timeseries of winter mean maximum temperatures between 1950 and 2018, illustrates just how unusual and widespread the event was. This heat was associated with a characteristic flow pattern: a narrow titled ridge extending from north-west Africa out to the southern tip of Scandinavia, advecting warm subtropical air north-east \citep{sousa_european_2018}, as shown in the geopotential height field in Fig. $1A$. This dynamical driver was accompanied by another synoptic feature that further enhanced the warming: widespread clear skies between the 25th - 27th, shown in Fig. $1B$. These clear skies resulted in a widespread and persistent strong diurnal cycle, reaching 20 \degree C in some locations. Further details of the meteorological mechanisms and historical context of the heatwave are provided in refs. \citep{young_record-breaking_2020,kendon_temperature_2020,christidis_extremely_2021}.
  
  In order to quantify the direct impact of CO$_2$ on the heatwave in question within this study, we need to characterise the heatwave in an ``event definition''. The choice of event definition is subjective but can impact on the quantitative results of an attribution study significantly \citep{leach_anthropogenic_2020,uhe_comparison_2016,kirchmeieryoung_importance_2019}. The most remarkable feature of the February 2019 heatwave were the maximum temperatures observed, which peaked between the 25\textsuperscript{th} and 27\textsuperscript{th} for the majority of the affected area. Focusing on this relatively short time-period ensures that the synoptic situation driving the heat is coherent throughout the event definition window. For the spatial extent of the event, we use the eight European sub-areas described in ref. \citep{christensen_summary_2007}. The use of regions previously defined in the literature aims to avoid selection bias. Our resulting event definition is as follows: the hottest temperature observed between 2019-02-25 and 2019-02-27, then averaged over the land points within each region (the temporal maximum is calculated before the spatial averaging). Although we carry out our calculations for all sub-areas, several regions were characteristically very similar in terms of both the event itself, and the forecasts of the event. We therefore focus on three of the eight regions: the British Isles (BI), which experienced exceptional heat and was well predicted; France (FR), which experienced exceptional heat but where the magnitude of the heat was less well forecast; and the Mediterranean (MD), which experienced well-predicted but climatologically average heat.

  \clearpage
  \begin{figure}[h]
    \centering
    \includegraphics[width=0.65\textwidth]{{Fig3.1}.pdf}
    \caption[The 2019 February heatwave in Europe: synoptic characteristics \& historical context.]{\textbf{The 2019 February heatwave in Europe: synoptic characteristics \& historical context.}}
  \end{figure}
  \clearpage

  \subsection{Forecasts of the heatwave}\label{Ch4:forecasts}

    This heatwave was well-predicted by the European Centre for Medium-Range Weather Forecasts (ECMWF) ensemble prediction system. Their forecasts indicated ``extreme" heat was possible at a lead time of around two weeks, and probable at a lead time of around ten days (Fig. $2A$), despite the exceptional nature of the heatwave in both the model climatology and real world. As expected, the forecast's performance in predicting the extreme heat at the surface is reflected in variables more closely linked to the dynamic drivers of the heat, such as 500 hPa geopotential height (Fig. $2B$). 
    
    This successful forecast is a crucial part of our study as it means that we are not only confident that the model used is able to simulate the event in question; but that we are unequivocally performing an attribution analysis of the specific winter heatwave that occurred in Europe during February 2019. This is an important distinction to the framework used in ``conventional'' or ``risk-based'' \citep{shepherd_common_2016} attribution studies \citep{stott_human_2004,pall_anthropogenic_2011,sparrow_attributing_2018,leach_anthropogenic_2020}, which in general reduce the event to some impact-relevant quantitative index , then estimate the increase in likelihood of events that exceed the magnitude of the event in question. For example, a heatwave attribution study may choose to define the event as the hottest observed temperature during the heatwave, and then compute the attributable change in likelihood of temperatures hotter than this recorded maximum (eg. using models or historical records). While this does provide useful information, it does not answer the question of how much more likely anthropogenic activities have made the \emph{specific} heatwave that occurred, rather the question of how much more likely anthropogenic activities have made a mixture of events that share one or more characteristics. Studies have attempted to provide a more satisfactory answer to this first question by including a level of conditioning on the set of events considered by using circulation analogues \citep{yiou_statistical_2017}, or by nudging model simulations towards the specific dynamical situation that occurred during the event in question \citep{meredith_crucial_2015,van_garderen_methodology_2021}. Here we are evidently performing an attribution study of the specific record-breaking heatwave that occurred in February 2019 due to the use of these successful forecasts, that not only captured the heat experienced at the surface, but also the dynamical drivers behind the heat.

    As well as enabling us to answer the attribution question for a single specific heatwave, the use of a numerical weather prediction model provides additional benefits. Since large model ensembles are required to properly capture the statistics of extreme events, many previous attribution studies, especially in the context of heatwaves, have used relatively coarse, atmosphere-only climate models \citep{massey_weatherhome-development_2015,ciavarella_upgrade_2018,christidis_new_2013}, which may not fully capture all the physical processes required to credibly simulate the extreme in question \citep{sillmann_understanding_2017}. In particular, the use of atmosphere-only simulations may result in the full space of climate variability being under-sampled due to the lack of atmosphere-ocean interaction \citep{fischer_biased_2018}. This can lead to studies overestimating the impact of anthropogenic activity on weather extremes \citep{leach_anthropogenic_2020,bellprat_attribution_2016}. More generally, Bellprat et al., and Palmer and Weisheimer \citep{bellprat_towards_2019,palmer_simple_2018} have shown the importance of initial-value reliability in model ensembles underlying robust attribution statements. Model evaluation is therefore a key part of any robust model-based attribution study. Here, the demonstrably successful forecast enables us to be confident that the model used is providing credible realisations of the event.

    A clear distinction between the typical climate model simulations used for attribution \citep{massey_weatherhome-development_2015,christidis_new_2013} and the forecasts used here is that the climate model simulations are usually allowed to spin out for a sufficient length of time such that they have no memory of their initial conditions; an ensemble constructed in this way will therefore be representative of the climatology of the model. If such simulations use prescribed-SST boundary conditions, then the ensemble will be representative of the climatology conditioned on the prescribed SST pattern \citep{ciavarella_upgrade_2018}. Unlike climatological simulations, a successful forecast is conditioned upon the component of the weather that is predictable at initialisation. In general, the level of conditioning imposed upon the ensemble by the initial conditions reduces as the model integrates forwards from the initialisation date. Hence a forecast ensemble initialised only a few days before an event will be much more heavily conditioned (and therefore much less spread) than one initialised weeks before. As the lead time increases, a forecast ensemble will tend towards the model climatology, analogous to the climate model simulations discussed above. We can relate these situations to the two broad attribution frameworks discussed in \citep{shepherd_common_2016}: very long lead times, where the forecast simulates model climatology, are analogous to ``conventional'' attribution; while short lead times, in which the forecast ensemble is heavily dependent on the initial conditions and therefore conditional on the actual dynamical drivers that lead to the extreme event, are analogous to the ``storyline'' approach in \citep{hazeleger_tales_2015,van_garderen_methodology_2021}. In order to synthesize between these two frameworks, here we have chosen 4 initialisation times (3-, 9-, 15-, and 22-day leads) for our experiments that span the range from a near-unconditioned climatological forecast to a short-term forecast that is tightly conditioned on the actual dynamical drivers of the heatwave.

    \clearpage
    \begin{figure}[h]
      \centering
      \includegraphics[width=\textwidth]{{Fig3.2}.pdf}
      \caption[Medium- to extended-range forecasts of the heatwave.]{\textbf{Medium- to extended-range forecasts of the heatwave.}}
    \end{figure}
    \clearpage

\section{Perturbed CO$_2$ forecasts}\label{Ch4:experiments}

  In this study we choose to only change one feature of the operational forecast in our experiments: the CO$_2$ concentration. This means that the analysis we carry out is limited to attributing the impact of diabatic heating due to increased CO$_2$ concentrations above pre-industrial levels just over the days between the model initialisation date and the event. Although this results in a counterfactual that does not correspond to any ``real'' world (since it is one with approximately present-day temperatures but pre-industrial CO$_2$ concentrations), and thus reduces the relevance of our analysis to stakeholders or policymakers; it does significantly increase the interpretability of our results, and remove a major source of uncertainty associated with a “complete” attribution to human influence: the estimation of the pre-industrial ocean and sea-ice state vector used to initialise the model \citep{stone_benchmark_2021}. Here we define a complete attribution as an estimate of the total impact of human influence on the climate arising from anthropogenic emissions of greenhouse gases and aerosols since the pre-industrial period. For each lead time chosen, in addition to the operational forecast (ENS) we run two experiments using operational initial conditions and identical to the operational forecast in every way except the experiments have specified fixed CO$_2$ concentrations. One experiment has CO$_2$ concentrations fixed at pre-industrial levels of 285 ppm (PI-CO$_2$), while in the other they are increased to 600 ppm (INCR-CO$_2$). These represent approximately equal and opposite perturbations on global radiative forcing \citep{etminan_radiative_2016}. We carry out these two experiments for each lead time, perturbing the CO$_2$ concentration in opposite directions, to ensure that any changes to the likelihood of the event can be confidently attributed to the changed CO$_2$ concentrations. It is possible that, due to the chaotic nature of the weather, the operational conditions were ideal for generating the observed extreme, and any perturbation to the dynamical system would reduce the likelihood of its occurrence \citep{shepherd_common_2016}. If this were the case we would see a reduction in event probability regardless of whether we increased or reduced the CO$_2$ concentration.
  
  Some previous work has been done on the impact of reduced CO$_2$ concentrations in the absence of changes to global sea surface temperatures. Baker et al. \citep{baker_higher_2018} explored how temperature and precipitation extremes were affected by the direct effect of CO$_2$ concentrations (defined there as all the effects of CO$_2$ on climate beside those occurring through ocean warming), finding the direct effect of CO$_2$ increases risk of temperature extremes, especially within the Northern hemisphere summer. Our experimental design is also reminiscent of some of the earliest work done on investigating the impact of CO$_2$ on climate in global circulation models \citep{gates_preliminary_1981,mitchell_seasonal_1983}. This work found that, in the absence of changes to sea surface temperatures or sea ice concentrations, a doubling of CO$_2$ concentrations would change global mean surface temperatures over land by $\sim0.4$ \degree C. These early studies indicate that changes in global land temperatures are approximately linear with the logarithm of CO$_2$ concentration.
  
  We find that the best-estimate global mean change in land surface temperatures attributable to the additional diabatic heating due to CO$_2$ over pre-industrial levels (henceforth the ``CO$_2$ signal'', calculated as half the difference between the two experiments for a particular variable) at a lead time of two weeks (over the final 5 days of the forecasts initialised on 2019-02-11) is 0.22 [0.20 , 0.25] \degree C (square brackets indicate a 90 \% confidence interval throughout). In general, the further away from the initialisation date, the slower the rate of change of the globally-averaged ensemble mean CO$_2$ signal, and the larger the ensemble spread (Fig. $3A$). While in experiments with prescribed SSTs, we might expect the CO$_2$ signal in surface temperatures to approach a maximum value within timescales on the order of months, in our experiments the CO$_2$ signal will likely continue to increase in magnitude for centuries due to the ocean-coupling, as is the case in the abrupt-4xCO$_2$ experiment carried out in CMIP \citep{taylor_overview_2012,eyring_overview_2016,rugenstein_equilibrium_2020}. The zonal-mean patterns of surface temperature CO$_2$ signal are qualitatively similar to those exhibited by CMIP5 and CMIP6 models during the abrupt-4xCO$_2$ experiment \citep{flynn_climate_2020,andrews_dependence_2015}, despite the considerably shorter timescales involved: small and very confident changes in the tropics become larger but much less confident changes at the poles. This heterogeneity in the zonal distribution of warming appears to originate in the zonal distribution of the lapse-rate feedback; the weekly timescales of these experiments is insufficient for the surface-albedo feedbacks to have any significant impact \citep{smith_polar_2019}.
  
  We also examine the impact on the specific event dynamics over our region of interest; since these were crucial in developing the extremes observed. Fig. $3B$ shows the growth in Z500 errors (measured as the mean absolute distance from ERA5 over the European domain) for each of the experiments. This figure illustrates that there are no clear differences in the ability of each experimental ensemble to predict the dynamical characteristics of the event. In other words, we have not made the synoptic event any more or less likely as a result of our perturbations. This is crucial as it means that we can consider any changes to the magnitude of the temperatures observed to be entirely due to the thermodynamic effect of changed diabatic CO$_2$ heating, and not due to the attractor of the dynamical system having changed as a result of the perturbations we have made.
  
  Figs. $3C$ and $3D$ show analogous plots to $3B$, but for inter-experimental and intra-ensemble errors respectively. These indicate a couple of important features. Firstly, no two experiments are more similar than any other two; the magnitude of Z500 distances in Fig. $3C$ are near identical for all lead and validation times. Secondly, the error growth due to the CO$_2$ perturbation is slower than due to the initial condition perturbations; the errors in Fig. $3C$ increase slower than in $3D$. However, by the end of the longest lead forecast, we can see that the intra-ensemble errors have saturated, and the inter-experimental errors have grown to be the same magnitude. The saturation of intra-ensemble errors by the end of this lead time reinforces our assertion that at this lead the forecast is a good approximation of a climatological simulation.

  \clearpage
  \begin{figure}[h]
    \centering
    \includegraphics[width=0.8\textwidth]{{Fig3.3}.pdf}
    \caption[Global temperature and synoptic-scale dynamical response to CO$_2$ perturbations.]{\textbf{Global temperature and synoptic-scale dynamical response to CO$_2$ perturbations.}}
  \end{figure}
  \clearpage

\section{Attributing the heatwave to diabatic CO$_2$ heating}\label{Ch4:attribution}

  First, we examine the geographical pattern of the CO$_2$ signal in the heatwave in Fig. $4A-D$. These indicate several key features of the attributable direct CO$_2$ effect on the heatwave. The CO$_2$ effect tends to grow with lead time, consistent with its impact on global mean temperatures. It is generally stronger over land than ocean, also consistent with global mean temperatures. Finally, the ensemble tends to become less confident in its effect as the lead time increases and the ensemble members diverge. The CO$_2$ signal magnitude in the heatwave generally exceeds the signal in the global mean surface temperature (Fig. $3A$), in particular in Central Europe; possibly due to the high contribution of diabatic heating to the heatwave arising from ideal dynamical conditions. Fig. $4E$ shows boxplots of the heatwave CO$_2$ signal for the three regions of interest plus the full European land area. Although there is some region-specific variability, these reinforce the main messages illustrated by the maps: the CO$_2$ signal grows and decreases in confidence as the lead time increases.

  In addition to the absolute impact of the direct CO$_2$ effect on the heatwave, we also carry out a probabilistic assessment of its impact, consistent with conventional ``risk-based'' attribution studies \citep{shepherd_common_2016,winsberg_severe_2020}. Due to the novel approach we are taking within this study, it is worth clarifying exactly what question we are answering with this probabilistic analysis. The specific question is: ``given the forecast initial conditions, how did the direct impact of increased CO$_2$ concentrations compared to pre-industrial levels just over the days between initialisation and the heatwave itself change the probability of temperatures at least as hot as were observed?''. Using conventional attribution terminology, we call the operational forecast ensemble of the event as our ``factual'' ensemble, and the pre-industrial CO$_2$ experiment as our ``counter-factual'' ensemble. We calculate the probability of simulating an event at least as extreme as observed in the factual ensemble, $P_1$, and in the counterfactual ensemble, $P_0$. These probabilities are estimated by fitting a generalised extreme value distribution to the 51-member ensemble in each case. We then express the change in event probability as a risk-ratio, $RR=P_1/P_0$, which represents the fractional increase in the likelihood of an event at least as extreme as observed in the factual ensemble over the counterfactual ensemble \citep{stott_human_2004,stone_end--end_2005}. Uncertainties are estimated with a 100,000 member bootstrap with replacement, rejecting samples for which the probability of the event in the factual ensemble is zero. The resulting risk-ratios are shown in Fig. $4F$. There are several key factors that contribute to the best-estimate and confidence in the risk ratios: the CO$_2$ signal growth with lead time; the ensemble spread growth with lead time; how extreme the event was; and how well-forecast the event was. The larger the CO$_2$ signal, the greater the increase in risk; the larger the ensemble spread, the lesser the increase in risk and the lower the confidence; the more extreme the event, the greater the increase in risk; and the better the forecast (ie. the closer the event to the ensemble centre), the greater the confidence. 
  
  We find that on the shortest lead time, the direct CO$_2$ effect increases the probability of the event over all European regions (significant at the 5 \% level based on a one-sided test). For the well-forecast event experienced over the British Isles, the direct CO$_2$ effect increases the probability of the extreme heat by 42 [30 , 60] \%. For the France heatwave, which was well-forecast given its exceptional nature, but for which the ensemble did not quite reach the total magnitude of the heat experienced, the event probability increased by at least 100 \% (5\textsuperscript{th} percentile), but with a very wide uncertainty range. Finally, for the least remarkable but relatively well-forecast event over the Mediterranean, the direct impact of CO$_2$ increased the event probability by 6.7 [4.6 , 9.7] \%. These results from the very short lead experiments represent very highly conditioned statements: in both ensembles the dynamical evolution of the event was near-identical (pattern correlation of $>0.99$ for all ensemble members, Fig. $2B$).
  
  Moving out to the longer lead times, we find that the confidence in the change in event probability decreases almost ubiquitously. This is as expected, since the further we move away from the event, the less highly conditioned our ensemble is, and the more dynamical noise we are adding to the system \citep{shepherd_common_2016}. However, for the 9-day lead forecast, the uncertainty is low enough to have confidence in the results for the majority of study regions. In particular, the British Isles heatwave, for which the 9-day lead forecast was better than several of the regional 3-day lead forecasts (as measured by the Continuous Ranked Probability Skill Score), increases in probability by 52 [29 , 94] \% due to the direct CO$_2$ effect. However, for France the uncertainty range is so large that based on these results alone we would have no confidence in the direction of the CO$_2$ effect. Moving further out to the 15- and 22-day lead forecasts, this loss in confidence becomes more pronounced, especially for the British Isles region. For this region, we can get virtually no useful information out of these probabilistic results for the two longest lead experiments. This drop-off in confidence arises due to the increasing ensemble spread from dynamical noise, and large reduction in the number of factual ensemble members able to simulate an event as hot as occurred in reality between the 9- and 15-day leads. A similar, though generally less pronounced drop-off in confidence is found in all other regions. 
  
  We can make use of our INCR-CO$_2$ experiment to increase our confidence that the positive results we obtained in the probabilistic analysis above are in fact due to the direct CO$_2$ effect, and not just random variability. If CO$_2$ were driving the changes in event probability between the PI-CO$_2$ and operational forecasts, then we would expect to see an even more dramatic increase in event probability between the PI-CO$_2$ and INCR-CO$_2$ forecasts. This is indeed what we find. For all regions and lead times, our best-estimate change in event probability is above zero when CO$_2$ concentration is increased from pre-industrial levels of 285 ppm to 600 ppm. This therefore increases our confidence further that the positive attribution to CO$_2$ under high conditioning is genuinely significant. From these results, it also appears that there is a general trend of change in event probability increasing as the forecast lead increases, similar to the absolute impact of the direct CO$_2$ effect trend; though it is still somewhat masked by uncertainty.
  
  An important caveat on all of these results, probabilistic and absolute, is that they represent a lower bound on the estimate of the direct CO$_2$ effect. As is clear from the development of the CO$_2$ signal estimates with lead time, the model is still adjusting to the sudden change in CO$_2$ concentration (and would continue to do so for centuries due to the very long deep ocean equilibration timescales). Hence we would expect the ``full'' effect of CO$_2$ to be greater than the estimates we present here. This is consistent with a recent study that used unconditioned climate model simulations to carry out an attribution of the complete anthropogenic contribution to the same event, which produced much higher estimates of the risk-ratio \citep{christidis_extremely_2021}.

  \clearpage
  \begin{figure}[h]
    \centering
    \includegraphics[width=\textwidth]{{Fig3.4}.pdf}
    \caption[Attribution of the direct CO$_2$ influence on the heatwave.]{\textbf{Attribution of the direct CO$_2$ influence on the heatwave.}}
  \end{figure}
  \clearpage

\section{Discussion}\label{Ch4:discussion}

  Here we have presented a partial, forecast-based attribution of the European 2019 winter heatwave. Taking advantage of successful medium-range forecasts from ECMWF, we used a state-of-the-art numerical weather prediction model that was demonstrably able to predict the event to attribute the direct impact of CO$_2$ through diabatic heating over pre-industrial levels and just over the days immediately preceding the event on the high temperatures experienced in several regions of Europe. We explored how the level of dynamical conditioning imposed can be specified by changing the lead time of the forecasts. Finally, we presented our quantitative results using two different approaches: measuring the attributable absolute and probabilistic impacts of CO$_2$; inspired by the ``storyline'' and ``risk-based'' attribution frameworks \citep{stott_human_2004,shepherd_common_2016,winsberg_severe_2020,jezequel_behind_2018}.  
  
  There are several advantages associated with this novel forecast-based attribution methodology, compared to conventional climate model based attribution. One simple advantage is that forecast models generally represent the technological peak within the spectrum of General Circulation Models. They tend to have a higher resolution than the models used for climate simulation. In addition, the forecast model used here is coupled, while the large climate model ensembles used for attribution tend to use prescribed sea surface temperatures \citep{ciavarella_upgrade_2018}. The use of prescribed SSTs can lead to model biases that project strongly onto attribution results \citep{fischer_biased_2018}. A final advantage arising from the use of an operational forecast model is the wealth of literature and model analysis that will already be available before an attribution study is initiated. As well as these advantages associated with the type of model there is the crucial advantage associated with using successful forecasts: the specific and intrinsic model verification. Due to the difficulty in fully quantifying how well climate models can represent an individual specific event (in particular, the very large ensembles required to have a large enough sample of characteristically similar events), climate model based attribution studies tend to perform statistical model evaluations; or/and account for this uncertainty through multi-model ensembles \citep{philip_protocol_2020}. On the other hand, if a forecast model that demonstrably predicted the event as it occurred is used, no further model verification or evaluation is required to test whether the model is capable of producing a faithful representation of the specific event.
  
  Related to this intrinsic verification is an important point on the framing of forecast-based attribution studies. Climate model based attribution studies tend to characterise an event in terms of some quantitative index closely related to the impact of the event (such as the maximum temperature observed during a heatwave). They then use climate model simulations to determine how climate change has affected the probability of observing an event at least as extreme as the actual event. This is often done without imposing any dynamical conditioning on the simulations, though this is an area of active research \citep{yiou_statistical_2017,pall_diagnosing_2017}. This unconditional approach means that the specific question being answered is not ``how has anthropogenic climate change affected the probability of event X?'', but ``how has anthropogenic climate change affected the probability of all events that are at least as extreme as event X in terms of the index used to define X?''.  This second question does not fully answer the question of how climate change has affected the actual event that the study is concerned with. In contrast, the use of a forecast model that predicted the event ensures that any attribution analysis is unequivocally an attribution of that specific event \citep{hope_determining_2019}.
  
  In addition to its advantages, this novel forecast-based attribution methodology also has associated issues that must be overcome. Firstly, the forecast model must have produced a ``good'' forecast of the event. If the model is unable to represent the event as it happened, then we cannot have confidence in any estimates of the impact of climate change on that event. Issues can arise even in qualitatively ``good'' forecasts, such as the forecast of the heatwave over France in this study. As very few ensemble members, if any, exceeded the observed magnitude of the event for this region, the confidence in our estimates of the probabilistic impact of CO$_2$ on the event is extremely low (since we are extrapolating the distribution shape outside of the range of our data). Although the estimates of the absolute impact of CO$_2$ do not share this lack of confidence, this is still a problem. It is possible that applying some bias correction procedure \citep[e.g.][]{sippel_novel_2016,jeon_quantile-based_2016,li_reducing_2019} based on the model climatology to the model output before analysis might alleviate these issues to some extent, but not if the model is simply unable to predict the event in question (ie. a forecast bust). Secondly, the short timescales involved in these medium-range forecasts mean that the interpretation of any results becomes more difficult as the model is still adjusting to the perturbations imposed \citep{hope_contributors_2015}, at least in the case of the CO$_2$ perturbations applied here. This adjustment is clear on a global scale in Fig. $3A$. Due to this incomplete adjustment, any quantitative statements of attribution represent a lower bound on the ``true'' value. 
  
  We have shown that the direct effect of CO$_2$ concentrations over pre-industrial levels on the February heatwave is significant, even on timescales as short as a few days. Based on the very good 9-day lead forecast of the heatwave over the British Isles, the region that saw the most climatologically exceptional event, the direct effect of CO$_2$ was to increase the magnitude of the heatwave by 0.31 [0.24 , 0.37] K, and the conditional probability of the heatwave by 52 [29 , 94] \%. It is very important to bear in mind that this statement of risk is highly dynamically conditioned (Fig. $2B$). These estimates of the impact of CO$_2$ on the heatwave follow the storyline attribution framework, since we have effectively removed the dynamical uncertainty from our simulations with this strong conditioning imposed by the short lead time \citep{shepherd_common_2016,shepherd_storylines_2018,jezequel_behind_2018}. Our longer, 22-day lead experiments can contrast this storyline analysis with relatively unconditioned results much closer to the climatological simulations typically used in the conventional ``Risk-based'' attribution framework \citep{philip_protocol_2020,stott_human_2004}. At this lead, we find that although over all regions the best-estimate impact of the direct CO$_2$ effect is to enhance the heatwave by approximately 0.5 K, in none of the regions is this impact significantly positive at the 90 \% level (based on the bootstrapped confidence in the median value). Corresponding estimates of the risk ratio have so low confidence that they provide virtually no useful information. Increasing the forecast ensemble size, which is small compared to the climate model ensembles used in most attribution studies, would increase the confidence, potentially resulting in useful quantitative estimates of the risk ratio even at these longer lead times. Our results illustrate some of the concerns voiced recently over the conventional risk-based approach to attribution \citep{winsberg_severe_2020,shepherd_common_2016}. Due to the dynamical noise present in unconditioned ensembles, it is possible to obtain an inconclusive attribution within a conventional risk-based framework, and at the same time obtain a confident positive attribution if the dynamical uncertainty is removed through conditioning (in our case achieved by reducing the forecast lead).
  
  While this study provides a demonstration of the potential use for forecast models within attribution science, it remains a partial attribution to the direct CO$_2$ effect only. For forecast-based attribution to provide results that are fully comparable to conventional climate model-based attribution, we will need to demonstrate how the complete anthropogenic contribution to an extreme event could be estimated with successful forecasts. The next step to progress forecast-based attribution further will be to remove an estimate of the anthropogenic contribution to ocean temperatures from the model initial conditions \citep[e.g.][]{stone_benchmark_2021}. If performed in addition to reducing other greenhouse gas concentrations and aerosol climatology down to their pre-industrial levels, this should allow us to run pre-industrial forecasts of an event. This has been done previously for a seasonal forecast model by Hope et al. \citep{hope_contributors_2015,hope_what_2016,hope_determining_2019}. They removed the anthropogenic signal from 1960 onwards from the initial conditions, but we could in principle remove the signal from pre-industrial times onwards in order to estimate the complete anthropogenic contribution to an event. Although it is highly likely that there will be methodology specific issues that arise in this direction, we suggest that being able to estimate the complete anthropogenic contribution to an extreme event using a forecast model that was able to predict the event in question would be extremely valuable. Developing a methodology to allow us to do so might also provide a pathway to operational attribution being able to be carried out by weather prediction centres, due to the routine frequency at which they produce forecasts. In addition to attempting a ``complete'' forecast-based attribution of an extreme event, we would like to explore how increasing the ensemble size may allow us to provide confident forecast-based attribution analyses within the unconditioned risk-based framework (ie. at long forecast lead times). One potential avenue to allow us to do this efficiently might be to reduce the resolution of the forecasts, though this would not be appropriate if it reduced the ability of the model to represent the event in question. On a similar note, we would also like to extend our experiments out to seasonal timescales. This would reduce the issues with the interpretation of our medium-range results that occurred due to the model adjustment to the sudden changes to the CO$_2$ concentration. It is possible that seasonal forecasts have the greatest potential to target for an operational forecast-based attribution methodology.

\section{Chapter close}

\begin{savequote}[8cm]
    Quote
      \qauthor{--- author}
\end{savequote}
    
\chapter{\label{ch4}Forecast-based attribution: perturbing the initial and boundary conditions}

Chapter description.
\small\paragraph{Author contributions:} This chapter is based on the the following publication \footnote{with the author contributing as follows.} \par\vspace{1em}
\formatchref{Surname, I1. I2., Surname, I1. I2.}{year}{Title}{Journal}{vol}{issue}{pages}{DOI}

\clearpage

\minitoc

\clearpage

\section{Chapter open}



\section{Chapter close}

\include{Ch5/Ch5}

\begin{savequote}[8cm]
    Quote
      \qauthor{--- author}
\end{savequote}
    
\chapter{\label{discussion}Discussion} 

Chapter description.
% \small\paragraph{Author contributions:} This chapter is based on the the following publication \footnote{with the author contributions as follows.} \par\vspace{1em}
% \formatchref{Surname, I1. I2., Surname, I1. I2.}{year}{Title}{Journal}{vol}{issue}{pages}{DOI}

\minitoc

\clearpage

\section{An overview of this thesis}

  \blindtext

\section{This thesis in the context of previous work}

  References of relevant work. Hurricane forecast-based \citep{reed_attribution_2022,reed_forecasted_2020,patricola_anthropogenic_2018,lackmann_hurricane_2015}. Nested forecast modelling \citep{schaller_role_2020,meredith_crucial_2015}. Initialised (sub)seasonal \citep{hope_contributors_2015,hope_what_2016,hope_determining_2019,hope_subseasonal_2022,wang_initialized_2021,tradowsky_toward_2022,stone_effect_2022}. DADA \citep{hannart_dada_2016}. Probabilistic methodology ref \citep{pall_anthropogenic_2011}. Pseudo global warming \citep{schar_surrogate_1996}.

\section{Limitations}

  Although I have discussed various limitations of the individual studies that make up this thesis, in this section I consider some of the limitations of the forecast-based approach to attribution developed and explored here as a whole.

  \subsection*{Forecast adjustment}

    One key aspect of the counterfactual forecasts performed here is that the model --- or more specifically the model atmosphere --- adjusts continually to the imposed perturbations throughout the integration. This means that the further into the forecast the event of interest happens, the stronger the attributed impact of those perturbations is. At the same time, as the forecast evolves, this effect becomes more uncertain in general (though not necessarily always), due to the increasing dynamical noise arising from the chaotic nature of the weather system. This combination of this increasing strength and uncertainty can make analysing and interpreting the results of the counterfactual forecast experiments difficult. In chapter \ref{ch4} I alleviated this difficulty by making use of the fact that the attributable regional impacts of climate change were near-linearly related to the coincidental measured level of global warming. This linear relationship allowed me to benchmark the estimated impact at each forecast lead time to the same level of global warming, regardless of how adjusted they were at the time of the event in question. However, this linear relationship is not guaranteed for every extreme event, and therefore it would be valuable to find methodologies by which this adjustment could either be reduced or removed entirely. I consider a few ideas to achieve this below.

  \subsection*{Additional uncertainty dimensions}

    In the experiments performed here, I have only considered uncertainty arising from the chaotic nature of the weather system. However, there are additional uncertainties associated with the approach I have developed. 

  \subsection*{Single model}

  \subsection*{Single event class}

  \subsection*{Considering additional forcing agents}

  

\section{Future research directions}

  \subsection*{Assimilating the perturbations}

    \blindtext

  \subsection*{Alternative extremes}

    \blindtext

  \subsection*{Alternative forecast models}

    \blindtext

  \subsection*{Incorporating perturbation uncertainty}

    \blindtext

  \subsection*{Projections of future extremes}

    \blindtext

  \subsection*{Impact assessment}

    \blindtext

\section{Concluding remarks}


%% APPENDICES %% 
% Starts lettered appendices, adds a heading in table of contents, and adds a
%    page that just says "Appendices" to signal the end of your main text.
\startappendices
% Add or remove any appendices you'd like here:
\chapter{\label{resources}Resources}

Examples of output from the thesis besides research:
\begin{itemize}
    \item mystatsfunctions plus other packages (eg. FaIR)
    \item Code accompanying papers
    \item CEDA archive data
    \item MARS datasets
\end{itemize}


%%%%% REFERENCES

% JEM: Quote for the top of references (just like a chapter quote if you're using them).  Comment to skip.
\begin{savequote}[8cm]
  Quote
  \qauthor{--- author}
\end{savequote}

% NJL: this sets the spacing since baselineskip commands don't seem to work...
\singlespacing

\setlength{\baselineskip}{0pt} % JEM: Single-space References

{\renewcommand*\MakeUppercase[1]{#1}%
\printbibliography[heading=bibintoc,title={\bibtitle}]}


\end{document}