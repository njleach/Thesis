%%%%%%%%%%%%%%%%%%%%%%%%%%%%%%%%%%%%%%%%%%%%%%%%%%%%%%%%%%%%%%%
%% OXFORD THESIS TEMPLATE

% Use this template to produce a standard thesis that meets the Oxford University requirements for DPhil submission
%
% Originally by Keith A. Gillow (gillow@maths.ox.ac.uk), 1997
% Modified by Sam Evans (sam@samuelevansresearch.org), 2007
% Modified by John McManigle (john@oxfordechoes.com), 2015
%
% This version Copyright (c) 2015-2017 John McManigle
%
% Broad permissions are granted to use, modify, and distribute this software
% as specified in the MIT License included in this distribution's LICENSE file.
%

% I've (John) tried to comment this file extensively, so read through it to see how to use the various options.  Remember
% that in LaTeX, any line starting with a % is NOT executed.  Several places below, you have a choice of which line to use
% out of multiple options (eg draft vs final, for PDF vs for binding, etc.)  When you pick one, add a % to the beginning of
% the lines you don't want.


%%%%% CHOOSE PAGE LAYOUT
% The most common choices should be below.  You can also do other things, like replacing "a4paper" with "letterpaper", etc.

% This one will format for two-sided binding (ie left and right pages have mirror margins; blank pages inserted where needed):
\documentclass[a4paper,twoside]{ociamthesis}
% This one will format for one-sided binding (ie left margin > right margin; no extra blank pages):
%\documentclass[a4paper]{ociamthesis}
% This one will format for PDF output (ie equal margins, no extra blank pages):
%\documentclass[a4paper,nobind]{ociamthesis} 


%%%%% GENERIC TEXT
\usepackage{blindtext}


%%%%% SELECT YOUR DRAFT OPTIONS
% Three options going on here; use in any combination.  But remember to turn the first two off before
% generating a PDF to send to the printer!

% This adds a "DRAFT" footer to every normal page.  (The first page of each chapter is not a "normal" page.)
\fancyfoot[C]{\emph{DRAFT Printed on \today}}  

% This highlights (in blue) corrections marked with (for words) \mccorrect{blah} or (for whole
% paragraphs) \begin{mccorrection} . . . \end{mccorrection}.  This can be useful for sending a PDF of
% your corrected thesis to your examiners for review.  Turn it off, and the blue disappears.
\correctionstrue


%%%%% BIBLIOGRAPHY SETUP
% Note that your bibliography will require some tweaking depending on your department, preferred format, etc.
% The options included below are just very basic "sciencey" and "humanitiesey" options to get started.
% If you've not used LaTeX before, I recommend reading a little about biblatex/biber and getting started with it.
% If you're already a LaTeX pro and are used to natbib or something, modify as necessary.
% Either way, you'll have to choose and configure an appropriate bibliography format...

% The science-type option: numerical in-text citation with references in order of appearance.
\usepackage[style=phys, sorting=none, backend=biber, doi=false, isbn=false, natbib=true, citestyle=science, backref=true]{biblatex}
\newcommand*{\bibtitle}{References}
% defines the backreferencing string in the bibliography
\DefineBibliographyStrings{english}{%
  backrefpage = {p.},%
  backrefpages = {pp.}%
}

% The humanities-type option: author-year in-text citation with an alphabetical works cited.
%\usepackage[style=authoryear, sorting=nyt, backend=biber, maxcitenames=2, useprefix, doi=false, isbn=false]{biblatex}
%\newcommand*{\bibtitle}{Works Cited}

% This makes the bibliography left-aligned (not 'justified') and slightly smaller font.
\renewcommand*{\bibfont}{\raggedright\small}

% Change this to the name of your .bib file (usually exported from a citation manager like Zotero or EndNote).
\addbibresource{Introduction/Thesis-Introduction.bib}
\addbibresource{Ch1/Thesis-Ch1.bib}
\addbibresource{Ch2/Thesis-Ch2.bib}
\addbibresource{Ch3/Thesis-Ch3.bib}
\addbibresource{Discussion/Thesis-Discussion.bib}

% NJL: tidy up bibliography & allow custom notes (through the Zotero "note" interface) to be printed after entries

% NJL: these lines remove unwanted fields from the printed biliography
\AtEveryBibitem{\clearfield{issn}}
\AtEveryBibitem{\clearlist{issn}}
\AtEveryBibitem{\clearfield{language}}
\AtEveryBibitem{\clearlist{language}}
\AtEveryBibitem{\clearfield{note}}
\AtEveryBibitem{\clearlist{note}}

% NJL: adds in the annotation field
\DeclareFieldFormat{annotation}{\textbf{#1}}
\AtEveryBibitem{
\csappto{blx@bbx@\thefield{entrytype}}{\iffieldundef{annotation}{}{\par\printfield{annotation}}}
}

% Uncomment this if you want equation numbers per section (2.3.12), instead of per chapter (2.18):
%\numberwithin{equation}{subsection}



%%%%% THESIS / TITLE PAGE INFORMATION
% Everybody needs to complete the following:
\title{Robust attribution and projection of extreme heat events to human influence on the climate}
\author{Nicholas J. Leach}
\college{St. Cross College}

% Master's candidates who require the alternate title page (with candidate number and word count)
% must also un-comment and complete the following three lines:
%\masterssubmissiontrue
%\candidateno{933516}
%\wordcount{28,815}

% Uncomment the following line if your degree also includes exams (eg most masters):
%\renewcommand{\submittedtext}{Submitted in partial completion of the}
% Your full degree name.  (But remember that DPhils aren't "in" anything.  They're just DPhils.)
\degree{Doctor of Philosophy}
% Term and year of submission, or date if your board requires (eg most masters)
\degreedate{Trinity 2022}

% Supervisors
\supervisorone{Antje Weisheimer}
\supervisortwo{Myles R. Allen}

%%%%% YOUR OWN PERSONAL MACROS
% This is a good place to dump your own LaTeX macros as they come up.

% To make text superscripts shortcuts
	\renewcommand{\th}{\textsuperscript{th}} % ex: I won 4\th place
	\newcommand{\nd}{\textsuperscript{nd}}
	\renewcommand{\st}{\textsuperscript{st}}
	\newcommand{\rd}{\textsuperscript{rd}}

% NJL: to format chapter references nicely
\newcommand{\formatchref}[8]{#1 (#2). \textbf{#3}. \emph{#4}, \textbf{#5}(#6), #7. \url{#8}}

% NJL: to use tt font for urls
\urlstyle{tt}

% NJL: try to change spacing (works but messes up frontmatter)
\usepackage[]{setspace}
% \onehalfspacing

%%%%% THE ACTUAL DOCUMENT STARTS HERE
\begin{document}


%%%%% CHOOSE YOUR LINE SPACING HERE
% This is the official option.  Use it for your submission copy and library copy:
\setlength{\textbaselineskip}{22pt plus2pt}
% This is closer spacing (about 1.5-spaced) that you might prefer for your personal copies:
%\setlength{\textbaselineskip}{18pt plus2pt minus1pt}

% You can set the spacing here for the roman-numbered pages (acknowledgements, table of contents, etc.)
\setlength{\frontmatterbaselineskip}{17pt plus1pt minus1pt}

% Leave this line alone; it gets things started for the real document.
\setlength{\baselineskip}{\textbaselineskip}


%%%%% CHOOSE YOUR SECTION NUMBERING DEPTH HERE
% You have two choices.  First, how far down are sections numbered?  (Below that, they're named but
% don't get numbers.)  Second, what level of section appears in the table of contents?  These don't have
% to match: you can have numbered sections that don't show up in the ToC, or unnumbered sections that
% do.  Throughout, 0 = chapter; 1 = section; 2 = subsection; 3 = subsubsection, 4 = paragraph...

% The level that gets a number:
\setcounter{secnumdepth}{2}
% The level that shows up in the ToC:
\setcounter{tocdepth}{2}


%%%%% ABSTRACT SEPARATE
% This is used to create the separate, one-page abstract that you are required to hand into the Exam
% Schools.  You can comment it out to generate a PDF for printing or whatnot.
\begin{abstractseparate}
	Anthropogenic greenhouse gas emissions are now well-understood to be causing damaging changes to the climate. One of the many ways in which the climate is changing is through extreme weather events. Given the severe consequences of such events, understanding how human influence on the climate is affecting them is vital. This is the aim of the young field of `extreme event attribution'. There now exist many established methods for attributing individual weather events to climate change, from probabilistic approaches utilising large climate model ensembles to deterministic storyline approaches. However, questions still remain over the reliability of these approaches, especially when considering the most unprecedented events. In this thesis, I show how weather forecast models could provide us with such reliable information about human influence on extreme weather --- focusing on extreme heat. These models are state-of-the-art and can be shown to be unequivocally able to simulate the detailed physics of specific extreme weather through successful prediction. I develop a perturbed initial- and boundary-condition approach within an operational forecasting system that aims to produce forecasts of individual events as if they had occurred in a world without human influence on the climate. These `counterfactual' forecasts can be used to assess how not only the intensity, but also the probability of such events has changed. Although extreme weather attribution typically focuses on the past, the same approach could be used to produce forecasts in warmer future worlds --- thus providing vital information about how the most damaging weather may be expected to change in the future. I explore this theme of extreme weather projection, examining a novel approach to producing large climate model ensembles that span the range of uncertainty in future extreme weather. This work complements the specific nature of extreme event attribution, and they could together provide crucial information about climate risk.  % Create an abstract.tex file in the 'text' folder for your abstract.
\end{abstractseparate}


% JEM: Pages are roman numbered from here, though page numbers are invisible until ToC.  This is in
% keeping with most typesetting conventions.
\begin{romanpages}

% Title page is created here
\maketitle

%%%%% DEDICATION -- If you'd like one, un-comment the following.
%\begin{dedication}
%This thesis is dedicated to\\
%someone\\
%for some special reason\\
%\end{dedication}

%%%%% ACKNOWLEDGEMENTS -- Nothing to do here except comment out if you don't want it.
\begin{acknowledgements}
 	\subsection*{Personal}
% \begin{itemize}
%     \item Charlotte
%     \item Family
%     \item Friends
% \end{itemize}
\subsection*{Institutional}
% \begin{itemize}
%     \item Supervisors
%     \item ALL co-authors
%     \item Other people who have helped
%     \begin{itemize}
%         \item Paul Dando
%         \item Robin Hogan
%         \item Mat Chantry
%         \item Man-Suen
%         \item Lucy
%         \item Victoria
%         \item Lewis O?
%         \item Heather W
%     \end{itemize}
% \end{itemize}
\end{acknowledgements}

%%%%% ABSTRACT -- Nothing to do here except comment out if you don't want it.
\begin{abstract}
	Anthropogenic greenhouse gas emissions are now well-understood to be causing damaging changes to the climate. One of the many ways in which the climate is changing is through extreme weather events. Given the severe consequences of such events, understanding how human influence on the climate is affecting them is vital. This is the aim of the young field of `extreme event attribution'. There now exist many established methods for attributing individual weather events to climate change, from probabilistic approaches utilising large climate model ensembles to deterministic storyline approaches. However, questions still remain over the reliability of these approaches, especially when considering the most unprecedented events. In this thesis, I show how weather forecast models could provide us with such reliable information about human influence on extreme weather --- focusing on extreme heat. These models are state-of-the-art and can be shown to be unequivocally able to simulate the detailed physics of specific extreme weather through successful prediction. I develop a perturbed initial- and boundary-condition approach within an operational forecasting system that aims to produce forecasts of individual events as if they had occurred in a world without human influence on the climate. These `counterfactual' forecasts can be used to assess how not only the intensity, but also the probability of such events has changed. Although extreme weather attribution typically focuses on the past, the same approach could be used to produce forecasts in warmer future worlds --- thus providing vital information about how the most damaging weather may be expected to change in the future. I explore this theme of extreme weather projection, examining a novel approach to producing large climate model ensembles that span the range of uncertainty in future extreme weather. This work complements the specific nature of extreme event attribution, and they could together provide crucial information about climate risk. 
\end{abstract}

%%%%% MINI TABLES
% This lays the groundwork for per-chapter, mini tables of contents.  Comment the following line
% (and remove \minitoc from the chapter files) if you don't want this.  Un-comment either of the
% next two lines if you want a per-chapter list of figures or tables.
\dominitoc % include a mini table of contents
%\dominilof  % include a mini list of figures
%\dominilot  % include a mini list of tables

% This aligns the bottom of the text of each page.  It generally makes things look better.
\flushbottom

% This is where the whole-document ToC appears:
\tableofcontents

\listoffigures
	\mtcaddchapter
% \mtcaddchapter is needed when adding a non-chapter (but chapter-like) entity to avoid confusing minitoc

% Uncomment to generate a list of tables:
%\listoftables
%	\mtcaddchapter

%%%%% LIST OF ABBREVIATIONS
% This example includes a list of abbreviations.  Look at text/abbreviations.tex to see how that file is
% formatted.  The template can handle any kind of list though, so this might be a good place for a
% glossary, etc.
% First parameter can be changed eg to "Glossary" or something.
% Second parameter is the max length of bold terms.
\begin{mclistof}{List of Abbreviations}{3.2cm}

    \item[CDF] Cumulative distribution function
    \item[CI] Confidence interval 
    \item[CO$_2$] Carbon dioxide
    \item[DJF] December--January--February (meteorological winter)
    \item[ECMWF] European Centre for Medium-range Weather Forecasts
    \item[EV] Extreme value
    \item[GCM] General circulation model
    \item[GEV] Generalised extreme value
    \item[GMST] Global mean surface temperature
    \item[IFS] Integrated forecasting system
    \item[JJA] June--July--August (meteorological summer)
    \item[MSLP] Mean sea level pressure
    \item[PDF] Probability density function
    \item[PPE] Perturbed parameter ensemble
    \item[$P$] Indicates a quantified probability
    \item[PR] Probability ratio
    \item[SIC] Sea ice concentration
    \item[SST] Sea surface temperature
    \item[UK] United Kingdom
    \item[UKCP] UK Climate Projections
    \item[UKMO] UK Met Office
    \item[WWA] World Weather Attribution

\end{mclistof}

% The Roman pages, like the Roman Empire, must come to its inevitable close.
\end{romanpages}

% NJL: this sets the spacing in the main text
\onehalfspacing

%%%%% CHAPTERS
% Add or remove any chapters you'd like here, by file name (excluding '.tex'):
\flushbottom
\begin{savequote}[8cm]
    Quote
      \qauthor{--- author}
\end{savequote}
    
\chapter{\label{intro}Introduction} 

In this chapter I introduce the problem of attribution of individual extreme weather events to anthropogenic climate change. I review the current methodologies and frameworks that address this problem, in particular the contrasting storyline and probabilistic approaches to attribution. Although these frameworks are gaining acceptance and maturity, I suggest that a weather forecast-based approach could further increase the trustworthiness of attribution studies. Finally, I provide a conceptual sketch of these various attribution frameworks within a simple non-linear dynamical system.
\small\paragraph{Author contributions:} This chapter is based on the the following publication \footnote{with the author contributing as follows.} \par\vspace{1em}
\formatchref{Surname, I1. I2., Surname, I1. I2.}{year}{Title}{Journal}{vol}{issue}{pages}{DOI}

\minitoc

\clearpage

\section{The problem of extreme event attribution}

  % detection and attribution of climate change (hasselmann->gillet)
  % Being specific about the question ("cause" vs influence, hannart)
  % scientific challenges posed by EEA
  Review papers: \citep{allen_scientific_2007,stott_attribution_2013,stott_attribution_2016,otto_attribution_2016,otto_attribution_2017,swain_attributing_2020,easterling_detection_2016,noauthor_attribution_2016}

\section{Motivating the question}

  Now that I have posed the question, before I move on to how we might answer it, I think it would be useful for me to discuss why we want to answer it. In short: \emph{what's the point of this thesis?}

  % start off with liability (allen03)
  In 2003, Myles Allen wrote \citetitle{allen_liability_2003} \citep{allen_liability_2003}. This commentary is widely acknowledged as the first time the idea that individual extreme events could be attributed to external drivers such as human influence was proposed \citep{otto_attribution_2017}. Although \citeauthor{allen_liability_2003} touched on both methodology and motivation for extreme event attribution, here I shall focus on the latter aspect. The motivation behind extreme event attribution as proposed in \citetitle{allen_liability_2003} is compensation for damage to individuals caused by climate change or, as \citeauthor{allen_liability_2003} puts it, 
  \begin{quote}
    Will it ever be possible to sue anyone for damaging the climate?
  \end{quote}
  \citeauthor{allen_liability_2003} suggests that in the future those affected by particular extreme weather may, given sufficient scientific certainty, be able to claim compensation from greehouse gas emitters for damages caused by the extreme weather. He proposes a framework, grounded in concepts from epidemiology \citep{stone_end--end_2005}, in which emitters pay for the `fraction' of an extreme weather event that they caused, even in the abscence of absolute causation. This fraction is estimated probabilistically based on the change in likelihood of the event in a world in which the emissions never happened (ie. if the event is half as likely to occur without the emissions, then the fraction of the event that is attributable to the emitters is 50\%).
  This specific application is therefore using extreme event attribution as evidence in environmental tort law. Since \citet{allen_liability_2003}, much has been written in relation to this application. \citet{allen_scientific_2007} presents an overview of the state of climate change detection and attribution aimed at legal professionals, concluding with a set of related questions for the legal community. More recently, \citet{stuart-smith_filling_2021} provided a set of suggestions for potential plaintiffs on how to best make use of the climate science available (noting that evidence used in previous cases `lags substantially behind the state of the art'). Coming from the other side of the coin, \citet{marjanac_acts_2017} provide suggestions for climate change scientists, emphasising that `clear and confident expression of science in a manner that can be applied by non-scientists, including lawyers' is key. Elisabeth A. Lloyd has authored a number of studies exploring various issues including the different standards of proof in scientific and legal contexts \citep{lloyd_climate_2021}; how different approaches to extreme event attribution can complement one another to provide the most useful picture of climate change impacts for a broad range of contexts \citep{lloyd_climate_2018,lloyd_environmental_2020}; and finally, examines a specific tort law case that made use of extreme event attribution \citep{lloyd_climate_2021-1}. For thorough review of both the science and legal context of extreme event attribution, written from a legal perspective, with reference to specific cases, I recommend \citet{burger_law_2020,marjanac_extreme_2018}.
  
  In addition to the original application in tort law suggested in \citet{allen_liability_2003}, more recently it has been suggested that extreme event attribution could `play a significant roll in quantifying loss and damage' \citep{wehner_operational_2022}. Loss and damage is generally understood as the unavoidable adverse impacts of climate change \citep{mace_loss_2016}, and has become a key piece of international climate change policy since the inclusion of Article 8 of the Paris Agreement. Several recent extreme event attribution based studies may have considerable influence in the future in this space, including \citet{clarke_inventories_2021}, who set out a framework for recording key details of high-impact weather events as a new source of evidence for global stocktakes on loss and damage; and \citet{otto_assigning_2017,lott_quantifying_2021}, who adapt conventional extreme event attribution approaches to estimate the contributions of \emph{specific} emitters to individual extreme weather events (as opposed to the more usual broad `human influence' considered). Perhaps we don't have long to wait before the question posed by \citeauthor{allen_liability_2003} quoted above is answered?

  The other key nonscientific motivating factor for extreme event attribution is the public engagement and interest in the research  \citep{swain_attributing_2020}. The `headline' number in climate science has for a long time been change in global mean temperature \citep{stocker_climate_2013,ipcc_global_2018,masson-delmotte_climate_2021}. While this is clearly a very important number as the primary metric of the impact that humanity is having on the climate, it is not a number that individuals can easy relate to due to the large scales involved and indirect nature of the associated impacts. On the other hand, extreme weather events are phenomena that are actually experienced in real time by individuals --- and regularly reported on by the media. Since extreme weather events can cause severe and direct socioeconomic impacts \citep{fouillet_excess_2006}, increases in their frequency would likely be a considerably more relatable and concerning consequence of climate change than the distant change in global mean quantities. Previous work has shown that extreme event attribution may be an exceptionally useful tool for climate change communication \citep{ettinger_whats_2021}, though can prove unhelpful if results are not clear and comprehensible for a general audience \citep[for example if different attribution studies regarding a single event appear to provide conflicting headline results][]{osaka_natural_2020}. A specific study investigating the experience-perception link of climate disaters in the context of Floridians that had experienced hurricane Irma found that this experience increased both their belief and concern in global warming \citep{bergquist_experiencing_2019}, though a more recent study looking at connections between local weather and climate change awareness in Germany did not find a link \citep{gartner_experiencing_2021}.

  % improve models (high benchmark of test) - decide whether to include
  \citep{sillmann_understanding_2017}

\section{Answering the question}

  At this point, I have discussed the question that this thesis is primarily concerned with, and the reasons why it is of broad importance. Now, I shall describe and explore the ways in which previous studies have approached this question, in particular focusing on the probabilistic \citep[often ``conventional'',][]{stott_human_2004} and storyline \citep{hoerling_anatomy_2013} frameworks. 

  \subsection{Probabilistic attribution}

    Probabilistic attribution seeks, ultimately, to determine the change in probability of an extreme event arising due to some external driver. This was the approach to extreme event attributon proposed by \citet{allen_liability_2003} and first applied by \citet{stott_human_2004} to the 2003 European heatwave. They applied an optimal fingerprinting technique \citep{hasselmann_optimal_1993,hasselmann_multi-pattern_1997} to transient climate model simulations. They used five simulations, one set of four with all forcings included, plus one with natural forcings only; generating scaling factors of the correspondence between the modelled response and observed changes by regressing each set onto observed central European summer temperature. The scaling factors could then be used to determine the 1990s temperature anomalies attributable to all forcings combined and natural forcings alone. A third control run at a fixed, pre-industrial level of forcing was used to estimate internal variability corresponding likelihood functions for these temperature anomalies. They finally used a peak-over-threshold extreme value analysis of the same control run to determine the probability of exceeding the temperature of the pre-2003 hottest summer in worlds with and without climate change by adjusting the mean summer temperature to the estimated 1990s temperature anomalies both with and without anthropogenic forcing. These probabilities (and associated uncertainty) could be used to determine the likelihood function of the change in risk of the heatwave attributable to human influence. This fairly involved statistical approach (in particular, the necessity for the use of a control run to estimate uncertainty due to internal variability) has been largely replaced by the use of much larger single- or multiple- model ensembles.

    The next advance in probabilistic extreme event attribution came with \citet{van_oldenborgh_how_2007}, who developed a methodology for estimating the change in risk of an extreme event using observations alone. \citeauthor{van_oldenborgh_how_2007} took observed timeseries of autumn temperatures measured by the De Bilt weather station, and removed the climate change signal by regression onto low-pass filtered global mean surface temperature (a reasonable proxy for anthropogenic influence on the climate, given the small contributions from natural forcing). Using this detrended series and extreme value analysis, he then computed the return period of the exceptional warm 2006 autumn in Europe, and compared it to the return period estimated using the original series. This method was later extended to use the return period to compute the return period of the extreme for both the present-day and pre-industrial period by shifting the detrended series by the attributable trend computed in the regession \citep[eg.][]{philip_protocol_2020,leach_anthropogenic_2020}. In this way, the change in risk between pre-industrial and present climates can be estimated. It is worth noting that this methodology does not formally attribute and changes in risk to anthropogenic influence, since trends in local climate may be influenced by other factors, and no use is made of a counterfactual world without human influence on the climate (since no observations of such a world exist). This method can also be applied to transient simulations from climate models.

    The final advance that I shall highlight is from \citet{pall_anthropogenic_2011}, and was the first instance where specific fixed forcing (as opposed to transient) factual and counterfactual simulations were used. \citeauthor{pall_anthropogenic_2011} generated very large (2000+ member) atmosphere-only climate model ensembles of autumn 2000. One ensemble was driven using observed sea surface temperatures and sea ice, and corresponding atmospheric conditions (greenhouse gas, aerosol and ozone concentrations) for that time. The other four were driven using atmospheric conditions for the year 1900, and subtracting four estimates of attributable twentieth-century SST warming from the observed sea surface temperatures; the four estimates of attributable warming were derived from four different coupled climate model simulations. River runoff for England and Wales in the factual ensemble was compared to runoff in the four `naturalised' counterfactual ensembles to determine the difference in risk of exceeding the value actually observed in autumn 2000 in the different climates. In this case, the ensembles are sufficiently large that extreme value analysis was not required (the -- SST conditional -- return period could be calculated by simply counting the number of members that exceeded the observed threshold and dividing by the total ensemble size in each case). This methodology is attractive due to the clear and straightforward statistical analysis of the large ensembles (with no reliance on extreme value analysis or optimal fingerprinting techniques). However, it does generally require very large ensembles of the time period when the event took place (ie. autumn 2000 in this case); and is conditional on the prescribed SST pattern (requiring either that anthropogenic influence isn't affecting interannual modes of SST variability, or that the extreme in question is independent of such modes).

    The `standard' approach to probabilistic attribution in the present draws upon each of these previous advances \citep{philip_protocol_2020}. Here I am taking the World Weather Attribution project (WWA) methodology as standard, since they are the most prolific group both in terms of number of events analysed and media coverage of their results \citep{van_oldenborgh_pathways_2021}. Their approach involves:
    \begin{enumerate}
      \item Defining the event. Extreme events are, by the nature of the weather, exceptionally high dimensional and could be described in a practially unlimited number of ways (ie. what variables to use? What spatial scale? What temporal scale?). However, to be able to analyse changes to such events, we must be able to define them quantatively. The WWA attempts to select a metric that most closely corresponds to the impacts of the extreme event in question, taking in account what questions are being asked by the various stakeholders. For example, if the key impact of interest is heatwave-associated mortality, then peak 3-day moving average daily maximum temperatures may be selected due to their close connection to health impacts \citep{dippoliti_impact_2010}. Once a metric has been chosen by which to define the extreme event, they use a `class-based' framing considering all events of a similar magnitude, often by choosing the annual maximum values of the metric. This class-based framing results in a largely unconditional analysis, which I will discuss further below.
      \item Analysing trends in observed data following \citet{van_oldenborgh_how_2007}. This is typically done by fitting an appropriate distribution whose parameters shift or scale with low-pass filtered global mean surface temperature (GMST). The shifting or scaling depends upon the chosen metric and its observed or expected response to climate change. For example, the temperature based heatwave metrics that are the primary concern of this thesis are typically assumed to simply shift with global warming. From this GMST-covarying distribution, the return period is then computed for present-day and pre-industrial values of GMST. From these returns periods the change in risk of the observed extreme can be calculated. As with \citet{van_oldenborgh_how_2007}, these changes in risk are not strictly attributable to human influence on the climate due to the lack of a no-human counterfactual.
      \item Analysing simulations from as many models as possible. Transient model simulations are analysed in an identical way to observations. Fixed forcing simulations are analysed following \citet{pall_anthropogenic_2011}. Only models which are able to closely represent the observed climate are considered -- evaluated on the basis of the trends and distribution parameters implied by the model data.
      \item Synthesising these various strands of evidence. The aim behind using as many lines of evidence as possible, combining statistical and numerical models, is to try and determine the most robust conclusion possible within the context of the associated uncertainties. 
    \end{enumerate}
    This `standard' approach has been refined over the past decade by the WWA team \citep{van_oldenborgh_pathways_2021}, learning through application to a wide range of locations and types of extreme. The widespread recognition and understanding of extreme event attribution by the general public is due, in no small part, to this approach and how rapidly the WWA team have been able to apply it to extreme events in the past few years. Their rapid response has meant that they are able to answer the questions people ask in the aftermath of such events when they are actually asking them -- rather than following a lengthy peer-review process. However, this standard probabilistic approach to attribution is not without issues of its own -- hence this thesis -- which I shall now discuss.

    The first issue arises due to the unconditional use of climate models. By their nature, extreme events are typically produced by exceptional physical processes, or combinations of processes. The type of models used in extreme event attribution are typically coarse (O(100 km)), and may well not be able to physically represent all the processes involved in the production of specific extreme events even if they can accurately represent the average climate \citep{sillmann_understanding_2017,trenberth_attribution_2015}. Such models still have serious known biases relevant to the simulation of extremes, including poor representation of Euro-Atlantic blocking \citep{schiemann_resolution_2017,dorrington_how_2021}, which is a key synoptic driver of heatwaves over the continent. These biases become even more important when considering not only the models' ability to simulate the present climate, but also their response to external forcings such as anthropogenic climate change \citep{palmer_nonlinear_1999,palmer_simple_2018}. The use of biased models can lead to potentially incorrect quantitative attribution statements \citep{bellprat_attribution_2016,bellprat_towards_2019}. 
    
    The second key issue derives from the treatment of individual extremes as one of an event class. For example, in the conventional probabilistic approach to attributing a particular heatwave, one might consider all the previous annual maximum temperatures (eg. in order to apply extreme value analysis as discussed above). However, the particular heatwave in question might have arisen from a very different -- possibly unique in the context of the relatively short historical record -- set of meteorological circumstances and drivers compared to all the previous heatwaves. This not only renders such extreme value analysis as is often performed potentially inappropriate (as the heatwave in question is drawn from a different underlying distribution to the others), but also any estimated climate change responses. What if the combination of the particular processes involved in producing the heatwave in question responds fundamentally differently to the processes that have generated past heatwaves?
    
    The final issue I shall discuss is the one that has been most often commented on in previous work: the risk of type II errors \citep{shepherd_common_2016,trenberth_attribution_2015}. This risk arises because some aspects of the climate system response to external forcing are much more certain and well-understood than others. For example, the thermodynamic aspects of climate change are broadly very well understood and certain: rising greenhouse gas concentrations lead to a thicker troposphere, thus raising surface temperatures and increasing the moisture capacity of the troposphere. On the other hand, the dynamic aspects of climate change are considerably less certain, with models often disagreeing over the direction of changes in atmospheric dynamics \citep{masato_winter_2013}. At this point, I note that this is not an entirely independent issue to the first issue discussed since much of this uncertainty arises due to model biases and imperfections. Since extreme events are often driven by a combinations of both thermodynamic and dynamic processes, the very certain thermodynamic climate change impacts can be masked to some extent by the much less certain dynamic climate change impacts. This uncertainty can lead to falsely asserting that there is no overall impact -- a type II error. \citet{trenberth_attribution_2015} argue that it is better to focus on the aspects of the event that are well understood, for example by conditioning analyses upon the large scale circulation of the event in question, thus removing the potentiall very uncertain dynamic aspects of climate change. This suggestion was extended and discussed at length by \citet{shepherd_common_2016}, and has become the basis for the new kid on the block in extreme weather attribution: the storyline approach.

  \subsection{Attribution through storylines}

    The storyline approach \citep[or `Boulder' approach,][]{otto_attribution_2017} aims to determine the contribution of various causal factors that to the extreme event, and considers how anthropogenic climate change has affected those factors (and thus the extreme) in a deterministic manner. This approach was first applied by \citet{hoerling_anatomy_2013} to the 2011 summer combined Texas heatwave and drought. They used a variety of simulations, including atmosphere-only and coupled climate model runs, and seasonal forecasts. They examined the influence of rainfall deficit in the months preceding the heatwave, SST influence on the drought, and the overall predictability of the extreme at the start of May. As such, their intended goal was much broader than just assessing the anthropogenic contribution to the heatwave, aiming to advance the overall predictability of such events by examining all causes, human and natural. 
    % the aims of storyline approaches
    % Hoerling - hazelger - shepherd - benitez
    \citep{hazeleger_tales_2015,shepherd_common_2016,benitez_july_2022}
    % issues: does not provide estimate of change in risk

  \subsection{Do we need new approaches?}

  \subsection{Operationalising attribution}

\section{Conceptualising different approaches}

  % use the dynamicist's favourite "toy": Lorenz63
  % demonstrate "storyline / DADA", conventional attribution
  % propose forecast-based attribution
  \blindtext

\section{What to expect in this thesis}

  % exploratory conventional attribution study 
  % tangent study about climate change projections + link to forecast based approaches
  % first step to forecast-based attribution - partial attribution
  % "full" forecast-based attribution
  % conclusion discussing current limitations + future directions
  \blindtext

\begin{savequote}[8cm]
    Quote
      \qauthor{--- author}
\end{savequote}
    
\chapter{\label{ch1}Conventional probabilistic attribution} 

Here I present a probabilistic extreme event attribution of the 2018 European heatwave. Whilst demonstrating the methodologies behind this framework, I examine how one particular aspect of probablistic event attribution -- the definition of the event -- projects strongly onto the quantitative results. In the closing remarks, I reflect on potential issues with the approach taken within the chapter, and suggest ways in which these could be overcome.
\small\paragraph{Author contributions:} This chapter is based on the the following publication \footnote{with the author contributing as follows. Conceptualisation, Data curation, Formal analysis, Investigation, Methodology, Resources, Visualisation and Writing -- original draft.} \par\vspace{1em}
\formatchref{Leach, N. J., Li, S., Sparrow, S., van Oldenborgh, G. J., Lott, F. C., Weisheimer, A., \& Allen, M. R.}{2020}{Anthropogenic Influence on the 2018 Summer Warm Spell in Europe: The Impact of Different Spatio-Temporal Scales}{Bulletin of the American Meteorological Society}{101}{1}{S41-S46}{https://doi.org/10.1175/BAMS-D-19-0201.1}

\clearpage

\minitoc

\clearpage

\section{Chapter open}

\section{Abstract}

  We demonstrate that, in attribution studies, events defined over longer time scales generally produce higher probability ratios due to lower interannual variability, reconciling seemingly inconsistent attribution results of Europe's 2018 summer heatwaves in reported studies.

\section{The 2018 heatwave in Europe}

  The summer of 2018 was extremely warm in parts of Europe, particularly Scandinavia, the Iberian Peninsula, and central Europe, with a range of all-time temperature records set across the continent \citep{johnston_heatwave_2018,nesdis_record_2018}. Impacts were felt across Europe, with wildfires burning in Sweden \citep{krikken_attribution_2021,watts_wildfires_2018}, heatstroke deaths in Spain \citep{publico_nueve_2018}, and widespread drought \citep{harris_heat_2018}. During the summer, the World Weather Attribution (WWA) initiative released an analysis of the heat spell \citep{world_weather_attribution_heatwave_2018} based on observations/forecasts and models in specific locations (Dublin, Ireland; De Bilt, Netherlands; Copenhagen, Denmark; Oslo, Norway; Linkoping, Sweden; Sodankyla, Finland; Jokionen, Finland), which concluded that the increase in likelihood due to human induced climate change was at least 2 to 5 times. In December, the U.K. Met Office (UKMO) stated that they found the 2018 U.K. summer temperatures were made 30 times more likely \citep{press_office_chance_2018,mccarthy_drivers_2019}. These two estimates appear to quantitatively disagree; however, we show they can be reconciled by considering the effect of using different spatial domains and temporal scales in the event definition. We also demonstrate that prescribed SST model simulations can underrepresent the variability of temperature extremes, especially near the coast, with implications for any derived attribution results.

  % \clearpage
  \begin{figure}[h]
    \centering
    \includegraphics[width=\textwidth]{{Fig1.1}.pdf}
    \caption[The 2018 heatwave in Europe: observed mean temperature anomalies over a range of timescales]{\textbf{The 2018 heatwave in Europe: observed mean temperature anomalies over a range of timescales.} Shading indicates mean temperature anomalies for the the different temporal-scale heatwave metrics used. Black contours indicate z-scores of the 2018 heatwave for the three metrics based on detrended historical data from E-OBS. The contours indicate scores of 1-, 2-, and 3-σ, in order of increasing thickness.}
    \label{fig2.1}
  \end{figure}
  % \clearpage

\subsection{Defining the event}

  We consider various temperature-based event definitions to demonstrate the impact of this choice in attribution assessments, and assess to what extent human influence affected the seasonal and peak magnitudes of the 2018 summer heat event on a range of spatial scales. The metric we use is the annual maximum of the 1-, 10-, and 90- day running mean of daily mean 2-m temperature (hereafter TM1x, TM10x, and TM90x respectively). We analyze two spatial scales: model grid box and regional. For regional event definitions, the spatial mean is calculated before the annual maximum. Regional domains are taken from \citet{christensen_summary_2007}. Figure \ref{fig2.1} shows the 2018 anomaly field for each of these metrics. In their study, the WWA used the annual maxima of 3-day mean daily maximum temperatures at specific grid points for its connection to local health effects \citep{dippoliti_impact_2010}, whereas the UKMO used the JJA mean temperature over the entire United Kingdom in order to answer the question of how anthropogenic forcings have affected the likelihood of U.K. summer seasons as warm as 2018. The same justifications can be used here, although we add that different heat event time scales are important to different groups of people, and as such using several temporal definitions may increase interest in heat event attribution studies. However, we recognize that other definitions than those used here may be more relevant to some impacts observed (such as defining the event in the context of the atmospheric flow pattern and drought that accompanied the heat), and other lines of reasoning for selecting one particular event definition exist \citep{cattiaux_defining_2018}.

\section{Model simulations \& validation}

  We primarily use three sets of simulations from the UKMO Hadley Centre HadGEM3-A global atmospheric model \citep{christidis_new_2013,ciavarella_upgrade_2018}. These are a historical ensemble (1960--2013; Historical), and factual (ACT) and counterfactual (a “natural” world without anthropogenic forcings; NAT) ensembles of 2018. We compare results from this factual-counterfactual analysis with those from a trend-based analysis of Historical, ensembles from EURO-CORDEX \citep{vautard_simulation_2013,jacob_euro-cordex_2014,vrac_influence_2017} (1971--2018) and RACMO \citep{aalbers_local-scale_2018,lenderink_preparing_2014} (1950--2018), and observations from E-OBS (1950--2018). A full model description is provided in the online supplemental information. Initially, we performed our analysis with the weather@home HadRM3P European-25 km setup \citep{massey_weatherhome-development_2015} but found that this model overestimates the variability over all Europe for daily through seasonal-scale event statistics, and so it was omitted.

\section{Methods}

  \subsection{Estimating the event threshold}

    We first use historical data to estimate the return time of the 2018 event, and the corresponding temperature threshold in each model. We start by calculating the return period for the 2018 event in E-OBS. Since the distribution of temperature extremes changes as the climate changes, to account for the non-stationarity of the time series we remove the attributable trend by regressing onto the anthropogenic component of globally averaged mean surface temperature \citep[the anthropogenic warming index, based on HadCRUT5;][]{haustein_real-time_2017,morice_updated_2021}. The regression coefficient or trend is shown in the supplemental material in Fig. ES1 \citep{diffenbaugh_quantifying_2017}. We then fit extreme value (EV) distribution parameters to this detrended E-OBS time series, and use these parameters to calculate the estimated return period of the 2018 event. We then find the temperature threshold in the model climatology that corresponds to this return period. We do this by fitting EV parameters to a detrended (by regressing onto the anthropogenic warming index, trends shown in Figs. ES2c7--9) climatological ensemble for each model. For HadGEM3-A, the climatology is Historical plus 15 randomly sampled members of ACT; for RACMO and EURO-CORDEX, the climatology is taken as the entire set of simulations described above. The calculation of model-specific climatological temperature thresholds from the E-OBS temperature threshold is illustrated in Figure \ref{fig2.2}. Using estimated event probabilities rather than observed magnitudes to define the event constitutes a quantile bias correction \citep{jeon_quantile-based_2016}, accounting for model biases in the mean and variability of the temperatures simulated.

    % \clearpage
    \begin{figure}[h]
      \centering
      \includegraphics[width=\textwidth]{{Fig1.2}.pdf}
      \caption[Calculating the heatwave threshold in HadGEM3-A from observations]{\textbf{Calculating the heatwave threshold in HadGEM3-A from observations.} The heatwave metric used here for illustration is TM1x. Solid black lines indicate empirical CDFs. Solid grey lines indicate GEV distributions fit to the data, with grey shading indicating confidence intervals of the fit. The dotted black line indicates the temperature observed during the 2018 heatwave in E-OBS. Dotted grey lines indicate the best-estimate quantile corresponding to the 2018 event in E-OBS based on the GEV fit, and the temperature of that quantile in the HadGEM3-A historical climatology.}
      \label{fig2.2}
    \end{figure}
    % \clearpage

  \subsection{Counterfactual attribution}

    For the main analysis, which we term ``counterfactual'' attribution \cite{stott_human_2004}, we estimate the probability ($P$) of exceeding this climatological temperature threshold in the ACT and NAT ensembles. We do this by fitting EV parameters to each ensemble, and using them to calculate $P_{\text{ACT}}$ and $P_{\text{NAT}}$. The estimate ACT and NAT distributions are shown for each metric in Figure \ref{fig2.3}. We express our results as the probability ratio, $PR = P_{\text{ACT}}/P_{\text{NAT}}$, representing the fractional change in probability of the 2018 event in the factual compared to the counterfactual world.

  \subsection{Trend-based attribution}

    We support the counterfactual attribution with a trend-based analysis \cite{van_oldenborgh_absence_2012} of E-OBS and all the model ensembles used. This trend-based analysis is based on the climatology alone, and does not require factual and counterfactual simulations. We start with the EV parameters fit to the detrended climatology, and then use the estimated climate change trend between 1900 and 2018 to shift the location of the EV distribution. This shifted distribution then represents the counterfactual distribution (analogous to NAT), and the original distribution represents the factual distribution (analogous to ACT), from which we can calculate probability ratios.

  \subsection{Statistical methods}

    We fit EV parameters using the method of L-Moments \cite{hosking_l-moments_1990}, modelling TM1x and TM10x using the generalized extreme value (GEV) distribution, and TM90x using the generalized logistic distribution. Uncertainties are calculated using a 10,000 resample nonparameteric bootstrap throughout.

  % \clearpage
  \begin{figure}[h]
    \centering
    \includegraphics[width=\textwidth]{{Fig1.3}.pdf}
    \caption[Factual and counterfactual PDFs of the 2018 heatwave defined over three temporal scales]{\textbf{Factual and counterfactual PDFs of the 2018 heatwave defined over three temporal scales.} The heatwave metric used is given in the title of each panel. Solid red and blue lines indicate best-estimate GEV distributions fit to HadGEM3-A 2018 ACT and NAT ensembles respectively. Dotted grey line indicates 2018 event threshold defined using HadGEM3-A and E-OBS climatology (see Figure \ref{fig2.2}). Shading illustrates confidence intervals.}
    \label{fig2.3}
  \end{figure}
  % \clearpage

\section{Results}
  
  Extreme daily heat events, measured by TM1x, are distributed heterogeneously throughout Europe. This is paralleled in the probability ratios seen in Figure \ref{fig2.4}, with large areas of the Iberian Peninsula, the Netherlands, and Scandinavia experiencing events that were highly unlikely in a climate without anthropogenic influence. A similar result is found on the regional scale in Figure \ref{fig2.5} with Scandinavia and the Iberian Peninsula respectively experiencing 1-in-150 [25--25,000]\footnote{numbers in brackets [] represent a 90\% confidence interval.} and 1-in-30 [9--550] year events in the current climate that were highly unlikely in the natural climate simulated in NAT. The remaining regions recorded maximum daily temperatures expected to be repeated within 4 years. The probability ratios for regional domains are typically larger than single gridboxes within them, though some regions contain clusters of extremely high probability ratios. This result is consistent with Uhe et al. (2016) and Angélil et al. (2018), who showed that increasing the spatial scale over which the event is defined tends to increase the probability ratio.

  Extreme 10-day heat events, TM10x, were also widespread in Europe, with the most extreme occurring in Scandinavia (Fig. ES1j). Regionally, the PRs become more uniform (Figure \ref{fig2.5}), although Scandinavia and the Iberian Peninsula still have very high best-estimate PRs of 800 [20--infinite] and 85 [25--40,000] respectively.

  The PR map for season-long heat events measured by TM90x is more uniform throughout Europe (Figure \ref{fig2.4}). Scandinavia, the British Isles, France, and central and eastern Europe all experienced on the order of 1-in-10 year events (Fig. ES1l). The corresponding best-estimate PRs are between 10 and 100 for all regions (Fig. 1d), including those with lower return periods.

  % \clearpage
  \begin{figure}[h]
    \centering
    \includegraphics[width=\textwidth]{{Fig1.4}.pdf}
    \caption[Maps of the estimated change in probability of the 2018 heatwave due to anthropogenic influence on the climate]{\textbf{Maps of the estimated change in probability of the 2018 heatwave due to anthropogenic influence on the climate.} The heatwave metric used is given in the title of each panel. Shading illustrates risk ratio of 2018 event at each gridpoint computed using HadGEM3-A ACT and NAT ensembles.}
    \label{fig2.4}
  \end{figure}
  % \clearpage

  A trend-based analysis yields similar results, with PRs for the British Isles region shown in Figure \ref{fig2.6}, though we note that for HadGEM-3A this results in generally higher PRs than the corresponding counterfactual analysis, due to the attributable anthropogenic trend in the climatology being greater than the mean difference between the ACT and NAT ensembles. For the vast majority of the regions and metrics analysed here, the trend-based observational, trend-based model, and counterfactual model estimates of the return period are consistent with one another, and Figure \ref{fig2.5} is a good representation of the synthesis of these different approaches and data sources. However, there are a few regions with notable discrepancies. The uncertainty in the E-OBS observed Scandinavia region TM1x trend are large enough that a 90 \% confidence interval is not able to rule out a negative trend. Hence the corresponding probability ratio confidence interval includes values of less than 1 (i.e. that TM1x events at least as hot as the 2018 event have been made less likely by climate change). This interval is large enough that it does still overlap with all the model-derived estimates, all of which suggest that the probability ratio is very likely greater than 70. This very large interval may arise due to natural variability affecting the relatively small sample size. Synthesizing these strands of information, we suggest that such daily extreme heat events over Scandinavia have been made more likely by climate change, but we are cautious about drawing very firm conclusions. The other clear discrepancy is for the TM90x metric British Isles results, shown in Figure \ref{fig2.6}. Despite good agreement between all other approaches and sources, the trend-based HadGEM3-A estimate is an order of magnitude higher and does not overlap with the others. This appears to be due to the variability of British Isles temperatures on this \textasciitilde seasonal timescale being underestimated by this model, even though the estimated trend closely matches the other models and observations. We discuss this further below. 

  % \clearpage
  \begin{figure}[h]
    \centering
    \includegraphics[width=\textwidth]{{Fig1.5}.pdf}
    \caption[Estimated changes in probability of the 2018 heatwave defined using regional mean temperatures]{\textbf{Estimated changes in probability of the 2018 heatwave defined using regional mean temperatures.} Color indicates heatwave metric. Dots indicate central risk-ratio estimate and bars indicate 90 \% confidence interval.}
    \label{fig2.5}
  \end{figure}
  % \clearpage

  The PR increases with the event statistic time scale for the majority of grid points and regions, demonstrated in Figures \ref{fig2.4} and \ref{fig2.5}. Figure \ref{fig2.3} illustrates the cause using the British Isles region: as the time scale increases, the variance in the event metric decreases, while the mean shift between the factual and counterfactual distributions remains comparable. The similarity in attributed anthropogenic trend for the three time scales is also true in the observations and other models. The decrease in variance usually results in higher PRs, given a particular event return time, for the longer time scales. There are exceptions due to the bounded upper tail of a GEV distribution with a negative shape parameter, resulting in the very high estimated PRs for TM1x in Scandinavia, the Iberian Peninsula, and the Netherlands (Figure \ref{fig2.5}). Now focussing on the British Isles region, Figure \ref{fig2.3} also shows another reason why the TM90x metric probability ratios are much higher than the TM10x or TM1x results: in addition to the decreased variance in the TM90x metric, the 2018 event was more unusual when measured with this metric (a return period of 10.3 [5.7--20] years) compared to the two shorter timescale metrics (return periods of 2.5 [1.7--3.8] and 3.6 [2.4--6.0] for TM10x and TM1x respectively). These two factors (reduced variance and rarer event) result in best-estimate probability ratios of 3.7 [2.9--4.9] for TM1x and 29 [17--57] for TM90x. We therefore suggest that changes in the variance of the event metric as the time scales used changes largely reconciles the differences between the ``2 to 5'' and ``30'' times increases in likelihood found by the WWA and UKMO reports, with other methodological factors, such as the spatial scale used in the event definition, playing a more minor role as we have demonstrated for the British Isles.

  % \clearpage
  \begin{figure}[h]
    \centering
    \includegraphics[width=\textwidth]{{Fig1.6}.pdf}
    \caption[Estimated changes in probability of the 2018 British Isles heatwave across a range of observations and model simulations]{\textbf{Estimated changes in probability of the 2018 British Isles heatwave across a range of observations and model simulations.} Here, risk-ratios are estimated using a trend-based analysis. Color indicates data source. Dots indicate central risk-ratio estimate and bars indicate 90 \% confidence interval.}
    \label{fig2.6}
  \end{figure}
  % \clearpage

  As mentioned above, the trend-based HadGEM3-A analysis appears to overestimate the probability ratio of the 2018 event when considering the other approaches taken here (Figure \ref{fig2.5}). This is due to an important deficiency in the model: the model distributions are narrower than the observed distributions for this heatwave metric, meaning the model has lower variability in temperatures on seasonal timescales than the real world. This reduced variance has a significant impact on attribution results \citep{bellprat_towards_2019} and means that the PRs for the British Isles presented here, especially for TM90x, are likely to be overestimated. Underrepresented variability often occurs in prescribed-SST models \citep{fischer_biased_2018,he_does_2016} and is present in HadGEM-3A for many coastal gridboxes in Europe. Figure \ref{fig2.7} shows the power spectrum of JJA summer temperatures over the British Isles, indicating that HadGEM3-A has broadly similar spectral characteristics to E-OBS, but underrepresents the intraseasonal 2-m temperature variability at almost all frequencies, which will likely result in overestimated PRs. Power spectra for other model ensembles are shown for comparison, demonstrating that only the fully bias-corrected EURO-CORDEX ensemble has variability characteristics and magnitude that closely match the observations.

  % \clearpage
  \begin{figure}[h]
    \centering
    \includegraphics[width=\textwidth]{{Fig1.7}.pdf}
    \caption[Historical power spectrum of summer daily mean temperatures over the British Isles across a range of observations and model simulations]{\textbf{Historical power spectrum of summer daily mean temperatures over the British Isles across a range of observations and model simulations.} Power spectra are estimated as periodograms averaged over all historical years available for each data source, expressed as a fraction of the E-OBS power at each frequency. Color indicates data source. Thick lines show ensemble mean power for each source. Thin lines show individual ensemble members for each source.}
    \label{fig2.7}
  \end{figure}
  % \clearpage

\section{Discussion}

  Our analysis highlights a key property of extreme weather attribution: the variance of the event definition used, both in terms of the statistic itself and its representation within any models used. The use of longer temporal event scales in general increases both the spatial uniformity and magnitude of the probability ratios found, consistent with \citet{kirchmeieryoung_importance_2019}, due to a decrease in variance compared to shorter scales. The difference in temporal scale between two reports concerning the 2018 summer heat is sufficient to explain the large discrepancy in attribution result between them. We find that several European regions experienced season-long heat events with a present-day return period greater than 10 years. The present-day likelihood of such events occurring is approximately 10 to 100 times greater than a ``natural'' climate without human influence. The attribution results also show that the extreme daily temperatures experienced in parts of Scandinavia, the Netherlands, and the Iberian Peninsula would have been exceptionally unlikely without anthropogenic warming. The prescribed-SST model used primarily here tends to underestimate the variability of temperature extremes near the coast, which may lead to the attribution results overstating the increase in likelihood of such extremes due to anthropogenic climate change \citep{bellprat_towards_2019}. We aim to properly quantify the impact of the underrepresented variability in further work. Although here we have used an unconditional temperature definition for consistency with the studies we aimed to reconcile, we plan to further investigate the effect of including both the atmospheric flow context and other impact-related variables such as precipitation in the event definition, and address issues models might have with realistically simulating the physical drivers of heatwaves.

\section{Chapter close}
%% Thoughts on this study in the context of the Thesis


\begin{savequote}[8cm]
    Quote
      \qauthor{--- author}
\end{savequote}
    
\chapter{\label{ch2}Chapter 2} 

Chapter description.
\small\paragraph{Author contributions:} This chapter is based on the the following publication \footnote{with the author contributions as follows.} \par\vspace{1em}
\formatchref{Leach, N. J., Watson, P. A. G., Sparrow, S. N., Wallom, D. C. H., \& Sexton, D. M. H.}{2022}{Generating samples of extreme winters to support climate adaptation}{Weather and Climate Extremes}{36}{}{100419}{https://doi.org/10.1016/j.wace.2022.100419}

\minitoc

\clearpage

\section{Section}

    \blindtext

{\onehalfspacing%
\begin{savequote}[8cm]
    Quote
      \qauthor{--- author}
\end{savequote}
    
\chapter{\label{ch3}Forecast-based attribution: perturbing the boundary conditions}

This chapter contains much of the conceptual description of, and motivation for, forecast-based attribution. Using the well-predicted February 2019 heatwave as a case study, I carry out forecasts with the operational medium-range ECMWF model in which I have instantaneously perturbed the CO$_2$ concentration at initialisation. These perturbed forecasts allow me to estimate the direct contribution of diabatic heating due to CO$_2$ to the heatwave. This partial attribution provides a proof-of-concept of the forecast-based approach, and I close with a discussion of how I could perform a more complete estimate of anthropogenic influence on a specific extreme event in following work.
{\small\paragraph{Author contributions:} This chapter is based on the following publication \footnote{with the author contributing as follows. Conceptualisation, Data curation, Formal analysis, Investigation, Methodology, Resources, Visualisation, Writing -- original draft and Writing --- Review \& Editing.} \par\vspace{1em}
\formatchref{Leach, N. J., Weisheimer, A., Allen, M. R., \& Palmer, T.}{2021}{Forecast-based attribution of a winter heatwave within the limit of predictability}{Proceedings of the National Academy of Sciences}{118}{49}{}{https://doi.org/10.1073/pnas.2112087118}}

\clearpage

\minitoc

\clearpage}

\section{Chapter open}\label{ch3:open}

  Just after the exceptionally warm period in February 2019 that is the subject of this chapter, my co-authors and I discussed this extreme event. Two features were particularly noteworthy: it appeared to be a particularly radiatively driven heat event; and it had been forecast exceptionally well at least a week in advance. Although at this point we had already discussed performing counterfactual forecasts using perturbed initial conditions, we hadn't yet worked out how we would actually achieve this in practice, and it seemed a long way off. Because of this, and the apparent radiative nature of the event, we wondered if we could start off by simply changing the CO$_2$ concentrations in the model --- and leaving everything else the same. Although this would only represent a very partial attribution, to the direct radiative effect of increased CO$_2$ concentrations over pre-industrial levels, \citet{baker_higher_2018} had recently published a study that suggested we might still find that this direct effect would be sufficient to notice a difference. It turned out that changing the CO$_2$ concentrations in the model was relatively straightforward, and after waiting some time for computing resource to be granted, we were able to perform these perturbed-CO$_2$ counterfactual forecasts. This partial attribution would allow us to begin exploring a number of questions we had on the approach: how would the predictability of the heatwave change when we changed the CO$_2$ levels?; how would the attribution statements depend on lead time?; would the direct effect of CO$_2$ be large enough for us to even detect it? All of these questions are relevant to not only the partial attribution presented in this chapter, but also the more complete attribution based on perturbed boundary \emph{and} initial condition forecasts that is the primary aim of this project.

\section{Abstract}\label{ch3:abstract}

  Attribution of extreme weather events has expanded rapidly as a field over the past decade. However, deficiencies in climate model representation of key dynamical drivers of extreme events have led to some concerns over the robustness of climate model-based attribution studies. It has also been suggested that the unconditioned risk-based approach to event attribution may result in false negative results due to dynamical noise overwhelming any climate change signal. The “storyline” attribution framework, in which the impact of climate change on individual drivers of an extreme event is examined, aims to mitigate these concerns. Here we propose a methodology for attribution of extreme weather events using the operational European Centre for Medium-Range Weather Forecasts (ECMWF) medium-range forecast model that successfully predicted the event. The use of a successful forecast ensures not only that the model is able to accurately represent the event in question, but also that the analysis is unequivocally an attribution of this specific event, rather than a mixture of multiple different events that share some characteristic. Since this attribution methodology is conditioned on the component of the event that was predictable at forecast initialisation, we show how adjusting the lead time of the forecast can flexibly set the level of conditioning desired. This flexible adjustment of the conditioning allows us to synthesize between a storyline (highly conditioned) and a risk-based (relatively unconditioned) approach. We demonstrate this forecast-based methodology through a partial attribution of the direct radiative effect of increased CO$_2$ concentrations on the exceptional European winter heatwave of February 2019.

\section{Introduction}\label{ch3:intro}

  Attribution of extreme weather events is a relatively young field of research within climate science. However, it has expanded rapidly from its conceptual introduction \citep{allen_liability_2003} over the past twenty years; it now has an annual special issue in \emph{The Bulletin of the American Meteorological Society} \citep{peterson_explaining_2012}. Extreme event attribution is of particular importance for communicating the impacts of climate change to the public \citep{hulme_attributing_2014,hassol_natural_2016}, since the changing frequency of extreme weather events due to climate change is an impact that is physically experienced by society. As a result of this rapid expansion, there now exist numerous methodologies for carrying out an event attribution \citep{herring_explaining_2021}. Many of these rely on large ensembles of climate model simulations, the credibility of which has been questioned by recent studies \citep{bellprat_attribution_2016,bellprat_towards_2019,palmer_simple_2018}. A particular issue is the dynamical response of the atmosphere to external forcing, which is highly uncertain within these models \citep{shepherd_common_2016}. As attribution studies try to provide quicker results, with an operational system a clear aim, it is vital that any such system provides trustworthy results. In this study we propose a “forecast-based” attribution methodology using medium-range weather forecasts which could provide several key advantages over traditional climate model-based approaches. Firstly, if an event is predictable within a forecasting system, we know that that system is capable of accurately representing the event. Secondly, we know that any attribution performed is unequivocally an attribution of the specific event that occurred; unlike in unconditioned climate model simulations. Finally, weather forecasts are run routinely by many national and research centres. The models used are generally state-of-the-art and extensively verified. We propose that the attribution community could and should take advantage of the massive amount of resources that are put into these forecasts by developing methodologies that use the same type of simulation. Ideally, the experiments required for attribution with forecast models would be able to be run with little additional effort on top of the routine weather forecasts; in this way they might provide a rapid operational attribution system. We discuss these ideas further throughout the text.

  There have been several studies that propose or perform methodologies related to the forecast-based attribution demonstrated here. \citet{hoerling_anatomy_2013} used two seasonal forecast ensembles to examine the predictability of the 2011 Texas drought/heatwave within a comprehensive attribution analysis involving several types of climate simulation. \citet{meredith_crucial_2015} used a triply nested convection-permitting regional forecast model to investigate the role of historical sea surface temperature (SST) warming within an extreme precipitation event. They conditioned their analysis on the large-scale dynamics of the event through nudging in the outermost domain. More recently, \citet{van_garderen_methodology_2021} employed spectrally nudged simulations to assess the contribution of human influence on the climate over the 20th century on the 2003 European and 2010 Russian heatwaves. Possibly the most similar studies to the one presented here are a series of studies by Hope and colleagues \citep{hope_contributors_2015,hope_what_2016,hope_determining_2019}. They used a seasonal forecast model to assess anthropogenic CO$_2$ contributions to record-breaking heat and fire weather in Australia. Two more similar studies carried out forecast-based hurricane attribution studies \citep{reed_forecasted_2020,lackmann_hurricane_2015}. Tropical cyclones are a natural candidate for forecast-based methodologies due to the high model resolution required to represent them accurately, if at all. A final distinct, but related study is \citet{hannart_dada_2016}, which proposes the use of Data Assimilation for Detection and Attribution (DADA). They suggest that operational causal attribution statements could be made in a computationally efficient manner using the kind of data assimilation procedure carried out by weather centres (to initialise forecasts) to compute the likelihood of a particular weather event under different forcings (these would be observed and estimated pre-industrial forcings for conventional attribution). Our forecast-based framework differs from these other studies in several regards. Firstly, we use a state-of-the-art forecast model to perform the attribution analysis of the event in question; rather than to solely assess the predictability of the event. We use free-running coupled ocean-atmosphere global integrations here, allowing the predictable component at initialisation to dynamically condition the ensemble; as opposed to nudging our simulations towards the dynamics of the event, using nested regional simulations, or using the highly observationally constrained output of data assimilation procedures. A final key difference is that here we present an attribution of the direct radiative effect of CO$_2$ in isolation, though we hope that our approach could be extended in the future to provide an estimate of the full anthropogenic contribution to extreme weather events as in these other studies. We argue that the relative simplicity in the validation, setup and conditioning of our simulations is desirable from an operational attribution perspective; and flexible across many different types of extreme event. 

  We begin by introducing the chosen case study, the 2019 February heatwave in Europe, describing its synoptic characteristics and formally defining the event quantitatively. We then demonstrate the predictability of the event within the ECMWF ensemble prediction system, showing that this operational weather forecast was able to capture both the dynamic and thermodynamic features of the event. In \hyperref[ch3:experiments]{Perturbed CO$_2$ forecasts}, we outline the experiments we have performed in order to quantitatively determine the direct CO$_2$ contribution to the heatwave. We then provide quantitative results from these experiments, and finally conclude with a discussion of the strengths and potential issues of our forecast-based attribution methodology, including our proposed directions for further work.

\section{The 2019 February heatwave in Europe}\label{ch3:heatwave}

  Between the 21st and 27th February 2019, climatologically exceptional warm temperature anomalies of 10-15 \degree C were experienced throughout Northern and Western Europe \citep{young_record-breaking_2020}, as shown in Figure \ref{fig3.1}A. In particular, the 25th - 27th February saw record-breaking temperatures measured at many weather stations and over wide areas of Iberia, France, the British Isles, the Netherlands, Germany and Southern Sweden, as shown in Figure \ref{fig3.1}C \citep{cornes_ensemble_2018}. Figure \ref{fig3.1}D, comparing the regional mean maximum temperatures during the 2019 heatwave with timeseries of winter mean maximum temperatures between 1950 and 2018, illustrates just how unusual and widespread the event was. This heat was associated with a characteristic flow pattern: a narrow titled ridge extending from north-west Africa out to the southern tip of Scandinavia, advecting warm subtropical air north-east \citep{sousa_european_2018}, as shown in the geopotential height field in Figure \ref{fig3.1}A. This dynamical driver was accompanied by another synoptic feature that further enhanced the warming: widespread clear skies between the 25th -- 27th, shown in Figure \ref{fig3.1}B. These clear skies resulted in a widespread and persistent strong diurnal cycle, reaching 20 \degree C in some locations. Further details of the meteorological mechanisms and historical context of the heatwave are provided in \citet{young_record-breaking_2020,kendon_temperature_2020,christidis_extremely_2021}.
  
  To quantify the direct impact of CO$_2$ on the heatwave in question within this study, we need to characterise the heatwave in an `event definition'. The choice of event definition is subjective but can impact on the quantitative results of an attribution study significantly \citep{leach_anthropogenic_2020,uhe_comparison_2016,kirchmeieryoung_importance_2019}. The most remarkable feature of the February 2019 heatwave was the maximum temperatures observed, which peaked between the 25\textsuperscript{th} and 27\textsuperscript{th} for the majority of the affected area. Focusing on this relatively short time-period ensures that the synoptic situation driving the heat is coherent throughout the event definition window. For the spatial extent of the event, we use the eight European sub-areas described by \citet{christensen_summary_2007}. The use of regions previously defined in the literature aims to avoid selection bias. Our resulting event definition is as follows: the hottest temperature observed between 2019-02-25 and 2019-02-27, then averaged over the land points within each region (the temporal maximum is calculated before the spatial averaging). Although we carry out our calculations for all sub-areas, several regions were characteristically very similar in terms of both the event itself, and the forecasts of the event. We therefore focus on three of the eight regions: the British Isles (BI), which experienced exceptional heat and was well predicted; France (FR), which experienced exceptional heat but where the magnitude of the heat was less well forecast; and the Mediterranean (MD), which experienced well-predicted but climatologically average heat.

  % \clearpage
  \begin{figure}[h]
    \centering
    \includegraphics[width=\textwidth]{{Fig3.1}.png}
    \caption[The 2019 February heatwave in Europe: synoptic characteristics \& historical context.]{\textbf{The 2019 February heatwave in Europe: synoptic characteristics \& historical context. A}, maximum temperature anomaly in E-OBS with overlying contours of mean Z500 anomaly from ERA5 \citep{hersbach_era5_2020} over 25-27 February 2019. \textbf{B}, mean total cloud cover with overlying contours of mean sea level pressure anomaly averaged over 25-27 February 2019. \textbf{C}, rank of the maximum temperature in E-OBS over 25-27 February 2019 out of all winter temperature maxima since 1950 and light-blue scatterplot of 216 HadISD stations (with > 30 winters of measurements) which recorded their highest observed value over the same three days \citep{dunn_hadisd_2012,dunn_pairwise_2014,dunn_expanding_2016,smith_integrated_2011}. \textbf{D}, historical winter maximum regional mean daily maximum temperatures in E-OBS. Solid purple line shows timeseries of winter maxima for 1950-2018; dashed pink line indicates maximum value observed over the 25-27 February 2019. Regions are as \citet{christensen_summary_2007}.}
    \label{fig3.1}
  \end{figure}
  \clearpage

\section{Materials \& methods}\label{ch3:methods}

  \subsection*{The ECMWF Integrated Forecasting System}

    In this study, we use the Integrated Forecasting System (IFS) model cycle CY45R1, the operational cycle at the time of the event. The 51-member ensemble prediction system comprises a 91-layer, TCo639 resolution atmospheric model coupled to the 75-level, 0.25\textdegree{} resolution Nucleus for European Modelling of the Ocean v3.4 \citep{ecmwf_ifs_2018}. Once the model integration reaches the extended range (day 15 onward), the atmospheric model resolution is reduced to TCo319.


    \paragraph{The IFS model climatology}

      We define the IFS model climatology, used to compute model anomalies and the continuous ranked probability skill score, in an identical manner as is done operationally \citep[for example, to calculate the Extreme Forecast Index product][]{ecmwf_ecmwf_2018}. This climatology is defined using nine consecutive reforecast sets, spanning 5 weeks centred on the forecast initialisation date (reforecast sets are run twice a week, every Monday and Thursday), of 11 members per reforecast. These sets are created by initialising the reforecast ensemble on the same calendar date over the previous 20 y. This procedure results in a model climatology of 9$\times$11$\times$20=1980 members covering the 1999 to 2018 period. Throughout this article, we use the model climatology defined for the forecast initialised on 11 February 2019. Climatologies defined for other initialisation dates are virtually identical.

  \subsection*{Statistical methods}

    \paragraph{Significance testing}

      For the significance stippling displayed on the maps, we use a non-parametric (binomial) pairwise sign test at a 90\% confidence level.

    \paragraph{Distribution fitting}

      When fitting statistical distributions to the ensembles during the probability ratio calculation, we employ the method of L-moments \citep{hosking_l-moments_1990}, due to its numerical stability under small sample sizes.

  \section{Forecasts of the heatwave}\label{ch3:forecasts}

    This heatwave was well-predicted by the European Centre for Medium-Range Weather Forecasts (ECMWF) ensemble prediction system. Their coupled ocean-atmosphere forecasts indicated `extreme' heat was possible at a lead time of around two weeks, and probable at a lead time of around ten days (Figure \ref{fig3.2}A), despite the exceptional nature of the heatwave in both the model climatology and real world. As expected, the forecasts' performance in predicting the extreme heat at the surface is reflected in variables more closely linked to the dynamic drivers of the heat, such as 500 hPa geopotential height (Figure \ref{fig3.2}B). 
    
    This successful forecast is a crucial part of our study as it means that we are not only confident that the model used is able to simulate the event in question; but that we are unequivocally performing an attribution analysis of the specific winter heatwave that occurred in Europe during February 2019. This is an important distinction to the framework used in `conventional' or `risk-based' \citep{shepherd_common_2016} attribution studies \citep{stott_human_2004,pall_anthropogenic_2011,sparrow_attributing_2018,leach_anthropogenic_2020}, which in general reduce the event to some impact-relevant quantitative index, then estimate the increase in likelihood of events that exceed the magnitude of the event in question. For example, a heatwave attribution study may choose to define the event as the hottest observed temperature during the heatwave, and then compute the attributable change in likelihood of temperatures hotter than this recorded maximum (e.g. using models or historical records). While this does provide useful information, it does not answer the question of how much more likely anthropogenic activities have made the \emph{specific} heatwave that occurred, rather the question of how much more likely anthropogenic activities have made a mixture of events that share one or more characteristics. Studies have attempted to provide a more satisfactory answer to this first question by including a level of conditioning on the set of events considered by using circulation analogues \citep{yiou_statistical_2017}, or by nudging model simulations towards the specific dynamical situation that occurred during the event in question \citep{meredith_crucial_2015,van_garderen_methodology_2021}. Here we are evidently performing an attribution study of the specific record-breaking heatwave that occurred in February 2019 due to the use of these successful forecasts, that not only captured the heat experienced at the surface, but also the dynamical drivers behind the heat.

    As well as enabling us to answer the attribution question for a single specific heatwave, the use of a numerical weather prediction model provides additional benefits. Since large model ensembles are required to properly capture the statistics of extreme events, many previous attribution studies, especially in the context of heatwaves, have used relatively coarse, atmosphere-only climate models \citep{massey_weatherhome-development_2015,ciavarella_upgrade_2018,christidis_new_2013}, which may not fully capture all the physical processes required to credibly simulate the extreme in question \citep{sillmann_understanding_2017}. In particular, the use of atmosphere-only simulations may result in the full space of climate variability being under-sampled due to the lack of atmosphere-ocean interaction \citep{fischer_biased_2018}. This can lead to studies overestimating the impact of anthropogenic activity on weather extremes \citep{leach_anthropogenic_2020,bellprat_attribution_2016}. More generally, Bellprat et al., and Palmer and Weisheimer \citep{bellprat_towards_2019,palmer_simple_2018} have shown the importance of initial-value reliability in model ensembles underlying robust attribution statements. Model evaluation is therefore a key part of any robust model-based attribution study. Here, the demonstrably successful forecast enables us to be confident that the model used is providing credible realisations of the event.

    A clear distinction between the typical climate model simulations used for attribution \citep{massey_weatherhome-development_2015,christidis_new_2013} and the forecasts used here is that the climate model simulations are usually allowed to spin out for a sufficient length of time such that they have no memory of their initial conditions; an ensemble constructed in this way will therefore be representative of the climatology of the model. If such simulations use prescribed-SST boundary conditions, then the ensemble will be representative of the climatology conditioned on the prescribed SST pattern \citep{ciavarella_upgrade_2018}. Unlike climatological simulations, a successful forecast is conditioned upon the component of the weather that is predictable at initialisation. In general, the level of conditioning imposed upon the ensemble by the initial conditions reduces as the model integrates forwards from the initialisation date. Hence, a forecast ensemble initialised only a few days before an event will be much more heavily conditioned (and therefore much less spread) than one initialised weeks before. As the lead time increases, a forecast ensemble will tend towards the model climatology, analogous to the climate model simulations discussed above. We can relate these situations to the two broad attribution frameworks discussed in \citep{shepherd_common_2016}: very long lead times, where the forecast simulates model climatology, are analogous to `conventional' attribution; while short lead times, in which the forecast ensemble is heavily dependent on the initial conditions and therefore conditional on the actual dynamical drivers that lead to the extreme event, are analogous to the `storyline' approach in \citep{hazeleger_tales_2015,van_garderen_methodology_2021}. In order to synthesize between these two frameworks, here we have chosen 4 initialisation times (3-, 9-, 15-, and 22-day leads) for our experiments that span the range from a near-unconditioned climatological forecast to a short-term forecast that is tightly conditioned on the actual dynamical drivers of the heatwave.

    \clearpage
    \begin{figure}[h]
      \centering
      \includegraphics[width=\textwidth]{{Fig3.2}.png}
      \caption[Medium- to extended-range forecasts of the heatwave.]{\textbf{Medium- to extended-range forecasts of the heatwave. A}, ensemble distribution of heatwave as event definition against forecast initialisation date for the British Isles region. Gray PDF on far left shows model climatology, thick black lines show lead times selected for the perturbed CO2 experiments, dashed gold line shows heatwave magnitude in ERA5. Dots show ensemble mean. \textbf{B}, forecasts of Z500 over Europe during the heatwave period compared to ERA5. y-axis shows the fraction of the forecast ensemble with a pattern correlation at least as great as the levels indicated by the contour lines, against forecast initialisation date. Thin dotted lines show lead times selected for the perturbed CO2 experiments.}
      \label{fig3.2}
    \end{figure}
    \clearpage

\section{Perturbed CO$_2$ forecasts}\label{ch3:experiments}

  In this study we choose to only change one feature of the operational forecast in our experiments: the CO$_2$ concentration. This means that the analysis we carry out is limited to attributing the impact of diabatic heating due to increased CO$_2$ concentrations above pre-industrial levels just over the days between the model initialisation date and the event. Although this results in a counterfactual that does not correspond to any `real' world (since it is one with approximately present-day temperatures but pre-industrial CO$_2$ concentrations), and thus reduces the relevance of our analysis to stakeholders or policymakers; it does significantly increase the interpretability of our results, and remove a major source of uncertainty associated with a “complete” attribution to human influence: the estimation of the pre-industrial ocean and sea-ice state vector used to initialise the model \citep{stone_benchmark_2021}. Here we define a complete attribution as an estimate of the total impact of human influence on the climate arising from anthropogenic emissions of greenhouse gases and aerosols since the pre-industrial period. For each lead time chosen, in addition to the operational forecast (indicated by `ENS' in the figures) we run two experiments using operational initial conditions and identical to the operational forecast in every way except the experiments have specified fixed CO$_2$ concentrations. One experiment has CO$_2$ concentrations fixed at pre-industrial levels of 285 ppm (PI-CO$_2$), while in the other they are increased to 600 ppm (INCR-CO$_2$). These represent approximately equal and opposite perturbations on global radiative forcing \citep{etminan_radiative_2016}. We carry out these two experiments for each lead time, perturbing the CO$_2$ concentration in opposite directions, to ensure that any changes to the likelihood of the event can be confidently attributed to the changed CO$_2$ concentrations. It is possible that, due to the chaotic nature of the weather, the operational conditions were ideal for generating the observed extreme, and any perturbation to the dynamical system would reduce the likelihood of its occurrence \citep{shepherd_common_2016}. If this were the case we would see a reduction in event probability regardless of whether we increased or reduced the CO$_2$ concentration.
  
  Some previous work has been done on the impact of reduced CO$_2$ concentrations in the absence of changes to global sea surface temperatures. Baker et al. \citep{baker_higher_2018} explored how temperature and precipitation extremes were affected by the direct effect of CO$_2$ concentrations (defined there as all the effects of CO$_2$ on climate beside those occurring through ocean warming), finding the direct effect of CO$_2$ increases risk of temperature extremes, especially within the Northern Hemisphere summer. Our experimental design is also reminiscent of some of the earliest work done on investigating the impact of CO$_2$ on climate in global circulation models \citep{gates_preliminary_1981,mitchell_seasonal_1983}. This work found that, in the absence of changes to sea surface temperatures or sea ice concentrations, a doubling of CO$_2$ concentrations would change global mean surface temperatures over land by $\sim0.4$ \textdegree{}C. These early studies indicate that changes in global land temperatures are approximately linear with the logarithm of CO$_2$ concentration.
  
  We find that the best-estimate global mean change in land surface temperatures attributable to the additional diabatic heating due to CO$_2$ over pre-industrial levels (henceforth the `CO$_2$ signal', calculated as half the difference between the two experiments for a particular variable) at a lead time of two weeks (over the final 5 days of the forecasts initialised on 2019-02-11) is 0.22 [0.20--0.25] \degree C\footnote{Numbers in square brackets [] represent a 90\% confidence interval throughout this chapter.}. In general, the further away from the initialisation date, the slower the rate of change of the globally-averaged ensemble mean CO$_2$ signal, and the larger the ensemble spread (Figure \ref{fig3.3}A). While in experiments with prescribed SSTs, we might expect the CO$_2$ signal in surface temperatures to approach a maximum value within timescales on the order of months, in our experiments the CO$_2$ signal will likely continue to increase in magnitude for centuries due to the ocean-coupling, as is the case in the abrupt-4xCO$_2$ experiment carried out in the Coupled Model Intercomparison Project (CMIP) \citep{taylor_overview_2012,eyring_overview_2016,rugenstein_equilibrium_2020}. The zonal-mean patterns of surface temperature CO$_2$ signal are qualitatively similar to those exhibited by CMIP5 and CMIP6 models during the abrupt-4xCO$_2$ experiment \citep{flynn_climate_2020,andrews_dependence_2015}, despite the considerably shorter timescales involved: small and very confident changes in the tropics become larger but much less confident changes at the poles. This heterogeneity in the zonal distribution of warming appears to originate in the zonal distribution of the lapse-rate feedback; the weekly timescales of these experiments is insufficient for the surface-albedo feedbacks to have any significant impact \citep{smith_polar_2019}.
  
  We also examine the impact on the specific event dynamics over our region of interest; since these were crucial in developing the extremes observed. Figure \ref{fig3.3}B shows the growth in 500 hPa geopotential height (Z500) errors (measured as the mean absolute distance from ERA5 over the European domain) for each of the experiments. This figure illustrates that there are no clear differences in the ability of each experimental ensemble to predict the dynamical characteristics of the event. In other words, we have not made the synoptic event any more or less likely as a result of our perturbations. This is crucial as it means that we can consider any changes to the magnitude of the temperatures observed to be entirely due to the thermodynamic effect of changed diabatic CO$_2$ heating, and not due to the attractor of the dynamical system having changed as a result of the perturbations we have made.
  
  Figures \ref{fig3.3}C and \ref{fig3.3}D show analogous plots to \ref{fig3.3}B, but for inter-experimental and intra-ensemble errors respectively. These indicate a couple of important features. Firstly, no two experiments are more similar than any other two; the magnitude of Z500 distances in Figure \ref{fig3.3}C are near identical for all lead and validation times. Secondly, the error growth due to the CO$_2$ perturbation is slower than due to the initial condition perturbations; the errors in Figure \ref{fig3.3}C increase slower than in \ref{fig3.3}D. However, by the end of the longest lead forecast, we can see that the intra-ensemble errors have saturated, and the inter-experimental errors have grown to be the same magnitude. The saturation of intra-ensemble errors by the end of this lead time reinforces our assertion that at this lead the forecast is a good approximation of a climatological simulation.

  % \clearpage
  \begin{figure}[h]
    \centering
    \includegraphics[width=\textwidth]{{Fig3.3}.png}
    \caption[Global temperature and synoptic-scale dynamical response to CO$_2$ perturbations.]{\textbf{Global temperature and synoptic-scale dynamical response to CO$_2$ perturbations. A}, CO$_2$ signal in global mean surface temperatures. Brown and blue features show quantities over land and ocean respectively. Line styles indicate initialisation date of the experiments. Boxplots show average over 25 to 27 February 2019, with the black line indicating the ensemble mean, dark shading the 90\% confidence interval around the mean, and light shading the 90\% range of the ensemble. \textbf{B}, mean absolute error in Z500 between experiments and ERA5. Colour indicates initialisation date and line style indicates experiment. Solid lines indicates ensemble mean. The shading shows the 5 to 95\% range of the operational ensemble (ENS). \textbf{C}, as in \textbf{B} but for mean absolute distance between corresponding ensemble members of different experiments. Line style here indicates the experiments being differenced. The shading shows the 5 to 95\% range of the differences between the PI-CO$_2$ and INCR-CO$_2$ experiments. \textbf{D}, as in \textbf{B} but for intraensemble distances of the operational ensemble.}
    \label{fig3.3}
  \end{figure}
  \clearpage

\section{Attributing the heatwave to diabatic CO$_2$ heating}\label{ch3:attribution}

  First, we examine the geographical pattern of the CO$_2$ signal in the heatwave in Figures \ref{fig3.4}A-D. These indicate several key features of the attributable direct CO$_2$ effect on the heatwave. The CO$_2$ effect tends to grow with lead time, consistent with its impact on global mean temperatures. It is generally stronger over land than ocean, also consistent with global mean temperatures. Finally, the ensemble tends to become less confident in its effect as the lead time increases and the ensemble members diverge. The CO$_2$ signal magnitude in the heatwave generally exceeds the signal in the global mean surface temperature (Figure \ref{fig3.3}A), in particular in Central Europe; possibly due to the high contribution of diabatic heating to the heatwave arising from ideal dynamical conditions. Figure \ref{fig3.4}E shows boxplots of the heatwave CO$_2$ signal for the three regions of interest. Although there is some region-specific variability, these reinforce the main messages illustrated by the maps: the CO$_2$ signal grows and decreases in confidence as the lead time increases.

  In addition to the absolute impact of the direct CO$_2$ effect on the heatwave, we also carry out a probabilistic assessment of its impact, consistent with conventional `risk-based' attribution studies \citep{shepherd_common_2016,winsberg_severe_2020}. Due to the distinct approach we are taking within this study, it is worth clarifying exactly what question we are answering with this probabilistic analysis. The specific question is: `given the forecast initial conditions, how did the direct impact of increased CO$_2$ concentrations compared to pre-industrial levels just over the days between initialisation and the heatwave itself change the probability of temperatures at least as hot as were observed?'. Using conventional attribution terminology, we call the operational forecast ensemble of the event as our `factual' ensemble, and the pre-industrial CO$_2$ experiment as our `counter-factual' ensemble. We calculate the probability of simulating an event at least as extreme as observed in the factual ensemble, $P_1$, and in the counterfactual ensemble, $P_0$. These probabilities are estimated by fitting a generalised extreme value distribution to the 51-member ensemble in each case. We then express the change in event probability as a probability ratio, $PR=P_1/P_0$, which represents the fractional increase in the likelihood of an event at least as extreme as observed in the factual ensemble over the counterfactual ensemble \citep{stott_human_2004,stone_end--end_2005}. Uncertainties are estimated with a 100,000 member bootstrap with replacement, rejecting samples for which the probability of the event in the factual ensemble is zero. The resulting probability ratios are shown in Figure \ref{fig3.4}F. There are several key factors that contribute to the best-estimate and confidence in the probability ratios: the CO$_2$ signal growth with lead time; the ensemble spread growth with lead time; how extreme the event was; and how well-forecast the event was. The larger the CO$_2$ signal, the greater the increase in risk; the larger the ensemble spread, the lesser the increase in risk and the lower the confidence; the more extreme the event, the greater the increase in risk; and the better the forecast (i.e. the closer the event to the ensemble centre), the greater the confidence. 
  
  We find that on the shortest lead time, the direct CO$_2$ effect increases the probability of the event over all European regions (significant at the 5\% level based on a one-sided test). For the well-forecast event experienced over the British Isles, the direct CO$_2$ effect increases the probability of the extreme heat by 42 [30--60]\%. For the France heatwave, which was well-forecast given its exceptional nature, but for which the ensemble did not quite reach the total magnitude of the heat experienced, the event probability increased by at least 100\% (5\textsuperscript{th} percentile), but with a very wide uncertainty range. Finally, for the least remarkable but relatively well-forecast event over the Mediterranean, the direct impact of CO$_2$ increased the event probability by 6.7 [4.6--9.7]\%. These results from the very short lead experiments represent very highly conditioned statements: in both ensembles the dynamical evolution of the event was near-identical (pattern correlation of > 0.99 for all ensemble members, Figure \ref{fig3.2}B).
  
  Moving out to the longer lead times, we find that the confidence in the change in event probability decreases almost ubiquitously. This is as expected, since the further we move away from the event, the less highly conditioned our ensemble is, and the more dynamical noise we are adding to the system \citep{shepherd_common_2016}. However, for the 9-day lead forecast, the uncertainty is low enough to have confidence in the results for the majority of study regions. In particular, the British Isles heatwave, for which the 9-day lead forecast was better than several of the regional 3-day lead forecasts (as measured by the Continuous Ranked Probability Skill Score), increases in probability by 52 [29--94]\% due to the direct CO$_2$ effect. However, for France the uncertainty range is so large that based on these results alone we would have no confidence in the direction of the CO$_2$ effect. Moving further out to the 15- and 22-day lead forecasts, this loss in confidence becomes more pronounced, especially for the British Isles region. For this region, we can get virtually no useful information out of these probabilistic results for the two longest lead experiments. This drop-off in confidence arises due to the increasing ensemble spread from dynamical noise, and large reduction in the number of factual ensemble members able to simulate an event as hot as occurred in reality between the 9- and 15-day leads. A similar, though generally less pronounced drop-off in confidence is found in all other regions. 
  
  We can make use of our INCR-CO$_2$ experiment to increase our confidence that the positive results we obtained in the probabilistic analysis above are in fact due to the direct CO$_2$ effect, and not just random variability. If CO$_2$ were driving the changes in event probability between the PI-CO$_2$ and operational forecasts, then we would expect to see an even more dramatic increase in event probability between the PI-CO$_2$ and INCR-CO$_2$ forecasts. This is indeed what we find. For all regions and lead times, our best-estimate change in event probability is above zero when CO$_2$ concentration is increased from pre-industrial levels of 285 ppm to 600 ppm. This therefore increases our confidence further that the positive attribution to CO$_2$ under high conditioning is genuinely significant. From these results, it also appears that there is a general trend of change in event probability increasing as the forecast lead increases, similar to the absolute impact of the direct CO$_2$ effect trend; though it is still somewhat masked by uncertainty.
  
  An important caveat on all of these results, probabilistic and absolute, is that they represent a lower bound on the estimate of the direct CO$_2$ effect. As is clear from the development of the CO$_2$ signal estimates with lead time, the model is still adjusting to the sudden change in CO$_2$ concentration (and would continue to do so for centuries due to the very long deep ocean equilibration timescales). Hence, we would expect the `full' effect of CO$_2$ to be greater than the estimates we present here. This is consistent with a recent study that used unconditioned climate model simulations to carry out an attribution of the complete anthropogenic contribution to the same event, which produced much higher estimates of the probability ratio \citep{christidis_extremely_2021}.

  \clearpage
  \begin{figure}[h]
    \centering
    \includegraphics[width=\textwidth]{{Fig3.4}.png}
    \caption[Attribution of the direct CO$_2$ influence on the heatwave.]{\textbf{Attribution of the direct CO$_2$ influence on the heatwave. A-D}, maps of the ensemble mean attributable CO2 signal in the heatwave for the four forecast lead times, which are indicated by the subplot titles. Stippling indicates a significant positive signal at the 90\% level. \textbf{E}, boxplot of the absolute CO$_2$ signal for the three regions of interest and over the four forecast lead dates. Black line indicates ensemble median. Dark shading indicates 90\% confidence in the median, and light shading indicates 90\% confidence in the ensemble. Gray line indicates median difference between the operational forecast and PI-CO$_2$ experiment. \textbf{F}, as in E, but showing probability ratios using the operational forecast as a factual and PI-CO$_2$ experiment as a counterfactual ensemble. \textbf{G}, as in F, but using the INCR-CO$_2$ experiment as a factual and PI-CO$_2$ experiment as a counterfactual ensemble.}
    \label{fig3.4}
  \end{figure}
  \clearpage

\section{Discussion}\label{ch3:discussion}

  Here we have presented a partial, forecast-based attribution of the European 2019 winter heatwave. Taking advantage of successful medium-range forecasts from ECMWF, we used a state-of-the-art numerical weather prediction model that was demonstrably able to predict the event to attribute the direct impact of CO$_2$ through diabatic heating over pre-industrial levels and just over the days immediately preceding the event on the high temperatures experienced in several regions of Europe. We explored how the level of dynamical conditioning imposed can be specified by changing the lead time of the forecasts. Finally, we presented our quantitative results using two different approaches: measuring the attributable absolute and probabilistic impacts of CO$_2$; inspired by the `storyline' and `risk-based' attribution frameworks \citep{stott_human_2004,shepherd_common_2016,winsberg_severe_2020,jezequel_behind_2018}.  
  
  There are several advantages associated with this forecast-based attribution methodology, compared to conventional climate model based attribution. One simple advantage is that forecast models generally represent the technological peak within the spectrum of General Circulation Models. They almost always have a higher resolution than the models used for climate simulation. In addition, the forecast model used here is coupled ocean-atmosphere, while the large climate model ensembles used for attribution often use prescribed sea surface temperatures \citep{ciavarella_upgrade_2018}. The use of prescribed SSTs can lead to model biases that project strongly onto attribution results \citep{fischer_biased_2018}. A final advantage arising from the use of an operational forecast model is the wealth of literature and model analysis that will already be available before an attribution study is initiated. As well as these advantages associated with the type of model there is the crucial advantage associated with using successful forecasts: the specific and intrinsic model verification. Due to the difficulty in fully quantifying how well climate models can represent an individual specific event (in particular, the very large ensembles required to have a large enough sample of characteristically similar events), climate model based attribution studies tend to perform statistical model evaluations; or/and account for this uncertainty through multi-model ensembles \citep{philip_protocol_2020}. On the other hand, if a forecast model that demonstrably predicted the event as it occurred is used, no further model verification or evaluation is required to test whether the model is capable of producing a faithful representation of the specific event.
  
  Related to this intrinsic verification is an important point on the framing of forecast-based attribution studies. Climate model based attribution studies tend to characterise an event in terms of some quantitative index closely related to the impact of the event (such as the maximum temperature observed during a heatwave). They then use climate model simulations to determine how climate change has affected the probability of observing an event at least as extreme as the actual event. This is often done without imposing any dynamical conditioning on the simulations, though this is an area of active research \citep{yiou_statistical_2017,pall_diagnosing_2017}. This unconditional approach means that the specific question being answered is not `how has anthropogenic climate change affected the probability of event X?', but `how has anthropogenic climate change affected the probability of all events that are at least as extreme as event X in terms of the index used to define X?'.  The latter question does not fully answer the question of how climate change has affected the actual event that the study is concerned with. In contrast, the use of a forecast model that predicted the event ensures that any attribution analysis is unequivocally an attribution of that specific event \citep{hope_determining_2019}.
  
  In addition to its advantages, this forecast-based attribution methodology also has associated issues that must be overcome. Firstly, the forecast model must have produced a `good' forecast of the event. If the model is unable to represent the event as it happened, then we cannot have confidence in any estimates of the impact of climate change on that event. Issues can arise even in qualitatively `good' forecasts, such as the forecast of the heatwave over France in this study. As very few ensemble members, if any, exceeded the observed magnitude of the event for this region, the confidence in our estimates of the probabilistic impact of CO$_2$ on the event is extremely low (since we are extrapolating the distribution shape outside the range of our data). Although the estimates of the absolute impact of CO$_2$ do not share this lack of confidence, this is still a problem. It is possible that applying some bias correction procedure \citep[e.g.][]{sippel_novel_2016,jeon_quantile-based_2016,li_reducing_2019} based on the model climatology to the model output before analysis might alleviate these issues to some extent, but not if the model is simply unable to predict the event in question (i.e. a forecast bust). Secondly, the short timescales involved in these medium-range forecasts mean that the interpretation of any results becomes more difficult as the model is still adjusting to the perturbations imposed \citep{hope_contributors_2015}, at least in the case of the CO$_2$ perturbations applied here. This adjustment is clear on a global scale in Figure \ref{fig3.3}A. Due to this incomplete adjustment, any quantitative statements of attribution represent a lower bound on the `true' value. 
  
  We have shown that the direct effect of CO$_2$ concentrations over pre-industrial levels on the February heatwave is significant, even on timescales as short as a few days. Based on the very good 9-day lead forecast of the heatwave over the British Isles, the region that saw the most climatologically exceptional event, the direct effect of CO$_2$ was to increase the magnitude of the heatwave by 0.31 [0.24--0.37] K, and the conditional probability of the heatwave by 52 [29--94]\%. It is very important to bear in mind that this statement of risk is highly dynamically conditioned (Figure \ref{fig3.2}B). These estimates of the impact of CO$_2$ on the heatwave follow the storyline attribution framework, since we have effectively removed the dynamical uncertainty from our simulations with this strong conditioning imposed by the short lead time \citep{shepherd_common_2016,shepherd_storylines_2018,jezequel_behind_2018}. Our longer, 22-day lead experiments can contrast this storyline analysis with relatively unconditioned results much closer to the climatological simulations typically used in the conventional `Risk-based' attribution framework \citep{philip_protocol_2020,stott_human_2004}. At this lead, we find that although over all regions the best-estimate impact of the direct CO$_2$ effect is to enhance the heatwave by approximately 0.5 K, in none of the regions is this impact significantly positive at the 90\% level (based on the bootstrapped confidence in the median value). Corresponding estimates of the probability ratio have so low confidence that they provide virtually no useful information. Increasing the forecast ensemble size, which is small compared to the climate model ensembles used in most attribution studies, would increase the confidence, potentially resulting in useful quantitative estimates of the probability ratio even at these longer lead times. Our results illustrate some of the concerns voiced recently over the conventional risk-based approach to attribution \citep{winsberg_severe_2020,shepherd_common_2016}. Due to the dynamical noise present in unconditioned ensembles, it is possible to obtain an inconclusive attribution within a conventional risk-based framework, and at the same time obtain a confident positive attribution if the dynamical uncertainty is removed through conditioning (in our case achieved by reducing the forecast lead).
  
  While this study provides a demonstration of the potential use for forecast models within attribution science, it remains a partial attribution to the direct CO$_2$ effect only. For forecast-based attribution to provide results that are fully comparable to conventional climate model-based attribution, we will need to demonstrate how the complete anthropogenic contribution to an extreme event could be estimated with successful forecasts. The next step to progress forecast-based attribution further will be to remove an estimate of the anthropogenic contribution to ocean temperatures from the model initial conditions \citep[e.g.][]{stone_benchmark_2021}. If performed in addition to reducing other greenhouse gas concentrations and aerosol climatology down to their pre-industrial levels, this should allow us to run pre-industrial forecasts of an event. This has been done previously for a seasonal forecast model by Hope et al. \citep{hope_contributors_2015,hope_what_2016,hope_determining_2019}. They removed the anthropogenic signal from 1960 onwards from the initial conditions, but we could in principle remove the signal from pre-industrial times onwards in order to estimate the complete anthropogenic contribution to an event. Although it is highly likely that there will be methodology specific issues that arise in this direction, we suggest that being able to estimate the complete anthropogenic contribution to an extreme event using a forecast model that was able to predict the event in question would be extremely valuable. Developing a methodology to allow us to do so might also provide a pathway to operational attribution being able to be carried out by weather prediction centres, due to the routine frequency at which they produce forecasts. In addition to attempting a `complete' forecast-based attribution of an extreme event, we would like to explore how increasing the ensemble size may allow us to provide confident forecast-based attribution analyses within the unconditioned risk-based framework (i.e. at long forecast lead times). One potential avenue to allow us to do this efficiently might be to reduce the resolution of the forecasts, though this would not be appropriate if it reduced the ability of the model to represent the event in question. On a similar note, we would also like to extend our experiments out to seasonal timescales. This would reduce the issues with the interpretation of our medium-range results that occurred due to the model adjustment to the sudden changes to the CO$_2$ concentration. It is possible that seasonal forecasts have the greatest potential to target for an operational forecast-based attribution methodology.

\section{Chapter close}\label{ch3:close}

  There are a number of interesting outcomes from this chapter. The most important one was that this approach to attribution, using reinitialised state-of-the-art operational weather forecasts, was possible --- even if here I used it to answer a limited and specific question. Another surprising finding was the rapidity on which the direct impact of CO$_2$ could be detected: my experiments suggest that if CO$_2$ were suddenly reduced to pre-industrial levels, it would detectably reduce the intensity of extreme heat events on timescales of days --- i.e. well before the SSTs have had any time to respond. This finding translated into a significant increase in the likelihood of the exceptional temperatures observed over the UK in February 2019 attributable to increased CO$_2$ concentrations above pre-industrial levels alone. The final key takeaway from this study was that the predictability of the heatwave remained remarkably intact despite the hefty `kick' we had given the model at initialisation in the perturbed CO$_2$ forecasts. This was vital for the reason described in more detail within \hyperref[ch3:experiments]{Perturbed CO$_2$ forecasts}: it means we could ascribe our results to the impact on CO$_2$ on the heatwave, rather than to the chaotic nature of the weather system. If we had found that the predictability was less stable, we would have had to investigate further why this was the case, potentially returning to the drawing board for this approach. However, despite the promise that forecast-based approaches demonstrated within this study, the attribution I carried out was very limited in scope: to a single component of anthropogenic influence on the climate. In order to make results from the approach truly relevant to and of use for stakeholders I needed to determine how the methodology could be altered to allow more complete estimates of human influence on individual weather events to be made. This attempt at more complete attribution, through perturbing both the forecast boundary \emph{and} initial conditions, is the focus of the next chapter of this thesis.

\begin{savequote}[8cm]
    Quote
      \qauthor{--- author}
\end{savequote}
    
\chapter{\label{ch4}Forecast-based attribution: perturbing the initial and boundary conditions}

Chapter description.
\small\paragraph{Author contributions:} This chapter is based on the the following publication \footnote{with the author contributing as follows.} \par\vspace{1em}
\formatchref{Leach, N. J., Roberts, C. D., Heathcote, D., Mitchell, D. M., Thompson, V., Palmer, T. N., Weisheimer, A., \& Allen, M. R.}{2022}{Reliable heatwave attribution based on successful operational weather forecasts}{Nature}{submitted}{}{}{}

\clearpage

\minitoc

\clearpage

\section{Chapter open}

  % I think this could be better
  Following on from the previous chapter, the primary aim was to try and use a forecast-based approach to provide a more complete estimate of human influence on an individual extreme. Based on previous work \citep{pall_anthropogenic_2011}, the key component we needed to include was anthropogenic influence on the ocean --- in addition to the changes to atmospheric CO$_2$ concentration. Although there is a considerable body of literature concerning how to perturb sea surface temperatures in climate models to include this effect \citep{stone_benchmark_2021}, the model we used previously, and continued to use, is coupled, and therefore we needed a way to perturb not just the SSTs, but also the ocean subsurface. This technical challenged posed by the coupling also presented an opportunity, as it means that the resulting counterfactual forecasts are consistent with observed changes in ocean heat content (the primary energy sink for thermal energy accumulated in the earth system as a result of global warming), and do not contain any infinite sources or sinks of heat, as is the case in the prescribed-SST simulations used widely in attribution. We considered a number of different methods for imposing the ocean perturbations, including a nudged ocean spin-up simulation, but settled on arguably the simplest approach, which has been used previously by Chris Roberts at ECMWF for testing the impact of different ocean resolutions. At the time when we were planning these simulations, the 2021 Pacific Northwest Heatwave had just occurred, presenting a very natural case study for us to apply the forecast-based approach to, as conventional approaches had clearly been pushed to their limit with this event. The overall aim of our study was to demonstrate that we could provide a near-complete estimate of the human influence on this unprecedented event using a weather forecast model that was demonstrably able to simulate it.

\section{Abstract}

  Extreme weather attribution, quantifying the role of human influence in specific weather events, is of interest to scientists, adaptation planners and the general public \cite{national_academies_of_sciences_engineering_and_medicine_attribution_2016}. However, the devastating 2021 Pacific Northwest heatwave challenged conventional statistical approaches to attribution due to the absence of similar events in the historical record, and model-based approaches due to poor representation of key causal processes in current climate models \cite{white_unprecedented_2022}. Here we use state-of-the-art operational medium-range and seasonal weather prediction systems, applied for the first time to this kind of climate question and unequivocally able to simulate the detailed physics of the heatwave in question, to show that human influence on the climate made this event at least 8 [2--30] times more likely to occur. Quantifying the absolute probability of such an unprecedented event is more challenging, but the length of the observational record suggests at least a multi-decade return-time in the current climate, with the likelihood doubling every 17 [10--50] years at the current rate of global warming. Our forecast-based approach synthesises the storyline approach, which examines human influence on the physical drivers of an event in a deterministic manner \cite{shepherd_common_2016}, and the probabilistic approach, which assesses how the frequency of a class of events has been affected by human influence \cite{stott_human_2004}. If developed as a routine service in a number of forecasting centres, it could provide reliable estimates of the changing probabilities of all extreme events that can be represented in forecast models, which is critical to supporting effective adaptation planning \cite{harrington_integrating_2022,mitchell_climate_2021}.

\section{Introduction}

  Although considerable progress has been made over the past decade in quantifying the impact of climate change on individual extreme weather events \cite{stott_human_2004,national_academies_of_sciences_engineering_and_medicine_attribution_2016}, challenges remain over the assessment of the most extreme events. Such events are particularly difficult to draw confident conclusions about due to the lack of historical analogues, and their often poor representation in the climate models normally used for event attribution. Two contrasting mainstream frameworks to event attribution have been developed: the storyline approach, which examines anthropogenic influence on the causal drivers of the extreme in question and is therefore highly conditioned on its specific characteristics \cite{trenberth_attribution_2015,shepherd_common_2016,hoerling_anatomy_2013}; and the probabilistic approach, which aims to determine how anthropogenic influence has affected the likelihood of events at least as extreme as the one in question \cite{philip_protocol_2020,pall_anthropogenic_2011}. 

  A key challenge for extreme event attribution is that we cannot make direct observations of a world without human influence on the climate, so all approaches must involve some kind of modelling, either statistical \cite{van_oldenborgh_how_2007} or dynamical \cite{pall_anthropogenic_2011}. Both face difficulties with the most extreme events, especially when considering the nonlinear processes that often drive unprecedented events. Statistical models used in conventional attribution can break down when faced with such events due to the lack of appropriately similar historical samples \cite{gessner_very_2021}, while numerical climate models generally used in attribution studies are typically coarse (O(100km) horizontal resolution), and poorly represent important processes involved in the development of extreme weather events, such as blocking \cite{masato_winter_2013} and atmospheric rivers \cite{payne_evaluation_2015}. Even with a “perfect” model of the earth system, the unconditioned nature of the vast majority of climate model simulations used in attribution means that obtaining enough analogues of unprecedented events \cite{fischer_increasing_2021} to avoid the same issue faced by statistical modelling of the observational record requires very large ensembles, possibly beyond current computational limits \cite{leach_generating_2022}. Crucially, the role of climate change in an individual event may differ from that in other events of the same class due to the specific physical processes behind it \cite{palmer_nonlinear_1999,palmer_simple_2018}.

  The storyline framework overcomes some of these issues, and the risk of a false negative, by examining the impact of climate change on the causal drivers of an event deterministically. For instance, one might separate out the thermodynamic (typically high confidence in response to climate change and well-represented in numerical models) and dynamic (typically much lower confidence in response to climate change, and more poorly represented in numerical models) drivers of an event, for example by conditioning on the concurrent large scale atmospheric circulation. One approach for applying such conditioning is to “nudge” climate model simulations towards the large-scale flow observed during a particular extreme \cite{van_garderen_methodology_2021,benitez_july_2022}. The storyline approach does not, however, provide quantitative information about how climate change has affected the probability of the event in question, which is of interest to the general public and relevant to policymakers for adaptation planning.

  We propose a forecast-based approach that could synthesise the probabilistic and storyline frameworks to extreme event attribution \cite{leach_forecast-based_2021}. Although they belong to the same class of dynamical model and often share components \cite{roberts_climate_2018}, operational weather forecast models are typically run at much higher resolutions than climate models, improving their overall physical representation of extremes. They are validated for producing predictions that span the range of possible weather by the centres that produce them to a much higher degree than climate models, where other aspects are more important. In addition to this high level of explicit validation, using a model that has successfully predicted an event ensures that the model is able to accurately represent all the processes involved in the event in question, increasing the reliability of attribution statements based upon it \cite{palmer_simple_2018}. Stepping back through lead times allows for a robust storyline-like framing by examining how climate change has affected the causal drivers of the specific event within the limits of predictability. Probabilistic attribution can be performed using a reliable forecast ensemble, with the level of conditioning set by the lead time - the limiting case of long lead times is equivalent to a conventional unconditioned analysis. There has been some previous work into forecast-based attribution, using seasonal forecast models \cite{hope_contributors_2015,hope_what_2016,hope_determining_2019,wang_initialized_2021,hope_subseasonal_2022} and exploring the conceptual framework \cite{pall_diagnosing_2017,wehner_estimating_2019,tradowsky_toward_2022}. To our knowledge, however, this study is the first time that a complete forecast-based attribution has been carried out in a coupled operational forecast model at such a high resolution.

  In this study we use the coupled operational ECMWF model (details in the Supplement) to analyse the Pacific Northwest heatwave, taking advantage of its successful predictions of this unprecedented event at leads of over a week. We perform counterfactual forecasts of the event by perturbing the initial and boundary conditions of the model in order to simulate how the heatwave might have emerged had it occurred in a cooler pre-industrial world, or a warmer future world. We then compare the counterfactual and operational forecasts to assess the impact of anthropogenic climate change on both the magnitude and probability-of-occurrence of the event. We believe that this forecast-based approach opens the door to not only a reliable and practical operational attribution system, but also to a robust way of generating projections of future weather explicitly referenced to the forecasts used already by adaptation planners \cite{hazeleger_tales_2015}.

  \subsection{The Pacific Northwest heatwave}

    At the end of June 2021, a large fraction of the Pacific Northwest region of the US and Canada experienced unprecedented high temperatures, including the cities of Portland, Salem, Seattle and Vancouver (Figure 1). This heatwave (the “PNW heatwave”) has been directly linked to many hundred excess deaths during and following it, making it the deadliest weather event on record for both Canada and Washington state \cite{henderson_analysis_2022}. The heatwave peak was observed between the 28th \& 30th June, though temperatures were still exceptionally high on the days immediately before and after this period \cite{menne_global_2012,menne_overview_2012}. A large number of local maximum temperature records were broken during this period, including the Canadian all-time record by a margin of 4.6 °C.

    Based on current understanding, the heatwave arose from an optimal combination of proximal drivers \cite{overland_causes_2021,lin_2021_2022,mo_anomalous_2022,white_unprecedented_2022}. Development of an omega block between the 23rd-27th coincided with the landfall of an atmospheric river (AR) on the 25th. Warm air was drawn up from the tropical West Pacific, heated diabatically through condensation in the river and then further heated adiabatically through subsidence: both the temperature and lapse rate at 500 hPa reached or approached record levels in the regions affected. This atmospheric heating was enhanced by soil moisture feedbacks \cite{thompson_2021_2022} and high insolation at the land surface during the hottest hours of the day (Figure 2). Given the unprecedented nature of the observed heatwave, any dynamical numerical model would need to capture all these processes, including the coupling between them, in order to produce an accurate representation of the event.
    
    Despite the observed temperatures lying far outside the historical record, the heatwave was well predicted by numerical weather forecast models such as from ECMWF at lead times of more than a week. The seasonal forecast from ECMWF captured one important aspect of the event: it predicted a thicker troposphere than average (measured by 500 hPa geopotential height) over the Pacific Northwest during the summer. A key change in the predictability of the exceptional temperatures occurred around June 21st, being the earliest point at which the penetration of the AR over land was well represented \cite{mo_anomalous_2022}. The success of these forecast models provides an opportunity to use them to examine the influence of anthropogenic climate change on the event as it actually occurred.

    \clearpage
    \begin{figure}[h]
      \centering
      \includegraphics[width=0.8\textwidth]{{Fig4.1}.png}
      \caption[Features and forecasts of the Pacific Northwest heatwave.]{\textbf{Features and forecasts of the Pacific Northwest heatwave. Top panel:} Surface temperature anomalies at the time of the peak heat during the heatwave within the region enclosed by 45-52N, 119-123W (indicated by the dotted rectangle). Solid black contours show the 500 hPa geopotential height anomaly averaged over 26-30th June 2021. Data are from ERA5 reanalysis \cite{hersbach_era5_2020}. \textbf{Inset:} timeseries of annual maximum temperatures for the same dotted region. \textbf{Bottom panels:} As above, but taken from the ensemble member within the forecast initialised on the date given above each panel that predicted the nearest temperature to the reanalysis within the dotted region.}
    \end{figure}
    \clearpage

\section{Forecast-based attribution}

  The date at which we initialise our perturbed forecasts is a key choice that allows us to condition our attribution analysis on different synoptic drivers of the heatwave, which become predictable at different leads \cite{lin_2021_2022,mo_anomalous_2022}. The climate change response of drivers already present in the initial conditions is clearly not incorporated into our attribution results for each lead time due to this conditioning. Starting with the operational configurations of the ECMWF forecast model, we chose to focus on three medium-range and one seasonal forecast lead: 3 days, 7 days, 11 days and 2-4 months. These leads highlight the following aspects of the attribution: 

  \begin{itemize}
    \item 3 days (2021-06-26): a forecast very highly conditioned on the synoptic drivers of the event, with several key drivers prescribed in the initial conditions, and the rest forecast near perfectly. At this lead, our experiments could be considered analogous to a storyline attribution framing.
    \item 7 days (2021-06-22): a highly conditioned forecast, with most simulated processes mirroring reality closely. However, the shape and gradient reversal magnitude of the block shows considerable variation in this ensemble.
    \item 11 days (2021-06-18, depicted in Figure 2): while the exceptional thickness of the tropospheric block was well predicted in a large proportion of the ensemble, the shape and associated gradient reversal was only captured in a few members. The occurrence of the AR was well predicted, but its location and penetration over land less so, with most members predicting a more southerly landfall. The low soil moisture and cloud cover was well captured by the majority of the ensemble.
    \item 2-4 months (2021-05-01): a considerably less conditioned forecast. For this lead, we take the peak heat event over the whole summer period since we do not expect the forecast to predict the timing of the heatwave. Although the forecasts were unusually successful at predicting elevated geopotential height and temperatures over the summer in general, none of the peak heat events within individual ensemble members capture all of the detailed features of the PNW heatwave. At this lead, the ensemble can be viewed as being near-analogous to a high resolution unconditioned climate model simulation (though one that we know is able to represent the processes involved in the PNW heatwave accurately).
  \end{itemize}

  We then perturb the boundary and initial conditions of the operational forecast (technical details in the Supplement). First, we perturb the CO$_2$ concentrations in the atmosphere back to pre-industrial levels of 285 ppm, similar to \cite{leach_forecast-based_2021}. Then we remove a balanced estimate of anthropogenic change between pre-industrial and the present-day in surface and sub-surface ocean temperatures, sea ice concentration, and sea ice thickness \cite{locarnini_world_2019,rayner_global_2003,zuo_ecmwf_2019} from the initial state of the model. Perturbing the temperatures over the entire ocean depth means that we produce forecasts that are thermodynamically consistent with the changes in upper ocean heat content, in contrast to prescribed SST approaches \cite{massey_weatherhome-development_2015,ciavarella_upgrade_2018}. We do not alter the land-surface, noting the high uncertainties in past trends for indicators such as soil moisture in this region \cite{masson-delmotte_water_2021,masson-delmotte_changing_2021,masson-delmotte_atlas_2021}. Removing anthropogenic influence from the ocean state and reducing CO$_2$ levels produces a counterfactual “pre-industrial” forecast; we also apply identical perturbations in the opposite direction to produce a “future” forecast, in which the ocean state and CO$_2$ levels of 615 ppm correspond to approximately twice the level of global warming experienced at the present-day.

  We find that despite the large impulse applied by the perturbed initial state upon forecast initialisation, the predictability of the heatwave is remarkably stable. The key synoptic drivers of the heatwave present in the original operational forecast remain intact. There are some changes consistent with the canonical response to global warming, including a thickening of the lower troposphere \cite{christidis_changes_2015} and increased tropospheric water vapour \cite{allen_constraints_2002} in the future forecast; and vice-versa in the pre-industrial forecast. As such, the perturbations have not altered the forecasts in such a way that they produce “different” weather, and we can compare our forecasts to estimate the influence of anthropogenic global warming on the Pacific Northwest heatwave. This is consistent with \cite{leach_forecast-based_2021}, but is not guaranteed to be the case for every weather event.

  This experiment design is consistent with the perturbed CO$_2$ experiments of \cite{leach_forecast-based_2021} in another important respect: the adjustment to the new “pre-industrial” or “future” climate state occurs continually throughout the forecast. This adjustment typically means that as the lead time increases, the estimated attributable influence on the heatwave also increases. Interplay between dynamical noise and attributable signal in the forecasts, both of which increase with the lead time (short leads correspond to more confident but smaller attributable impacts and vice-versa) is discussed further in \cite{leach_forecast-based_2021}. The adjustment means that any attributable impacts estimated directly from the forecasts are lower-bounds on the true anthropogenic impact. However, we find that attributable impacts on the heatwave are approximately linear with the coincidental global warming level within the perturbed forecasts across the range of leads explored (Supplementary Figure S7), consistent with \cite{seneviratne_regional_2020}. Hence in addition to the impacts estimated directly from the perturbed forecasts, we also present impacts scaled to the global warming level within the forecast at the time of the heatwave.

  \clearpage
  \begin{figure}[h]
    \centering
    \includegraphics[width=0.8\textwidth]{{Fig4.2}.png}
    \caption[Drivers of the PNW heatwave and their predictability in the 11-day lead forecast.]{\textbf{Drivers of the PNW heatwave and their predictability in the forecast initialised 2021-06-18 (11 days). Top row:} temperature anomaly fields for the PNW heatwave in the ensemble mean, nearest member and reanalysis. Solid black contours indicate 500 hPa geopotential height anomalies and stippling indicates regions with total cloud cover greater than 25\%. \textbf{Second row:} mean total column water vapour anomalies on the 25th June. The study region of 45-52N, 119-123W, over which fields are aggregated into timeseries, is indicated by the dotted rectangle. Anomalies shown are calculated relative to the 2001-2020 period. \textbf{Bottom three rows:} timeseries of daily maximum temperatures, total column water vapour and total cloud cover in each forecast ensemble member. The solid black line shows the reanalysis timeseries and the thick solid line shows the nearest member. The colour of each line indicates the rank of that ensemble member in terms of the peak temperature simulated during the heatwave period (dark grey = coolest, dark red = warmest). The solid black bar on the time axis of each panel indicates the averaging period used for the total column water vapour maps.}
  \end{figure}
  \clearpage

  \subsection{Results}

  The results of our forecast-based approach can be presented either as the attributable human influence on the intensity of the heatwave or the probability of the heatwave. We find that the intensity of the heatwave is reduced in the pre-industrial forecasts for all lead times (Figure 3). Due to the continual adjustment of the forecasts to the initial condition perturbations, the attributable influence on the heatwave peak temperature, estimated as half the difference between the pre-industrial and future forecasts to maximise the signal to noise ratio, increases as the lead time increases, ranging from 0.28 °C [0.25 , 0.33]\footnote{Numbers in square brackets represent a likely confidence range (17-83\%).} using the 3-day lead to 0.7 °C [0.35 , 1.0] using the seasonal forecast. We account for the continual adjustment of the perturbed forecasts by scaling the attributable influence by the ratio of the coinciding global land warming level to the observed present-day level of 1.6 °C \cite{osborn_land_2021}. This results in a best-estimate attributable impact on the heatwave intensity of 1.3 °C [0.5 , 1.9] for a current level of anthropogenic warming of 1.25 °C \cite{haustein_real-time_2017}. This accounts for approximately 20\% of the 7 °C 2021 anomaly over previous annual maxima.

  We quantify the attributable change in probability due to anthropogenic global warming using relative risk \cite{stone_end--end_2005}, estimating the probability of observing an extreme at least as extreme as the observed 2021 heatwave using an appropriate extreme-value or tail distribution, and then shifting this distribution by the attributable change in intensity for each lead time. As with the heatwave intensity, the relative risk tends to increase with forecast lead time due to the adjustment to the initial conditions. Our results are consistent with a linear relationship between log probabilities and the coinciding global land warming level. If we account for this adjustment by scaling log probabilities by the current global land warming level of 1.6 °C, we find a best-estimate relative risk of a factor of 8 times [2 , 30] considering all lead times, or analogously a fraction of attributable risk of 0.9 [0.5 , 0.97]. 

  Using the current rate of global warming over land \cite{haustein_real-time_2017} we can further estimate that the probability of observing an event at least as warm as the 2021 Pacific Northwest heatwave is doubling every 17 [10 , 50] years, and will continue to do so unless the rate of global warming decreases. Given the length of the historical record and our estimated change in probability over this period, such an event would be associated with a multi-decade to multi-century return period at the present-day, thus making this doubling time very relevant for adaptation planning. 

  \clearpage
    \begin{figure}[h]
      \centering
      \includegraphics[width=\textwidth]{{Fig4.3}.png}
      \caption[Return-time diagram of the PNW heatwave in the operational and counterfactual forecast ensembles.]{\textbf{Return-time diagram of the PNW heatwave in the operational and counterfactual forecast ensembles.} Each panel shows ensembles initialised at the lead given above the panel. Red, grey and blue dots indicate empirical return-time plots based on the ensemble members of the future, current and pre-industrial forecasts. The dashed grey line shows the temperature threshold observed during the PNW heatwave. The black dots indicate the recent climatology, based on detrended ERA5 reanalysis over 1950--2020. The solid grey line indicates the model climatology estimated using detrended hindcasts over 2001--2020 for the medium-range forecast, and using detrended and bias-corrected hindcasts over 1981--2020 for the seasonal forecast. The arrow in the left hand panel indicates, for illustration, the displacement along the log-scaled x-axis equivalent to a 5-fold increase in occurrence probability.}
    \end{figure}
    \clearpage

\section{Discussion}

  The results presented here provide strong evidence of the impact of climate change on a specific extreme event, based on a model that has been demonstrated unequivocally to be able to simulate the event in question through a successful medium-range forecast. Our estimates of relative risk are lower than previous climate model-based estimates \cite{philip_rapid_2021}, albeit are not entirely incompatible within the context of the associated uncertainties and the fact that our estimates represent a lower bound on the impact of climate change on the heatwave \cite[as was the case in][]{leach_forecast-based_2021}. The primary reason is that our model (unlike a typical climate model) is capable of simulating the multiple physical factors that contributed to the heatwave that occurred, so we are not relying on extrapolation of distributions from physically dissimilar events. Moreover, our imposed perturbations do not include the total sum of human influence on the climate. It is known that land surface feedbacks are important in the development of extreme heatwaves \cite{fischer_contribution_2007}, and is plausible that if we had removed the influence of anthropogenic climate change from the initial land state in addition to the ocean state, the resulting attribution statement might have been stronger.

  Nevertheless, we argue that the forecast-based methodology presented here represents an important advance in both attribution in general, and operational attribution. Rather than relying on multiple lines of evidence that would each be unsatisfactory in isolation, here we have presented a single adequate line. The key to the adequacy of the result is the ability of the model used to represent the event in question, demonstrated through successful prediction. This not only means that we have increased confidence in the model's response to external forcing \cite{palmer_simple_2018,palmer_nonlinear_1999}, but also that the analysis is a genuine attribution of the specific event that occurred (rather than a mixture of events that share some characteristic like extreme temperatures, but differ in other important meteorological aspects). Forecast-based attribution provides many of the advantages of the storyline approach to attribution, but can still be used to provide quantitative estimates of the changing probability of extreme events with climate change. The use of an operational weather forecast model demonstrates how this approach could be easily adapted to provide an operational system for attribution in real-time \cite[or potentially even in advance,][]{wang_initialized_2021}. Such a system would involve re-running operational forecasts with perturbed initial and boundary conditions as in the counterfactual forecasts we have presented here \cite{wehner_operational_2022}.

  There remain a number of ways in which the forecast-based approach explored here could be further developed. Firstly, analysis of the forecasts would be simplified if they were started from balanced states, rather than continually adjusting to the new initial conditions throughout the forecast. This could be done by either including additional perturbations to the initial conditions \cite[ie. to the land-surface and atmospheric states,][]{wang_initialized_2021,reed_attribution_2022,wehner_operational_2022}, or possibly by perturbing the initial state using the operational data-assimilation itself. Secondly, while here we have chosen to use the exact setup used operationally by ECMWF, the uncertainty of forecast-based attribution statements could be reduced by increasing the ensemble size \cite[we note that 51 members is a relatively small ensemble in the context of traditional attribution-specific experiments,][]{massey_weatherhome-development_2015,ciavarella_upgrade_2018}, particularly for the longer, relatively less-conditioned lead times.

  The focus of this study was on the attribution question, but this forecast-based methodology could be applied to produce projections designed to inform climate change adaptation. Analogous to our “future” counterfactual forecast, which we used here check the linearity of the climate change response, perturbations consistent with specific levels of global warming could be applied in order to, for example, simulate specific extreme events as if they occurred in a world of 2 °C. Such simulations of potential future extremes could be used to test the limits of regional adaptation in a targeted manner based on impactful events that have already occurred \cite{hazeleger_tales_2015}, complementing other approaches such as \cite{leach_generating_2022}, which was designed to produce a rich set of different extreme events rather than specific “grey-swan” type events.

  \paragraph*{Concluding remarks}

    In this study, we have used a numerical weather forecast-based approach to determine the contribution of human influence to a specific unprecedented extreme event. We used a state-of-the-art coupled operational weather forecast model that was unequivocally able to simulate the event in question, demonstrated by a successful prediction. Our perturbed initial condition approach maintains consistency with the measured changes in upper ocean heat content, unlike many previous approaches. We view this forecast-based approach as synthesising the storyline and probabilistic approaches to event attribution, keeping the event specificity of the storyline approach while still providing meaningful estimates of the changing risk of the extreme in question. Given that it is increasingly clear that we need to go beyond the meteorology of event attribution, and into the societal impacts \cite{mitchell_climate_2021,mitchell_increased_2022}, we suggest that our approach would be particularly well placed to advance this agenda, especially in the context of extremes in a future climate.

\section{Methods}

\subsection*{Event definition}

  How the extreme event of interest is quantified - the event definition - is a key methodological decision that must be made in extreme event attribution studies. A significant amount of previous work has shown the impact of the event definition on the quantitative outcome of the analysis \cite{uhe_comparison_2016,kirchmeieryoung_importance_2019,angelil_nonlinearity_2018}. In this study we use a definition consistent with a previous attribution study of the PNW heatwave \cite{philip_rapid_2021} to allow for a comparison between our forecast-based approach and their probabilistic statistical and climate-model based approach.

  We first average maximum temperatures over the region enclosed by 45--52N, 119--123W (indicated by the dotted rectangles in Figures 1 \& 2). For the event as observed in the ERA5 reanalysis \cite{hersbach_era5_2020} we then take the peak temperature recorded during the heatwave, which occurred at 00 UTC on 2021-06-29. For the event as simulated in the medium-range forecast ensemble members, we take the peak temperature that occurred between the 26-30th June, the period over which the heatwave occurred in reality. For the event as simulated in the seasonal forecast ensemble members, which we would not expect to predict the precise timing of the heatwave, we take the peak temperature over the full summer season. The differences between the event definitions of the medium-range and seasonal cases lead to the discrepancies in the climatologies shown in Figure 3.

\subsection*{Experiment details}

  \paragraph*{Model details}

    The medium-range experiments we have performed use the version of the IFS EPS that was operational at the time of the PNW heatwave, CY47R2 \cite{noauthor_ifs_2020}. The forecast model atmosphere is run at a resolution of O640 (~18km) and has 137 vertical levels. The atmosphere is coupled to a 0.25 degree wave model \cite{janssen_interaction_2004}, 0.25 degree sea ice model \cite{fichefet_sensitivity_1997}, LIM2, and 0.25 degree ocean model \cite{madec_nemo_2008}, NEMO v3.4, with 75 vertical levels (ORCA025Z75 configuration). We maintain the same number of ensemble members as the operational system, 51, throughout our experiments.
  
    The seasonal experiments are performed with ECMWF`s operational seasonal forecasting system, SEAS5 \cite{johnson_seas5_2019}. This uses IFS CY43R1 \cite{noauthor_ifs_2016} at a horizontal resolution of Tco319 (~36 km) with 91 vertical levels. The seasonal configuration of IFS CY43R1 is coupled to a 0.5 degree wave model, LIM2, and NEMO v3.4 in the ORCA025Z75 configuration. We maintain the same number of ensemble members as the operational system, 51, throughout our experiments.

  \paragraph*{Simulation setup}

  Our experiments all use the exact operational setup (model configuration and initial conditions) as their base. To this setup, we:

  \begin{enumerate}
    \item Change the CO2 concentrations used to a “pre-industrial” level of 285 ppm, and a “future” level of 615 ppm. These represent the same fractional change in opposite directions from the present-day concentration of 420 ppm used in the operational forecast system.
    \item Subtract (for the pre-industrial forecast) or add (for the future forecast) a perturbation of the estimated anthropogenic influence on the ocean state since the pre-industrial period from the initial conditions of the forecasts (through the ocean restart files). The estimation of this perturbation is described below. We use estimated perturbations for 3D temperature, sea ice concentration, and sea ice thickness.
    \item Check the sea ice fields for unphysical values. In the perturbed restarts, we ensure that sea ice concentration does not exceed 1 or subceed 0. We ensure that sea ice thickness does not subceed 0. Values outside these bounds are set to their nearest bound. Finally, we set sea ice thickness to 0 where sea ice concentration is 0, and vice versa.
    \item Modify ocean salinity such that in-situ ocean density is preserved following the 3D temperature perturbation as calculated using the equation of state from the forecast ocean model. The salinity compensation is achieved to machine precision using a simple gradient descent algorithm. The resulting coupled forecasts are thermodynamically consistent with the imposed ocean heat content anomalies without any adjustments to the initial ocean circulation, mixed layer depths, or horizontal pressure gradients. Importantly, and unlike uncoupled forecasts constrained by specified sea-surface temperatures, there are no infinite sources or sinks of heat in the resulting counterfactual forecasts. This approach is justifiable in shorter-range forecasts as there is no direct influence of salinity on the overlying atmosphere. This assumption may eventually break down at lead times comparable to ocean advective processes, for which there could be indirect feedbacks on the atmosphere associated with salinity-driven changes in the ocean state. Nevertheless, this approach works well for the medium-range and seasonal forecasts described in this study.
  \end{enumerate}

  The perturbations used are computed using an optimal fingerprint analysis \cite{hasselmann_optimal_1993,hasselmann_multi-pattern_1997,haustein_real-time_2017}. We first calculate the Anthropogenic Warming Index (AWI) using anthropogenic and natural radiative forcings from AR6 \cite{masson-delmotte_earths_2021} and the HadCRUT5 global mean surface temperature dataset \cite{morice_updated_2021}. The AWI provides us with a plausible estimate of the fingerprint of anthropogenic influence on other climate variables \cite{hasselmann_optimal_1993}. For each perturbed variable, we then regress observed timeseries at each gridpoint onto the AWI, using the following data sources:

  \begin{itemize}
    \item Sea ice thickness: ORAS5 (1958:2019) \cite{zuo_ecmwf_2019}
    \item Sea ice concentration: ORAS5 (1958:2019) \cite
    {zuo_ecmwf_2019}
    \item Sea surface temperature: HadISSTv1.1 (1870-2019) \cite{rayner_global_2003}
    \item Subsurface temperature: WOA18 (1950-2017) \cite{locarnini_world_2019}
  \end{itemize}

  We then scale the computed regression coefficients at each point by the change in AWI between the pre-industrial period of 1850-1900 and 2019 to produce our final estimated perturbations. The sea surface, and zonally and globally averaged temperature profiles are shown in Figure 4.

  Finally, we combine the sea surface and subsurface temperature perturbations. We did not use a subsurface temperature dataset in isolation since observations of the sea surface temperature are considerably more abundant in the early 20th century than observations of subsurface temperatures, and since the temperatures at and near the surface are likely to be the most important for the medium-range forecasts performed, we leveraged the additional information contained in observed sea surface temperatures. We combine the two by relaxing the sea surface perturbation towards the subsurface perturbation using a relaxation depth scale of 60m (the surface autocorrelation scale in WOA18).

  We note that estimation of the perturbation, and in particular the subsurface temperatures, is associated with considerable uncertainty due to the lack of observations in the pre-ARGO era. Here we have used a single best-estimate perturbation due to constraints on the available computational resource, but to account for this uncertainty an ensemble of perturbations could be applied \cite{sparrow_attributing_2018}. A possible way in which such an ensemble could be derived would be to apply optimal fingerprinting to an ensemble of coupled climate models.

  \clearpage
  \begin{figure}[h]
    \centering
    \includegraphics[width=\textwidth]{{Fig4.S1}.png}
    \caption[The initial ocean state perturbation applied.]{\textbf{The initial ocean state perturbation applied. Left panel:} map of the surface temperature perturbation. \textbf{Inset:} timeseries of annual maximum temperatures for the same dotted region. \textbf{Top right panel:} map of zonally averaged temperature perturbations as a function of depth. \textbf{Bottom right panel:} globally averaged temperature perturbation as a function of depth. Note that the x-axis switches from a linear to logarithmic scale at a depth of 500m.}
  \end{figure}
  \clearpage

\subsection*{Bias correction of seasonal forecast ensembles}

  Climate drift can be an issue in the use of coupled seasonal forecast models \cite{stockdale_coupled_1997}. We find a non-negligible drift in the daily maximum temperature SEAS5 forecast ensemble initialised in May over the PNW region. This drift results in a positive temperature bias that grows with lead time. Hence using the raw model output in our analysis would overestimate the probability of the PNW heatwave. 

  To account for the drift, we perform a simple bias-correction procedure on the seasonal forecast ensembles, informed by comparing the SEAS5 hindcasts over 1981-2020 with ERA5 reanalysis data over 1950-2020 (using the full time period that data is available and excluding the year of the event, 2021). We do this in three steps:
  
  \begin{enumerate}
    \item Remove the attributable forced trend from both the reanalysis and hindcasts by regressing mean JJA daily maximum temperatures onto the AWI \cite{hasselmann_optimal_1993}.
    \item Remove the drift from these detrended hindcasts, estimated by averaging the hindcasts for each lead time over all years and ensemble members, subtracting this from the corresponding reanalysis average over all years, and then regressing this timeseries onto the (linear) lead times, producing a linearly lead-time dependent drift correction \cite{stockdale_coupled_1997}.
    \item The drift-corrected hindcasts still exhibit a positive bias during periods of extreme high temperatures. Hence we finally remove the remaining mean bias in annual maximum temperatures in the hindcasts compared to reanalysis.
  \end{enumerate}

  We apply this bias correction procedure to both the seasonal hindcasts shown in Figure 3 and used to estimate the return time of the event, and to the operational and perturbed seasonal forecasts of the 2021 summer. Figure S2 shows the results of this bias correction procedure, following Thompson et al. \cite{thompson_high_2017}.

  We note that validation of the bias correction procedure on the SEAS5 distribution of annual maximum temperatures (TXx) is challenging due to the unprecedented nature of the 2021 event. If we perform an analysis of the higher-order moments of the SEAS5 and “observed” (ERA5 reanalysis over 1950-2020) distributions of TXx \cite{thompson_high_2017}, we find that the bias-corrected ensemble tends to have larger values of higher-order moments than the observed timeseries. However, if the 2021 event is included in the observed distribution, then the opposite is found, due to the large impact of such an outlier on these moments. This sensitivity to inclusion / exclusion of the 2021 event, demonstrated in Fig S2, is why we have opted to perform a simple but physically motivated bias correction rather than a more complex statistical correction such as a quantile map.

  \clearpage
  \begin{figure}[h]
    \centering
    \includegraphics[width=\textwidth]{{Fig4.S2}.png}
    \caption[Validation of the bias correction applied to the SEAS5 seasonal forecast simulations.]{\textbf{Validation of the bias correction applied to the SEAS5 seasonal forecast simulations}, following \cite{thompson_high_2017} Figure 2.}
  \end{figure}
  \clearpage

\subsection*{Statistical methodology}

  \paragraph*{Intensity changes}

    We calculate changes in intensity as the difference between the average of the nearest quintile of each ensemble to the event (in terms of peak temperatures). For the three longer leads, this is effectively the difference between the averages of the uppermost quintile of the two ensembles.

  \paragraph*{Risk changes}

    We calculate the risk ratio by first fitting either a generalised extreme value (GEV) distribution to the full operational ensemble (for the shortest lead) or a straight line on a return-time diagram (ie. an exponential tail) to the nearest quintile of either the operational ensemble (for the other two medium-range leads) or the model climatology (for the seasonal lead, since the tail of the operational ensemble lies considerably further below the event threshold than the tail of the much larger model climatology). We do this because while the shortest lead ensemble is well represented by a GEV distribution, the other three are not, and have generally heavier tails than estimated by likelihood-maximising GEV distributions. In these cases, where the event threshold lies in the extreme tail of the ensemble, the tail properties of the approximating distribution project considerably onto the estimated probability of the event. Hence to avoid any undue assumptions on the tail shape, we fit a straight line on a return-time diagram such as Figure 3 (assuming an exponential tail) to the nearest quintile. 

    After fitting an appropriate distribution, we then shift the location of this distribution by the estimated attributable warming. We then calculate the probability of observing an event at least as intense as the PNW heatwave (the dashed line in Figure 3) in the original distribution and the shifted distribution. The risk ratio is the ratio of these two probabilities ($P_\text{current} / P_\text{shifted}$).

    Throughout, confidence intervals are calculated using a 10,000 member non-parametric bootstrap with replacement.

\section{Chapter close}

\begin{savequote}[8cm]
    Quote
      \qauthor{--- author}
\end{savequote}
    
\chapter{\label{discussion}Discussion} 

Chapter description.
% \small\paragraph{Author contributions:} This chapter is based on the the following publication \footnote{with the author contributions as follows.} \par\vspace{1em}
% \formatchref{Surname, I1. I2., Surname, I1. I2.}{year}{Title}{Journal}{vol}{issue}{pages}{DOI}

\minitoc

\clearpage

\section{An overview of this thesis}

  \blindtext

\section{This thesis in the context of previous work}

  References of relevant work. Hurricane forecast-based \citep{reed_attribution_2022,reed_forecasted_2020,patricola_anthropogenic_2018,lackmann_hurricane_2015,takayabu_climate_2015}. Nested forecast modelling \citep{schaller_role_2020,meredith_crucial_2015}. Initialised (sub)seasonal \citep{hope_contributors_2015,hope_what_2016,hope_determining_2019,hope_subseasonal_2022,wang_initialized_2021,tradowsky_toward_2022,stone_effect_2022}. DADA \citep{hannart_dada_2016}. Probabilistic methodology ref \citep{pall_anthropogenic_2011}. Pseudo global warming \citep{schar_surrogate_1996,pall_diagnosing_2017}.

\section{Limitations}

  Although I have discussed various limitations of the individual studies that make up this thesis, in this section I consider some of the limitations of the forecast-based approach to attribution developed and explored here as a whole.

  \paragraph*{Forecast adjustment}

    One key aspect of the counterfactual forecasts performed here is that the model --- or more specifically the model atmosphere and land surface --- adjusts continually to the imposed perturbations throughout the integration. This means that the further into the forecast the event of interest happens, the stronger the attributed impact of those perturbations is. At the same time, as the forecast evolves, this effect becomes more uncertain in general (though not necessarily always), due to the increasing dynamical noise arising from the chaotic nature of the weather system. This combination of this increasing strength and uncertainty can make analysing and interpreting the results of the counterfactual forecast experiments difficult. In chapter \ref{ch4} I alleviated this difficulty by making use of the fact that the attributable regional impacts of climate change were near-linearly related to the coincidental measured level of global warming. This linear relationship allowed me to benchmark the estimated impact at each forecast lead time to the same level of global warming, regardless of how adjusted they were at the time of the event in question. However, this linear relationship is not guaranteed for every extreme event, and therefore it would be valuable to find methodologies by which this adjustment could either be reduced or removed entirely. I consider a few ideas to achieve this below.

  \paragraph*{Additional uncertainty dimensions}

    In the experiments performed here, I have only considered uncertainty arising from the chaotic nature of the weather system. However, there are additional uncertainties associated with the approach I have developed. One dimension that has been explored in prior work is the uncertainty in the estimation of the ``human fingerprint'' that is removed from the model initial conditions. For example, \citet{pall_anthropogenic_2011}, who removed such a fingerprint from the SSTs (and sea ice fields) of the inital conditions in their naturalised simulations, used four estimates of the warming pattern based on different coupled climate models. This allowed them to test the sensitivity of their attribution statements to the warming pattern used. This approach of using an ensemble of coupled climate models to derive a corresponding ensemble of human fingerprints in order to more completely sample this dimension of uncertainty space has since been used in several other studies \citep{schaller_human_2016}; though many attribution studies and systems still use a single estimate, often based on a multi-model mean pattern \citep{ciavarella_upgrade_2018,stone_benchmark_2021}. 

    In this thesis, I used a single pattern estimated from observations. I used observations to avoid over-reliance on a single coupled climate model, given the biases and known issues present in such models. Although it would have been extremely interesting to more completely explore this dimension of uncertainty (given observations of the ocean subsurface are by no means perfect, especially pre-2000), limits to computer resources prevented me from doing so within the scope of this thesis. However, I hope that such an exploration could be carried out in the future. Doing so would be conceptually straightforward, simply involving treating historical output from a set of different coupled climate models exactly as if they were the observations that I used. Quantitative attribution results derived from each coupled model estimate could be compared to test the sensitivity of such results to the estimate of the human fingerprint used. Because of the variation in the representation of transient historical climate within coupled climate models, each model-derived fingerprint might have to be scaled by (for example) the present-day level of global warming within the model for consistency \citep{tokarska_past_2020}.

  \paragraph*{Single model}

    Within the core research of this thesis concerning forecast-based attribution, I have used a single model, ECMWF's IFS. There were a number of reasons for this limitation: ECMWF provided a mechanism for me to access the computing resources I required through their special project; there were also individuals at ECMWF who provided the technical expertise I needed to design and perform the counterfactual forecast experiemnts; the IFS is one of the (if not the) best performing numerical weather prediction models on a global basis \citep{hagedorn_comparing_2012}; and applying the couterfctual forecast methodology to other models would have presented considerable technical challenges that lie beyond the scope of this thesis. However, the quantitative results presented here may be sensitive to this choice of model. 
    
    There is considerable variation in both the global and regional response to external forcing among climate models \cite{meehl_context_2020,seneviratne_regional_2020,masson-delmotte_human_2021,masson-delmotte_earths_2021,masson-delmotte_linking_2021,masson-delmotte_weather_2021}. This variation is the reason why \citet{philip_protocol_2020} suggest that having as large a set of different climate models as possible is important for a probabilistic attribution study. Although still a point that should not be overlooked, I argue that such a multi-model assessment is not as important when using the counterfactual forecast approach introduced here. Firstly, a successful prediction ensures (when combined with a limited validation that the prediction did not occur for the wrong reasons) that the model used is able to represent the physical processes of the event in question, and that vital processes are not missing, as may be the case in some climate models. This grounding in the specific physics of the event means that the simulated response to external forcing is considerably more certain and less model dependent. Secondly, the use of a \emph{reliable} forecast ensemble to assess probability ensures that these probabilities are representative of the full space of possible states of the climate system given the initial conditions of the forecast \citep{murphy_new_1973}. This is not the case for climate model simulations, including perturbed parameter ensembles. However, despite these mitigating factors, exploring the sensitivity of the couterfactual forecast approach to the model used is an important question for future research. This could be done by carrying out an identical experiment in (for example) the Met Office's numerical weather prediction systems \citep{walters_met_2017,maclachlan_global_2015}.

  \paragraph*{Single event class}

    This thesis has concentrated on extreme heat events. However, given the major contribution of the thesis to attribution literature has been the forecast-based approach taken, rather than understanding the specific type of extreme event studied, this does represent a limitation. There are good reasons for this focus on heatwaves, given the possible scope of a thesis: they have severe associated impacts; are generally well understood; and have been the subject of a large body of prior attribution literature. However, demonstrating that the forecast-based approach can be used for other classes of extreme event will be vital for the method to be taken up widely. A possible candidate for the next class to study would be a high precipitation event. Attribution of high precipitation extremes is generally more challenging than of heatwaves, due to the smaller spatial scales involved, though there still exists a considerable amount of prior work that addresses this question. The high resolution of weather forecast models certainly makes them a more appropriate tool than coarse climate models for studying localised extremes.

  \paragraph*{Considering additional forcing agents}

    In chapter \ref{ch4}, the `complete' estimate of human influence on the Pacific Northwest Heatwave was derived by removing human influence on ocean heat content (by reducing the 3D ocean temperature) and reducing the levels of CO$_2$ in the atmosphere back to their pre-industrial levels. Although we argue that this represents a good estimate of the total sum of human influence, there are a number of additional sources of anthropogenic forcing on the climate system that may need to be considered in future work. Increased levels of other greenhouse gases such as methane or nitrous oxide have a similar radiative effect to CO$_2$, though the forcing from these other agents is relatively small in magnitude compared to CO$_2$ \citep{masson-delmotte_earths_2021}. The other significant human contribution arises from aerosol emissions. Unlike greenhouse gases, historical aerosol emissions have reduced the energy imbalance of the earth, thus masking some of the global warming caused by greenhouse gases. Additionally, while greehouse gases are well-mixed throughout the atmosphere, aerosols are highly localised in space due to their short lifetime. This means that their effect on local climate can vary considerably from region to region. In this thesis we only considered forcing from increases in CO$_2$ concentrations since i) forcings from these other sources approximately cancel each other out on global scales, and ii) the IFS does not include an interactive atmospheric chemistry model, but instead uses an aerosol climatology \citep{bozzo_aerosol_2020}. There has been relatively little research into the effects of aerosols on heatwaves specifically \citep{horton_review_2016}, but it has been found that aerosol reductions in the future exacerbate increases in heatwave magnitude arising from continued greenhouse gas emissions \citep{zhao_strong_2019}. Including the effect of these additional forcing agents on specific extreme weather events would be an extremely interesting direction for future research. The effect of aerosols may be especially interesting for precipitation extremes, since aerosols are known to have direct impacts on cloud formation. However, while including the radiative effects from other greenhouse gases would be straightforward, and could be done exactly as has been for CO$_2$, including the effects from aerosol emissions would likely be considerably more technically complicated and subject to large uncertainty, though might be possible using a version of IFS that includes a tropospheric aerosol scheme \citep{remy_description_2019}.

\section{Future research directions}

  \subsection{Addressing the rapid atmospheric adjustment}

    In this section, I discuss possibilities for how the methodology used here could be altered in order to remove the issues associated with the rapid adjustment of the forecast model, in the atmosphere and at the land surface, to the perturbed intial state. Removing (or alleviating) this adjustment would considerably simplify the interpretation and analysis of couterfactual forecast experiments.

    \paragraph*{Perturbing the initial atmospheric state}

      The simplest way in which to initialise the model from an atmospheric (and land-surface) state that is closer to thermodynamic equilibrium would be to attempt to perturb it such that it is consistent with the changes made to the oceanic and boundary (CO$_2$) conditions. This approach is based on the pseudo-global warming framework \citep{schar_surrogate_1996}, and has been used by a number of recent attribution studies \citep[][]{pall_diagnosing_2017,patricola_anthropogenic_2018,wehner_estimating_2019,reed_forecasted_2020,reed_attribution_2022}. However, unlike these studies and the original framework, which perturb the lateral boundary and initial conditions of a high-resolution nested regional model, in our case we would need to perturb the forecast model globally. These studies typically perturb 3D thermodynamic fields such as temperature, humidity and geopotential based on simulated climate change in GCMs. However, it would be possible to avoid reliance on such models by estimating anthropogenic fingerprints in these fields from long-term reanalysis data \cite{hersbach_era5_2020,laloyaux_cera-20c_2018} exactly as was done to estimate the ocean state perturbations in chapter \ref{ch4}. The disadvantage of this approach would be that the imposed atmospheric perturbations could cause unexpected changes to the physical processes driving the event in question that would be difficult to distinguish from real attributable changes to the event. For example, changes to the temperature and moisture fields could affect local atmospheric circulation in a way that is not necessarily physically consistent with how the same event might have evolved in a climate without human influence.

    \paragraph*{Assimilating the perturbations}

      The simple perturbation approach could be extended to counter some of the issues with physical consistency by coupling it to the data assimilation procedure used to generate the model initial conditions. Data assimilation aims to create the best possible forecast initial state by combining recent model predictions with observations \citep{kalman_new_1960}. It may therefore have the potential to generate a balanced --- but also physically consistent --- initial state for the couterfactual forecasts. Data assimilation has been previously proposed as a technique that could be used for extreme event attribution by \citet{hannart_dada_2016}, though they suggested using likelihoods output from the data assimilation procedure directly, rather than using the procedure to create initial conditions for couterfactual forecasts. The basic idea would be to replace the operational version of the forecast model with an `unforced' version during the data assimilation cycle. This unforced version would essentially be identical to the perturbed initial and boundary condition model run to produce the counterfactual forecasts in chapter \ref{ch4}. The aim behind this unforced data assimilation cycle would be to produce a physically consistent initial state in the unforced model that is as close as possible (in such a climate without human influence) to the observed state of the climate system. How similar the original operational and unforced counterfactual initial states produced would be would depend heavily on the real-world state at the time of the data assimilation. Although this is a promising idea in theory, it would likely face significant technical challenges. For example, some of the observations used in the data assimilation may also have to be perturbed for the procedure to succeed (especially when using a perturbed ocean state if observations of the ocean are used). Given the incomplete nature of the observations and variety of their sources, altering observations before the data assimilation step could be extremely difficult. These potential challenges mean that collaboration with an expert in the operational data assimilation system used would be essential.

    \paragraph*{An operational approach using successive forecasts}

      The previous two methods could be used to perform a counterfactual forecast for a single event. My final suggestion is potentially simpler than both, but would only work in the case of an operational, regularly run couterfactual forecast system for attribution and projection. This idea would be to use the previous forecast to calculate the perturbation required to create a balanced initial state, and is probably clearest expressed mathematically. If we call the operational (assimilated) initial state at time $\tau$ be $\chi(\tau)$, the operational `forecast operator', that transforms an initial state into a prediction at time $t$ after initialisation $G_t^1$, and the equivalent counterfactual operator $G_t^0$, then operational and counterfactual forecast states $X$ at time $t$ after initialisation time $\tau$ can be written respectively as

      \begin{align*}
        X^0(t|\tau) &= G_t^0[\chi(\tau)] \ \textnormal{and}\\
        X^1(t|\tau) &= G_t^1[\chi(\tau)]\,.
      \end{align*}

      \noindent And the difference between the factual operational state and counterfactual state, as estimated by the physics of the forecast model can be written

      \begin{equation}
        \Delta X(t|\tau) = X^1(t|\tau) - X^0(t|\tau)\,.
      \end{equation}

      \noindent Now for a sufficiently small time $t$, such that the model has significant skill and dynamical noise is low, this $\Delta X$ represents a good, and physically consistent, estimate of the difference between the factual and counterfactual worlds at time $\tau + t$. Hence if we then want to issue a successive forecast at time $\tau + t$, rather than simply using $\chi(\tau+t)$ to initialise both forecasts, we could use $\chi(\tau+t)$ for the operational forecast, and $\chi(\tau+t) - \Delta X(t|\tau)$ for the counterfactual forecast. As this routine is applied to several successive operational and counterfactual forecasts in a row, the differences between $\Delta X(t|\tau)$ and $\Delta X(t|\tau+t)$ should tend to a small (though non zero) value, as $\Delta X$ tends towards the `real' difference between a balanced factual initial state and an analogue state in the counterfactual world. This difference won't ever reach zero, since the difference between factual and counterfactual worlds at a certain time is dependent on the climate state at that same time. After performing this routine several times, this $\Delta X$ should tend towards the difference between assimilated factual and counterfactual initial states (ie. as would be obtained by assimilating the perturbations, described above).

      The need to continually apply this adjustment to successive forecasts, perhaps a few days apart, is why this routine would only work in the case of an operational system. Its relative simplicity in comparison to the perturbed data assimilation approach, and physical consistency compared with the simple perturbation approach make it attractive. Conceptually, it is somewhat similar to the methodology developed by \citet{wang_initialized_2021}. They used separate long-running integrations of the same forecast model at different CO$_2$ levels to derive the perturbations applied to the initial conditions for their counterfactual simulations. The main differences are that i) this approach does not require costly separate simulations to determine the perturbations since the perturbations are developed through several successive forecasts, and ii) the perturbations generated here would be more closely linked to the actual state of the climate system at the time the forecasts are initialised. However, despite the advantages of this approach, it is still very likely that there would be technical challenges to address. For example, we would have to ensure that errors in $\Delta X$, possibly arising from dynamical noise or forecast error, do not grow between successive forecasts. If this happened, the factual and counterfactual initial states would move further and further apart, and any differences between the factual and counterfactual forecasts could then not be attributed to human influence because of the confounding error present. The shorter the time between successive runs of such an operational attribution system, the less likely for issues like this to occur. The other clear challenge would be determining what variables to perturb. Although it would be possible to perturb everything in the initial conditions, it might be more robust to only apply this adjustment to thermodynamic variables that have a physical basis for being perturbed (just as in the pseudo-global warming approach). Nevertheless, this approach is an intriguing prospect for robust operational attribution using a weather forecast system. 

  \subsection{Expanding the scope of this work}

    This section explores various tests for the methodology used here that should be carried out in further work to more completely assess the robustness of the approach. These tests could be completed without any major changes to the methodology itself.

    \paragraph*{Alternative extremes}

      As mentioned in the Limitations above, this thesis has focused on heatwaves. However, there are many other weather extremes that are of scientific and public interest. Hence, testing the robustness of the forecast-based approach to other extremes would be a natural next step to take. Given their significant coverage in the literature, a high precipitation event would be a good choice for such a test. I note that there would be potential additional considerations when examining a precipitation extreme compared to a heatwave. Firstly, the smaller spatial scale of such precipitation extremes means that these scales might have to be taken into account during the attribution step --- what if the center of the extreme shifts in the counterfactual world \citep{schaller_role_2020}? Such shifts mean that, especially if linking the precipitation to flooding impacts, pooled catchments, rather than catchments on an individual basis, may have to be considered. The other significant difference is that precipitation forecasts are typically less skillful than temperature forecasts for lead times of more than a few days \citep{rodwell_medium-range_2006,vitart_evolution_2014,swinbank_tigge_2016,monhart_skill_2018,mishra_multi-model_2019,haiden_evaluation_2021}. This reduced skill (particularly during the 1-2 week period, where tempreature forecasts are generally still good) means that shorter lead times may have to be used to ensure that the forecast model is still able to capture the extreme event within its ensemble --- even if the forecast is reliable. Reducing the lead time would make addressing the forecast adjustment to the perturbed initial conditions even more important.

    \paragraph*{Alternative forecast models}

      Another limitation discussed above was the use of a single model, ECMWF's IFS. As such, testing the sensitivity of this approach to the particular forecast model used would be an important step. It would make sense to begin with a forecast system that uses the same ocean model as the IFS, NEMO 3.4 in ORCA025Z75 configuration \citep{madec_nemo_2008}. This would allow bit-identical perturbations to be used, thus minimising differences that arise from experimental setup as opposed to those that we are interested in that arise from model choice. However, a shared ocean model is not an absolute necessity as one of the steps taken to produce the ocean state perturbations in chapter \ref{ch4} was to interpolate the perturbations onto the ORCA025Z75 model grid --- in theory any ocean grid (and thus any ocean model) could have been used. The Met Office's medium-range MOGREPS-G and seasonal GloSea5 ensemble forecasting systems both use the NEMO ocean model on the same grid, and so may make good candidates for applying the forecast-based approach in an alternative model \citep{maclachlan_global_2015}.

    \paragraph*{Incorporating perturbation uncertainty}

      Including uncertainties associated with the estimation of the perturbed initial state in the forecast-based approach may be important, as discussed above and shown in previous work \citep{pall_anthropogenic_2011,sparrow_attributing_2018}. The main difficulty of including these uncertainties comes from the additional computational cost: if we were to simply repeat our experiments using climate model-derived estimates of these perturbations the cost would scale with the number of climate models used. In order to get a reasonable representation of the uncertainty, O(10) model estimates would be required \citep[as in][]{sparrow_attributing_2018}. This would immediately increase the computational cost by a factor of 10 --- possibly feasible for a single experiment, but much less desirable from an operational perspective. However, given the current operational ensemble prediction system at ECMWF already uses a 5-member ensemble of ocean analyses from which the 51 forecast ensemble members are initialised from, it is possible that we could use a similar ensemble of perturbations within a single forecast ensemble. There are a number of additional outstanding questions: i) how to choose the climate models from which these perturbations are estimated; ii) whether to give the climate models equal weight, or weight them based on some form of model evaluation \citep[this is especially relevant for the latest generation of climate models, the CMIP6 ensemble,][]{eyring_overview_2016,hausfather_climate_2022}; and iii) if using both observation- and climate model-based estimates, how to combine them?

  \subsection{Alternative applications}

    This thesis has largely focussed on the physical attribution of extreme weather events --- which is an intrinsically backward-looking question. However, the approach explored here has potential to inform future projections of climate change as well. In this section, I discuss forward-looking applications of this work and also how it might further research into attribution and projection of the societal impacts arising from extreme weather, which is a rapidly developing field at the moment.

    \paragraph*{Projections of future extremes}

      The question that this thesis has been concerned with answering is how human influence on the climate over the past century or so has affected the probability and severity of extreme events occuring in the present-day. This question is of considerable importance for the numerous reasons detailed in chapter \ref{intro}. However, an arguably more policy-relevant question is that of how extreme weather events may change in the future --- this is especially critical for adaptation planning \citep{harrington_integrating_2022}. Providing information that is specific enough to be useful in a policy context often requires more granularity than coarse climate models are able to provide. Hence, statistical or dynamical downscaling is typically used in order to increase the utility of climate model simulations --- for example in the United Kingdom Climate Projections (UKCP) reports \citep{lowe_uk_2009,murphy_uk_2009,lowe_ukcp18_2018,murphy_ukcp18_2018}. However, given the structural errors in current climate models, this approach cannot be expected to provide entirely robust and reliable probabilstic imformation, especially on the scales that adaptation planners require. This was the reason why in 2015 \citeauthor{hazeleger_tales_2015} suggested that a complementary approach would be to construct `what if' scenarios using high-resolution weather forecast models that would be able to provide the specific information required; especially given their position in current extreme weather hazard warning systems \citep{schaller_role_2020}. This approach was called `Tales of future weather' \citep{hazeleger_tales_2015}.

      I suggest that the forecast-based approach developed in this thesis could be an attractive methodology for constructing such Tales. The idea would be to take a set of damaging historical extreme weather events (that were successfully forecast), perturb the forecast initial conditions exactly as done in chapter \ref{ch4}, and thus produce realisations of these events in a warmer world. These future forecasts could then be used to examine how future impacts might be worse than in the present, and thus how adaptation policies could be implemented in order to mitigate such impacts. I have already essentially done this future forecast experiment --- though here they were used to test the linearity of the response, rather than for climate projection specifically. The methodology I have used to calculate the anthropogenic fingerprints to be removed from the forecast initial conditions could not just determine the estimated perturbation between pre-industrial and present-day climates, but between climates separated by specified levels of global warming \citep{hasselmann_optimal_1993}. For instance, one could construct forecast-based Tales for policy-relevant future warming levels of 1.5, 2, 3 and 4 °C. One advantage that this forecast-based approach has over a storyline approach \citep[eg.][]{benitez_july_2022} in this context is that ensemble forecasts do not just reproduce the event as it unfolded, but also possible alternative realisations that may be even more extreme. This `ensemble-boosting' aspect of such a forecast-based approach to extreme weather projections could help to ensure potential impacts are not underestimated as a result of limiting our view to the outcome that did occur by exploring the range of physically consistent possible outcomes \citep{gessner_very_2021}.
      
      One issue with using an optimal fingerprinting approach to estimating the perturbations required is that it assumes that the pattern of global warming remains constant into the future, which is not certain to be the case, in particular for the higher levels of warming. The discussion above on incorporating perturbation uncertainty is relevant to this \citep{zhou_greater_2021}. For example, perturbations to particular levels of global warming could be derived from coupled climate models in addition to observations in order to more completely span the space of possible future patterns of warming. Another apparent issue with this forecast-based approach arises due to the reliance on historically damaging events. The length of the historical record means that regional coverage of such events will vary considerably. For example, while many regions may have experienced 1-in-100 to 1-in-1000 year heatwaves over the course of the historical record, many will not have. This could potentially leave these regions under-informed in terms of the risk from climate change exacerated extremes. However, there are already-developed approaches to counter this issue in the literature. One of the most relevant is the UNSEEN approach \citep{thompson_high_2017,kelder_using_2020}. Briefly, this approach uses seasonal ensemble hindcasts to considerably increase the effective sample size of such events within the historical record. I suggest that one way in which useful Tales could be constructed would be to not only look for damaging events in the historical record, but also within the seasonal and medium-range hindcast ensembles that are available. Such `unseen' events could then be re-forecast within a future climate to explore how they may change under continued global warming. A disadvantage of this approach is that such unseen events were not necessarily successfully forecast (which is one of the key features of the forecast-based approaches exlpored in this thesis), though model fidelity and reliability could be validated in other ways in this case \citep{kelder_interpreting_2022}.

    \paragraph*{Impact assessment}

      Impact attribution is a rapidly growing field of research \citep{perkins-kirkpatrick_attribution_2022,burger_law_2020}. Linking the attributable physical changes due to climate change to the socio-economic impacts can be extremely powerful as a communication tool and as a way to drive policy \citep{clarke_inventories_2021}. Some examples of this linkage include economic damages from hurricanes \citep{frame_economic_2020,strauss_economic_2021} and mortality from heatwaves \citep{mitchell_attributing_2016,mitchell_climate_2021,lo_estimating_2022}. Despite its clear importance, impact attribution has only taken off recently, possibly due to the difficulties that arise as a result of the additional uncertainties (and non-linearities) associated with linking physical to socioeconomic impacts. This section will not be a lengthy discussion, but I suggest that there are a number of reasons why forecast-based approaches could complement and advance current approaches.

      One key reason, that I have mentioned previously, is that weather forecasts are already built into the modelling chains used to assess risk from extreme weather by combining physical hazard and vulnerability information \citep{schaller_role_2020}. Given how important the vulnerability aspect of extreme weather risk is \citep{raju_stop_2022,mitchell_increased_2022}, using models already familiar to those with relevant expertise is a significant advantage. In additional to this familiarity advantage, the fact that weather forecasts are already key components of many well-validated impact prediction systems \citep[for example, the GloFAS flood warning system][]{alfieri_glofas_2013} means that impact attribution may be able to be carried out with very little technical work --- simply by switching operational weather forecasts for counterfactual ones. For example, \citet{wilkinson_consequence_2022} developed a scheme for translating weather forecasts into damages. Such a scheme could essentially be used `as is' to generate estimates of attributable damages to climate change. Trustworthy estimates of attributable damages could further support litigation relating to increased extreme weather risk. The arguments that I have made at length for forecast-based approaches in this thesis are also very relvant here: significantly increased resolution, well-established reliability, implicit and explicit model validation and event specificity. This is certainly not to say that climate models cannot provide useful information about general impacts arising from climate change (for example, through the storyline framework), but for assessing risks from many types of extreme weather, I argue that weather forecast models would be a more robust tool \citep{palmer_simple_2018}.

\section{Concluding remarks}


%% APPENDICES %% 
% Starts lettered appendices, adds a heading in table of contents, and adds a
%    page that just says "Appendices" to signal the end of your main text.
\startappendices
% Add or remove any appendices you'd like here:
\chapter{\label{resources}Resources}

Examples of output from the thesis besides research:
\begin{itemize}
    \item mystatsfunctions plus other packages (eg. FaIR)
    \item Code accompanying papers
    \item CEDA archive data
    \item MARS datasets
\end{itemize}

Media articles concerning research from this thesis
\begin{itemize}
    \item CarbonBrief
    \item Science article
\end{itemize}
\chapter{\label{app2}Appendix 2}


%%%%% REFERENCES

% JEM: Quote for the top of references (just like a chapter quote if you're using them).  Comment to skip.
\begin{savequote}[8cm]
  Quote
  \qauthor{--- author}
\end{savequote}

% NJL: this sets the spacing since baselineskip commands don't seem to work...
\singlespacing

\setlength{\baselineskip}{0pt} % JEM: Single-space References

{\renewcommand*\MakeUppercase[1]{#1}%
\printbibliography[heading=bibintoc,title={\bibtitle}]}


\end{document}
