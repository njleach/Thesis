\begin{savequote}[8cm]
    Quote
      \qauthor{--- author}
\end{savequote}
    
\chapter{\label{ch1}Conventional probabilistic attribution} 

Here I present a probabilistic extreme event attribution of the 2018 European heatwave. Whilst demonstrating the methodologies behind this framework, I examine how one particular aspect of probablistic event attribution -- the definition of the event -- projects strongly onto the quantitative results. In the closing remarks, I reflect on potential issues with the approach taken within the chapter, and suggest ways in which these could be overcome.
\small\paragraph{Author contributions:} This chapter is based on the the following publication \footnote{with the author contributing as follows. Conceptualisation, Data curation, Formal analysis, Investigation, Methodology, Resources, Visualisation and Writing -- original draft.} \par\vspace{1em}
\formatchref{Leach, N. J., Li, S., Sparrow, S., van Oldenborgh, G. J., Lott, F. C., Weisheimer, A., \& Allen, M. R.}{2020}{Anthropogenic Influence on the 2018 Summer Warm Spell in Europe: The Impact of Different Spatio-Temporal Scales}{Bulletin of the American Meteorological Society}{101}{1}{S41-S46}{https://doi.org/10.1175/BAMS-D-19-0201.1}


\minitoc

\clearpage

\section{Chapter open}

\section{Abstract}

  We demonstrate that, in attribution studies, events defined over longer time scales generally produce higher probability ratios due to lower interannual variability, reconciling seemingly inconsistent attribution results of Europe's 2018 summer heatwaves in reported studies.

\section{The 2018 heatwave in Europe}

  The summer of 2018 was extremely warm in parts of Europe, particularly Scandinavia, the Iberian Peninsula, and central Europe, with a range of all-time temperature records set across the continent \citep{johnston_heatwave_2018,nesdis_record_2018}. Impacts were felt across Europe, with wildfires burning in Sweden \citep{krikken_attribution_2021,watts_wildfires_2018}, heatstroke deaths in Spain \citep{publico_nueve_2018}, and widespread drought \citep{harris_heat_2018}. During the summer, the World Weather Attribution (WWA) initiative released an analysis of the heat spell \citep{world_weather_attribution_heatwave_2018} based on observations/forecasts and models in specific locations (Dublin, Ireland; De Bilt, Netherlands; Copenhagen, Denmark; Oslo, Norway; Linkoping, Sweden; Sodankyla, Finland; Jokionen, Finland), which concluded that the increase in likelihood due to human induced climate change was at least 2 to 5 times. In December, the U.K. Met Office (UKMO) stated that they found the 2018 U.K. summer temperatures were made 30 times more likely \citep{press_office_chance_2018,mccarthy_drivers_2019}. These two estimates appear to quantitatively disagree; however, we show they can be reconciled by investigating the effects of using different spatial domains and temporal scales in the event definition. We also demonstrate that prescribed SST model simulations can underrepresent the variability of temperature extremes, especially near the coast, with implications for any derived attribution results.

  % \clearpage
  \begin{figure}[h]
    \centering
    \includegraphics[width=\textwidth]{{Fig1.1}.pdf}
    \caption[LOF caption]{}
    \label{fig2.1}
  \end{figure}
  % \clearpage

\subsection{Defining the event}

  We consider various temperature-based event definitions to demonstrate the impact of this choice in attribution assessments, and assess to what extent human influence affected the seasonal and peak magnitudes of the 2018 summer heat event on a range of spatial scales. The statistic we use is the annual maximum of the 1-, 10-, and 90- day running mean of daily mean 2-m temperature (hereafter TM1x, TM10x, and TM90x respectively). We analyze three spatial scales: model grid box, regional, and European. For regional and European event definitions, the spatial mean is calculated before the running mean. Regional extents are taken from \citet{christensen_summary_2007}, and European extent is the E-OBS \citep{cornes_ensemble_2018} domain (land points within 25 -- 71.5\textdegree{} N, 25\textdegree{} W -- 45\textdegree{} E). The WWA used the annual maxima of 3-day mean daily maximum temperatures at specific grid points for its connection to local health effects \citep{dippoliti_impact_2010}, whereas the UKMO used the JJA mean temperature over the entire United Kingdom in order to answer the question of how anthropogenic forcings have affected the likelihood of U.K. summer seasons as warm as 2018. The same justifications can be used here, although we add that different heat event time scales are important to different groups of people, and as such using several temporal definitions may increase interest in heat event attribution studies. However, we recognize that other definitions than those used here may be more relevant to some impacts observed (such as defining the event in the context of the atmospheric flow pattern and drought that accompanied the heat), and other lines of reasoning for selecting one particular event definition exist \citep{cattiaux_defining_2018}.

\section{Model simulations \& validation}

  Three sets of simulations from the UKMO Hadley Centre HadGEM3-A global atmospheric model \citep{christidis_new_2013,ciavarella_upgrade_2018} are used. These are a historical ensemble (1960--2013; Historical) and factual (ACT) and counterfactual (a “natural” world without anthropogenic forcings; NAT) ensembles of 2018. We compare results from this factual-counterfactual analysis with those from a trend-based analysis of Historical, ensembles from EURO-CORDEX \citep{vautard_simulation_2013,jacob_euro-cordex_2014,vrac_influence_2017} (1971--2018) and RACMO \citep{aalbers_local-scale_2018,lenderink_preparing_2014} (1950--2018), and observations from E-OBS (1950--2018). A full model description is provided in the online supplemental information. Initially, we performed our analysis with the weather@home HadRM3P European-25 km setup \citep{massey_weatherhome-development_2015} but found that this model overestimates the variability over all Europe for daily through seasonal-scale event statistics, and so it was omitted.

\section{Methods}
%This needs a re-write
  We calculate the return period (RP) for the 2018 event in a distribution fit to E-OBS using the generalized extreme value (GEV) distribution to model TM1x and TM10x, and the generalized logistic distribution to empirically model TM90x throughout. Since the distribution of temperature extremes changes as the climate does, to account for the non-stationarity of the time series we first remove the trend attributable to low-pass-filtered globally averaged mean surface temperature (GMST, from Berkeley Earth; Rohde et al. 2013) in an ordinary least squares regression \citep[the regression coefficient or trend is shown in the supplemental material in Fig. ES1;][]{diffenbaugh_quantifying_2017}. We then find the temperature threshold corresponding to the RP in a distribution fit to the model's climatology. In the factual/counterfactual analysis, we do this by fitting parameters to a detrended (against GMST; trends shown in Figs. ES2c7--9) climatological ensemble of Historical plus 15 randomly sampled members of ACT. We finally calculate the probability (P) of exceeding this climatological temperature threshold in distributions fit to the ACT and NAT ensembles and calculate the probability ratio, $PR = P_{\mathrm{ACT}}/P_{\mathrm{NAT}}$, representing the increased likelihood of the 2018 event in the factual compared to the counterfactual world. Using estimated event probabilities rather than observed magnitudes constitutes a quantile bias correction \citep{jeon_quantile-based_2016}, minimizing model biases in the mean and variability of the temperatures analyzed. A description of uncertainty calculation and the trend-based analysis discussed below is included in the supplemental material.

  % \clearpage
  \begin{figure}[h]
    \centering
    \includegraphics[width=\textwidth]{{Fig1.2}.pdf}
    \caption[LOF caption]{}
    \label{fig2.2}
  \end{figure}
  % \clearpage

  % \clearpage
  \begin{figure}[h]
    \centering
    \includegraphics[width=\textwidth]{{Fig1.3}.pdf}
    \caption[LOF caption]{}
    \label{fig2.3}
  \end{figure}
  % \clearpage

\section{Results}
  % this will need a re-write for consistency with updated figures
  Extreme daily heat events, measured by TM1x, are distributed heterogeneously throughout Europe (Fig. ES1i). This is paralleled in the factual/ counterfactual PRs seen in Fig. 1a, with large proportions of the Iberian Peninsula, the Netherlands, and Scandinavia experiencing events that were highly unlikely in a climate without anthropogenic influence. A similar result is found on the regional scale (Fig. 1d) with Scandinavia and the Iberian Peninsula respectively experiencing 1-in-150 [26--26,000] and 1-in-30 [9--550] year events in the current climate that were highly unlikely in the natural climate simulated in NAT. The remaining regions record maximum daily temperatures likely to be repeated within 4 years. Considering the whole of Europe, the likelihood of the 2018 maximum of daily European mean temperature occurring without climate change is zero. This result is consistent with Uhe et al. (2016) and Angélil et al. (2018), who showed that increasing spatial scale tends to increase the probability ratio.

  Extreme 10-day heat events, TM10x, were also widespread in Europe, with the most extreme occurring in Scandinavia (Fig. ES1j). Regionally, the PRs become more uniform (Fig. 1d), although Scandinavia and the Iberian Peninsula still have very high bestestimate PRs of 185 [17--infinite] and 110 [18--56,000] respectively. The best-estimate PR for the average of Europe is still formally infinite. 

  The PR map for season-long heat events measured by TM90x is more uniform throughout Europe (Fig. 1c). Scandinavia, the British Isles, France, and central and eastern Europe, all of which experienced on the order of 1-in-10 year events (Fig. ES1l), and the corresponding best-estimate PRs are between 10 and 100 for all regions (Fig. 1d), including those with lower return periods. The PR for the European
  average is 1,000 [500--2,000].

  % \clearpage
  \begin{figure}[h]
    \centering
    \includegraphics[width=\textwidth]{{Fig1.4}.pdf}
    \caption[LOF caption]{}
    \label{fig2.4}
  \end{figure}
  % \clearpage

  Trend-based analysis [Figs. ES1m--p (observations) and Fig. ES2b (models)] yields similar results, although we note that for HadGEM-3A this results in generally higher PRs, due to the linear trend with GMST in the climatology being greater than the difference between the two ensembles used in the factual/counterfactual analysis. Observational and model analysis contradict in some grid boxes in northern Scandinavia for TM1x and TM10x, since the observed best-estimate trend against GMST is negative, reducing the event probability for the presentday compared to the preindustrial climate, therefore yielding PRs of less than 1. Comparing the regional factual/counterfactual model with observational analysis (Fig. 1d vs Fig. ES1p) shows that the large observational uncertainties overlap with the model results: the difference could be due to natural variability affecting the small observational sample size. However, we are cautious of drawing any conclusions regarding the change in likelihood of extreme heat events as defined here for these locations.

  % \clearpage
  \begin{figure}[h]
    \centering
    \includegraphics[width=\textwidth]{{Fig1.5}.pdf}
    \caption[LOF caption]{}
    \label{fig2.5}
  \end{figure}
  % \clearpage

  The PR increases with the event statistic time scale for the majority of grid points and regions (shown in Fig. 1). Figure 2 illustrates the cause using the British Isles region: as the time scale increases, the event statistic distribution variance decreases, while the mean shift between the factual and counterfactual distributions remains constant. Figure ES1t shows that the similarity in trends with GMST between the three time scales is also true for the observations. The decrease in variance usually results in higher PRs, given a particular event return time, for the longer time scales. There are exceptions due to the bounded upper tail of a GEV distribution with a negative shape parameter, resulting in the very high PRs for TM1x in Scandinavia, the Iberian Peninsula, and the Netherlands. The solid and dotted black lines compare the temperature thresholds when using event return periods to anomaly magnitudes in E-OBS. This explains why the TM90x PR is much higher than the other time scales for the British Isles: in addition to the decreased variance, the seasonal-scale heat event was more unusual than the other time scales, with a longer return period (10.6 [5.7--21] years) than TM10x (2.6 [1.8--3.9] years) and TM1x (3.6 [2.5--6.2] years). These factors together result in PRs of 3.6 [2.9--4.8] for TM1x and 43 [27--84] for TM90x. We suggest that the change in variance between the time scales used largely reconciles the differences between the “2 to 5” and “30” times increases in likelihood found by the WWA and UKMO reports, with other methodological factors playing a minor role as we have demonstrated for the British Isles. Although higher return periods for TM90x do impact the PRs found, this effect is generally less significant than changes in variability between the time scales.

  % \clearpage
  \begin{figure}[h]
    \centering
    \includegraphics[width=\textwidth]{{Fig1.6}.pdf}
    \caption[LOF caption]{}
    \label{fig2.6}
  \end{figure}
  % \clearpage

  Figure 2 also demonstrates a relevant deficiency in the model: the model distributions are narrower than the observed distributions, meaning the model has lower variability than the real world. This reduced variance has a significant impact on attribution results \citep{bellprat_towards_2019} and means that the PRs for the British Isles presented here, especially for TM90x, are likely to be overestimated. Underrepresented variability often occurs in prescribed SST models \citep{fischer_biased_2018,he_does_2016} and is visible in HadGEM-3A for many coastal locations over Europe (Figs. ES2a7--9). Figure 2d shows the power spectrum of JJA summer temperatures over the British Isles, indicating that HadGEM3-A has similar spectral characteristics to E-OBS, but underrepresents the intraseasonal 2-m temperature variability at almost all frequencies, which will likely result in overestimated PRs. Power spectra for other model ensembles are shown for comparison, demonstrating that the fully bias-corrected EURO-CORDEX ensemble has the same variability characteristics and magnitude as the observations.

  % \clearpage
  \begin{figure}[h]
    \centering
    \includegraphics[width=\textwidth]{{Fig1.7}.pdf}
    \caption[LOF caption]{}
    \label{fig2.7}
  \end{figure}
  % \clearpage

\section{Discussion}

  Our analysis highlights a key property of extreme weather attribution: the variance of the event definition used, both in terms of the statistic itself and its representation within any models used. The use of longer temporal event scales in general increases both the spatial uniformity and magnitude of the probability ratios found, consistent with Kirchmeier-Young et al. (2019), due to a decrease in variance compared to shorter scales. The difference in temporal scale between two reports concerning the 2018 summer heat is sufficient to explain the large discrepancy in attribution result between them. We find that several European regions experienced season-long heat events with a present-day return period greater than 10 years. The present-day likelihood of such events occurring is approximately 10 to 100 times greater than a “natural” climate. The attribution results also show that the extreme daily temperatures experienced in parts of Scandinavia, the Netherlands, and the Iberian Peninsula would have been highly unlikely without anthropogenic warming. The prescribed SST model experiments used here tend to underestimate the variability of temperature extremes near the coast, which may lead to the attribution results overstating the increase in likelihood of such extremes due to anthropogenic climate change \citep{bellprat_towards_2019}. We aim to properly quantify the impact of the underrepresented variability in further work. Although here we have used an unconditional temperature definition for consistency with the studies we try to reconcile, we plan to further investigate the effect of including both the atmospheric flow context and other impact-related variables such as precipitation in the event definition, and address issues models might have with realistically simulating the physical drivers of heatwaves.

\section{Chapter close}
%% Thoughts on this study in the context of the Thesis
