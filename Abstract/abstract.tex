Anthropogenic greenhouse gas emissions are now well-understood to be causing widespread and often damaging changes to the climate. One of the many ways in which the climate is changing is through the characteristics of extreme weather. Given the severe consequences that these devastating events can have, understanding how human influence on the climate is affecting them is of considerable importance. This is the aim of the relatively new field of `extreme event attribution'. Although there now exist a number of established methods for attributing individual weather events to climate change, ranging from probabilistic approaches with large climate model ensembles to very conditioned storyline approaches, questions still remain over the reliability of these approaches, especially when considering the most unprecedented events. We need to have confidence in this understanding in order to plan adaptation measures effectively to mitigate changes in risk into the future. In this thesis, I show how the models that we use to forecast the weather could provide us with such reliable information about how humanity is affecting extreme weather --- and in particular extreme heat. These models are state-of-the-art and can be shown to be unequivocally able to simulate the detailed physics of specific extreme weather through a successful prediction. Here I develop a perturbed initial and boundary condition approach using the European Center ensemble prediction system that aims to produce forecasts of individual events as if they had occurred in a world without human influence on the climate. These `counterfactual' forecasts can then be used to assess how not only the intensity, but also the probability of these events has changed. Although extreme weather attribution typically focuses on the past climate, the same approach can in theory be used to produce forecasts in possible warmer future worlds --- thus providing vital information about how the most damaging weather in the present may be expected to change in the future. In the final chapter, I depart from attribution but continue with this theme of extreme weather projection, examining a novel approach to producing large model ensembles that explore the range of uncertainty in future extreme weather. This work contrasts the very specific nature of the attribution of single weather events, but both approaches to extreme weather projection together could provide complementary and useful information for adaptation planning. 