Anthropogenic greenhouse gas emissions are now well-understood to be causing damaging changes to the climate. One of the many ways in which the climate is changing is through extreme weather events. Given the severe consequences of such events, understanding how human influence on the climate is affecting them is vital. This is the aim of the young field of `extreme event attribution'. There now exist many established methods for attributing individual weather events to climate change, from probabilistic approaches utilising large climate model ensembles to deterministic storyline approaches. However, questions still remain over the reliability of these approaches, especially when considering the most unprecedented events. In this thesis, I show how weather forecast models could provide us with such reliable information about human influence on extreme weather --- focusing on extreme heat. These models are state-of-the-art and can be shown to be unequivocally able to simulate the detailed physics of specific extreme weather through successful prediction. I develop a perturbed initial- and boundary-condition approach within an operational forecasting system that aims to produce forecasts of individual events as if they had occurred in a world without human influence on the climate. These `counterfactual' forecasts can be used to assess how not only the intensity, but also the probability of such events has changed. Although extreme weather attribution typically focuses on the past, the same approach could be used to produce forecasts in warmer future worlds --- thus providing vital information about how the most damaging weather may be expected to change in the future. I explore this theme of extreme weather projection, examining a novel approach to producing large climate model ensembles that span the range of uncertainty in future extreme weather. This work complements the specific nature of extreme event attribution, and they could together provide crucial information for adaptation planning. 