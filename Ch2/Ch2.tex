\begin{savequote}[8cm]
    Quote
      \qauthor{--- author}
\end{savequote}
    
\chapter{\label{ch2}Conventional probabilistic attribution} 

Here I present a probabilistic extreme event attribution of the 2018 European heatwave. Whilst demonstrating the methodologies behind this framework, I examine how one particular aspect of probabilistic event attribution --- the definition of the event --- projects strongly onto the quantitative results. In the closing remarks, I reflect on potential issues with the approach taken within the chapter, and suggest ways in which these could be overcome.
{\small\paragraph{Author contributions:} This chapter is based on the following publication \footnote{with the author contributing as follows. Conceptualisation, Data curation, Formal analysis, Investigation, Methodology, Resources, Visualisation, Writing -- original draft and Writing --- Review \& Editing.} \par\vspace{1em}
\formatchref{Leach, N. J., Li, S., Sparrow, S., van Oldenborgh, G. J., Lott, F. C., Weisheimer, A., \& Allen, M. R.}{2020}{Anthropogenic Influence on the 2018 Summer Warm Spell in Europe: The Impact of Different Spatio-Temporal Scales}{Bulletin of the American Meteorological Society}{101}{1}{S41-S46}{https://doi.org/10.1175/BAMS-D-19-0201.1}}

\clearpage

\minitoc

\clearpage

\section{Chapter open}\label{ch2:open}

  The aim of this chapter, both in this thesis and at the time I started the work, is to provide a practical example of how probabilistic extreme event attribution is typically carried out. At the time, the 2018 European heatwave was a clear case study to pick --- it had not only been responsible for very significant damages, but had already been analysed by two separate attribution teams, who came to seemingly quite different conclusions. One group at the UK Met Office estimated that the heatwave had been made 30 times more likely, while the other group, the World Weather Attribution (WWA) project, stated that it had been made 2--5 times more likely due to human influence on the climate. These two numbers are an order of magnitude apart, and I aimed to determine why they were so different.  Whilst resolving this discrepancy was the main research question, the other clear goal was to provide me with some experience in actually carrying out an attribution study, and therefore identify the many gaps in my knowledge that existed at the time. I collaborated with several researchers from the WWA to ensure that the methods I was using were consistent with those that they had used themselves. 

\section{Abstract}\label{ch2:abstract}

  We demonstrate that, in attribution studies, events defined over longer time scales generally produce higher probability ratios due to lower interannual variability, reconciling seemingly inconsistent attribution results of Europe's 2018 summer heatwaves in reported studies.

\section{The 2018 heatwave in Europe}\label{ch2:heatwave}

  The summer of 2018 was extremely warm in parts of Europe, particularly Scandinavia, the Iberian Peninsula, and Central Europe, with a range of all-time temperature records set across the continent \citep{johnston_heatwave_2018,nesdis_record_2018}. Impacts were felt across Europe, with wildfires burning in Sweden \citep{krikken_attribution_2021,watts_wildfires_2018}, heatstroke deaths in Spain \citep{publico_nueve_2018}, and widespread drought \citep{harris_heat_2018}. During the summer, the World Weather Attribution (WWA) initiative released an analysis of the heat spell \citep{world_weather_attribution_heatwave_2018} based on observations/forecasts and models in specific locations (Dublin, Ireland; De Bilt, Netherlands; Copenhagen, Denmark; Oslo, Norway; Linkoping, Sweden; Sodankyla, Finland; Jokionen, Finland), which concluded that the increase in likelihood due to human induced climate change was at least 2 to 5 times. In December, the U.K. Met Office (UKMO) stated that they found the 2018 U.K. summer temperatures were made 30 times more likely \citep{press_office_chance_2018,mccarthy_drivers_2019}. These two estimates appear to quantitatively disagree; however, we show they can be reconciled by considering the effect of using different spatial domains and temporal scales in the event definition. We also demonstrate that prescribed SST model simulations can under-represent the variability of temperature extremes, especially near the coast, with implications for any derived attribution results.

  % \clearpage
  \begin{figure}[h]
    \centering
    \includegraphics[width=\textwidth]{{Fig2.1}.pdf}
    \caption[The 2018 heatwave in Europe: observed mean temperature anomalies over a range of timescales]{\textbf{The 2018 heatwave in Europe: observed mean temperature anomalies over a range of timescales.} Shading indicates mean temperature anomalies for the the different temporal-scale heatwave metrics used. Black contours indicate z-scores of the 2018 heatwave for the three metrics based on detrended historical data from E-OBS. The contours indicate scores of 1-, 2-, and 3-σ, in order of increasing thickness.}
    \label{fig2.1}
  \end{figure}
  % \clearpage

\subsection{Defining the event}\label{ch2:defining}

  We consider various temperature-based event definitions to demonstrate the impact of this choice in attribution assessments, and assess to what extent human influence affected the seasonal and peak magnitudes of the 2018 summer heat event on a range of spatial scales. The metric we use is the annual maximum of the 1-, 10-, and 90- day running mean of daily mean 2-m temperature (hereafter TM1x, TM10x, and TM90x respectively). We analyse two spatial scales: model grid box and regional. For regional event definitions, the spatial mean is calculated before the annual maximum. Regional domains are taken from \citet{christensen_summary_2007}. Figure \ref{fig2.1} shows the 2018 anomaly field for each of these metrics. In their study, the WWA used the annual maxima of 3-day mean daily maximum temperatures at specific grid points for its connection to local health effects \citep{dippoliti_impact_2010}, whereas the UKMO used the JJA mean temperature over the entire United Kingdom in order to answer the question of how anthropogenic forcings have affected the likelihood of U.K. summer seasons as warm as 2018. The same justifications can be used here, although we add that different heat event time scales are important to different groups of people, and as such using several temporal definitions may increase interest in heat event attribution studies. However, we recognize that other definitions than those used here may be more relevant to some impacts observed (such as defining the event in the context of the atmospheric flow pattern and drought that accompanied the heat), and other lines of reasoning for selecting one particular event definition exist \citep{cattiaux_defining_2018}.

\section{Materials \& methods}\label{ch2:methods}

  \subsection*{Model simulations \& validation}

    We primarily use three sets of simulations from the UKMO Hadley Centre HadGEM3-A global atmospheric model \citep{christidis_new_2013,ciavarella_upgrade_2018}. These are a 15-member historical ensemble (1960--2013; Historical), and 525-member factual (ACT, referred to as HistoricalExt by \citeauthor{ciavarella_upgrade_2018}) and counterfactual (a “natural” world without anthropogenic forcings; NAT, referred to as HistoricalNatExt by \citeauthor{ciavarella_upgrade_2018}) ensembles of 2018. Historical and ACT are forced by observed SSTs and sea ice concentrations from HadISST \citep{rayner_global_2003}. NAT is forced by naturalised SST and sea ice concentrations estimated by subtracting the multi-model mean difference between the CMIP5 `historical' and `historicalNat' experiments \citep{taylor_overview_2012}, and run with pre-industrial greenhouse gas concentrations. For a complete description of these ensembles, see \citet{ciavarella_upgrade_2018}. We compare results from these factual-counterfactual simulations with those from a trend-based analysis of the 15-member HadGEM3-A Historical ensemble, a 10-member ensemble from EURO-CORDEX \citep[1971--2018;][]{vautard_simulation_2013,jacob_euro-cordex_2014,vrac_influence_2017} and a 16-member ensemble from the RACMO regional downscaling of EC-EARTH 2.3 \citep[1950--2018;][]{aalbers_local-scale_2018,lenderink_preparing_2014}. The EURO-CORDEX ensemble used here is bias-corrected using the cumulative distribution function transform. Both the EURO-CORDEX and RACMO ensembles follow the CMIP5 `historical' scenario to 2005 and the RCP4.5 scenario thereafter \citep{taylor_overview_2012,thomson_rcp45_2011}. Observations are taken from E-OBS \citep[1950--2018;][]{cornes_ensemble_2018} throughout. Initially, we performed our analysis with the weather@home HadRM3P European-25 km setup \citep{massey_weatherhome-development_2015} but found that this model overestimates the variability over all Europe for daily through seasonal-scale event statistics, and so it was omitted.

  \subsection*{Attribution methodology}

    \subsubsection*{Estimating the event threshold}

      We first use historical data to estimate the return time of the 2018 event, and the corresponding temperature threshold in each model. We start by calculating the return period for the 2018 event in E-OBS. Since the distribution of temperature extremes changes as the climate changes, to account for the non-stationarity of the time series we remove the attributable trend by regressing onto the anthropogenic component of globally averaged mean surface temperature \citep[the anthropogenic warming index, based on HadCRUT5;][]{haustein_real-time_2017,morice_updated_2021,diffenbaugh_quantifying_2017}. We then fit extreme value (EV) distribution parameters to this detrended E-OBS time series, and use these parameters to calculate the estimated return period of the 2018 event. We then find the temperature threshold in the model climatology that corresponds to this return period. We do this by fitting EV parameters to a detrended (by regressing onto the anthropogenic warming index) climatological ensemble for each model. For HadGEM3-A, the climatology is Historical plus 15 randomly sampled members of ACT; for RACMO and EURO-CORDEX, the climatology is taken as the entire set of simulations described above. The calculation of model-specific climatological temperature thresholds from the E-OBS temperature threshold is illustrated for the British Isles region in Figure \ref{fig2.2}. Using estimated event probabilities rather than observed magnitudes to define the event constitutes a quantile bias correction \citep{jeon_quantile-based_2016}, accounting for model biases in the mean and variability of the temperatures simulated.

      % \clearpage
      \begin{figure}[h]
        \centering
        \includegraphics[width=\textwidth]{{Fig2.2}.pdf}
        \caption[Calculating the heatwave threshold in HadGEM3-A from observations]{\textbf{Calculating the heatwave threshold in HadGEM3-A from observations.} The heatwave metric used here for illustration is TM1x for the BI region. Solid black lines indicate empirical CDFs. Solid grey lines indicate GEV distributions fit to the data, with grey shading indicating confidence intervals of the fit. The dotted black line indicates the temperature observed during the 2018 heatwave in E-OBS. Dotted grey lines indicate the best-estimate quantile corresponding to the 2018 event in E-OBS based on the GEV fit, and the temperature of that quantile in the HadGEM3-A historical climatology.}
        \label{fig2.2}
      \end{figure}
      % \clearpage

    \subsubsection*{Counterfactual attribution}

      For the main analysis, which we term ``counterfactual'' attribution \cite{stott_human_2004}, we estimate the probability ($P$) of exceeding this climatological temperature threshold in the ACT and NAT ensembles. We do this by fitting EV parameters to each ensemble, and using them to calculate $P_{\text{ACT}}$ and $P_{\text{NAT}}$. The estimate ACT and NAT distributions are shown for each metric in Figure \ref{fig2.3}. We express our results as the probability ratio, $PR = P_{\text{ACT}}/P_{\text{NAT}}$, representing the fractional change in probability of the 2018 event in the factual compared to the counterfactual world.

    \subsubsection*{Trend-based attribution}

      We support the counterfactual attribution with a trend-based analysis \cite{van_oldenborgh_absence_2012} of E-OBS and all the model ensembles used. This trend-based analysis is based on the climatology alone, and does not require factual and counterfactual simulations. We start with the EV parameters fit to the detrended climatology, and then use the estimated climate change trend between 1900 and 2018 to shift the location of the EV distribution. This shifted distribution then represents the counterfactual distribution (analogous to NAT), and the original distribution represents the factual distribution (analogous to ACT), from which we can calculate probability ratios.

  \subsection*{Statistical methods}

    We fit EV parameters using the method of L-Moments \cite{hosking_l-moments_1990}, modelling TM1x and TM10x using the generalized extreme value (GEV) distribution, and TM90x using the generalized logistic distribution. Uncertainties are calculated using a 10,000 resample non-parametric bootstrap throughout.

  % \clearpage
  \begin{figure}[h]
    \centering
    \includegraphics[width=\textwidth]{{Fig2.3}.pdf}
    \caption[Factual and counterfactual PDFs of the 2018 heatwave defined over three temporal scales]{\textbf{Factual and counterfactual PDFs of the 2018 heatwave defined over three temporal scales.} The heatwave metric used is given in the title of each panel. Solid red and blue lines indicate best-estimate GEV distributions fit to HadGEM3-A 2018 ACT and NAT ensembles respectively. Dotted grey line indicates 2018 event threshold defined using HadGEM3-A and E-OBS climatology (see Figure \ref{fig2.2}). Shading illustrates confidence intervals.}
    \label{fig2.3}
  \end{figure}
  % \clearpage

\section{Results}\label{ch2:results}
  
  Extreme daily heat events, measured by TM1x, are distributed heterogeneously throughout Europe. This is paralleled in the probability ratios seen in Figure \ref{fig2.4}, with large areas of the Iberian Peninsula, the Netherlands, and Scandinavia experiencing events that were highly unlikely in a climate without anthropogenic influence. A similar result is found on the regional scale in Figure \ref{fig2.5} with Scandinavia and the Iberian Peninsula respectively experiencing 1-in-150 [25--25,000]\footnote{Numbers in brackets [] represent a 90\% confidence interval throughout this chapter.} and 1-in-30 [9--550] year events in the current climate that were highly unlikely in the natural climate simulated in NAT. The remaining regions recorded maximum daily temperatures expected to be repeated within 4 years. The probability ratios for regional domains are typically larger than single gridboxes within them, though some regions contain clusters of extremely high probability ratios. This result is consistent with Uhe et al. (2016) and Angélil et al. (2018), who showed that increasing the spatial scale over which the event is defined tends to increase the probability ratio.

  Extreme 10-day heat events, TM10x, were also widespread in Europe, with the most extreme occurring in Scandinavia (Figure \ref{fig2.1}). Regionally, the PRs become more uniform (Figure \ref{fig2.5}), although Scandinavia and the Iberian Peninsula still have very high best-estimate PRs of 800 [20--infinite] and 85 [25--40,000] respectively.

  The PR map for season-long heat events measured by TM90x is more uniform throughout Europe (Figure \ref{fig2.4}). Scandinavia, the British Isles, France, and central and Eastern Europe all experienced on the order of 1-in-10 year events. The corresponding best-estimate PRs are between 10 and 100 for all regions (Figure \ref{fig2.5}), including those with lower return periods.

  % \clearpage
  \begin{figure}[h]
    \centering
    \includegraphics[width=\textwidth]{{Fig2.4}.pdf}
    \caption[Maps of the estimated change in probability of the 2018 heatwave due to anthropogenic influence on the climate]{\textbf{Maps of the estimated change in probability of the 2018 heatwave due to anthropogenic influence on the climate.} The heatwave metric used is given in the title of each panel. Shading illustrates risk ratio of 2018 event at each gridpoint computed using HadGEM3-A ACT and NAT ensembles.}
    \label{fig2.4}
  \end{figure}
  % \clearpage

  A trend-based analysis yields similar results, with PRs for the British Isles region shown in Figure \ref{fig2.6}, though we note that for HadGEM-3A this results in generally higher PRs than the corresponding counterfactual analysis, due to the attributable anthropogenic trend in the climatology being greater than the mean difference between the ACT and NAT ensembles. For the vast majority of the regions and metrics analysed here, the trend-based observational, trend-based model, and counterfactual model estimates of the return period are consistent with one another, and Figure \ref{fig2.5} is a good representation of the synthesis of these different approaches and data sources. However, there are a few regions with notable discrepancies. The uncertainty in the E-OBS observed Scandinavia region TM1x trend is large enough that a 90\% confidence interval is not able to rule out a negative trend. Hence, the corresponding probability ratio confidence interval includes values of less than 1 (i.e. that TM1x events at least as hot as the 2018 event have been made less likely by climate change). This interval is large enough that it does still overlap with all the model-derived estimates, all of which suggest that the probability ratio is very likely greater than 70. This very large interval may arise due to natural variability affecting the relatively small sample size. Synthesizing these strands of information, we suggest that such daily extreme heat events over Scandinavia have been made more likely by climate change, but we are cautious about drawing very firm conclusions. The other clear discrepancy is for the TM90x metric British Isles results, shown in Figure \ref{fig2.6}. Despite good agreement between all other approaches and sources, the trend-based HadGEM3-A estimate is an order of magnitude higher and does not overlap with the others. This appears to be due to the variability of British Isles temperatures on this \textasciitilde seasonal timescale being underestimated by this model, even though the estimated trend closely matches the other models and observations. We discuss this further \hyperlink{ch2:vardiscuss}{below}. 

  % \clearpage
  \begin{figure}[h]
    \centering
    \includegraphics[width=\textwidth]{{Fig2.5}.pdf}
    \caption[Estimated changes in probability of the 2018 heatwave defined using regional mean temperatures]{\textbf{Estimated changes in probability of the 2018 heatwave defined using regional mean temperatures.} Colour indicates heatwave metric. Dots indicate central risk-ratio estimate and bars indicate 90\% confidence interval.}
    \label{fig2.5}
  \end{figure}
  % \clearpage

  The PR increases with the event statistic timescale for the majority of grid points and regions, demonstrated in Figures \ref{fig2.4} and \ref{fig2.5}. Figure \ref{fig2.3} illustrates the cause using the British Isles region: as the timescale increases, the variance in the event metric decreases, while the mean shift between the factual and counterfactual distributions remains comparable. The similarity in attributed anthropogenic trend for the three time scales is also true in the observations and other models. The decrease in variance usually results in higher PRs, given a particular event return time, for the longer time scales. There are exceptions due to the bounded upper tail of a GEV distribution with a negative shape parameter, resulting in the very high estimated PRs for TM1x in Scandinavia, the Iberian Peninsula, and the Netherlands (Figure \ref{fig2.5}). Now focussing on the British Isles region, Figure \ref{fig2.3} also shows another reason why the TM90x metric probability ratios are much higher than the TM10x or TM1x results: in addition to the decreased variance in the TM90x metric, the 2018 event was more unusual when measured with this metric (a return period of 10.3 [5.7--20] years) compared to the two shorter timescale metrics (return periods of 2.5 [1.7--3.8] and 3.6 [2.4--6.0] for TM10x and TM1x respectively). These two factors (reduced variance and rarer event) result in best-estimate probability ratios of 3.7 [2.9--4.9] for TM1x and 29 [17--57] for TM90x. We therefore suggest that changes in the variance of the event metric as the time scales used changes largely reconciles the differences between the ``2 to 5'' and ``30'' times increases in likelihood found by the WWA and UKMO reports, with other methodological factors, such as the spatial scale used in the event definition, playing a more minor role as we have demonstrated for the British Isles.

  % \clearpage
  \begin{figure}[h]
    \centering
    \includegraphics[width=\textwidth]{{Fig2.6}.pdf}
    \caption[Estimated changes in probability of the 2018 British Isles heatwave across a range of observations and model simulations]{\textbf{Estimated changes in probability of the 2018 British Isles heatwave across a range of observations and model simulations.} Here, risk-ratios are estimated using a trend-based analysis. Colour indicates data source. Dots indicate central risk-ratio estimate and bars indicate 90\% confidence interval.}
    \label{fig2.6}
  \end{figure}
  % \clearpage

  \hypertarget{ch2:vardiscuss}{As} mentioned above, the trend-based HadGEM3-A analysis appears to overestimate the probability ratio of the 2018 event when considering the other approaches taken here (Figure \ref{fig2.6}). This is due to an important deficiency in the model: the model distributions are narrower than the observed distributions for this heatwave metric, meaning the model has lower variability in temperatures on seasonal timescales than the real world. This reduced variance has a significant impact on attribution results \citep{bellprat_towards_2019} and means that the trend-based PRs for this model and over the British Isles region presented here, especially for TM90x, are likely to be overestimated. Underrepresented variability often occurs in prescribed-SST models \citep{fischer_biased_2018,he_does_2016} and is present in HadGEM-3A for many coastal gridboxes in Europe. Figure \ref{fig2.7} shows the power spectrum of JJA summer temperatures over the British Isles, indicating that HadGEM3-A has broadly similar spectral characteristics to E-OBS, but under-represents the intraseasonal 2 m temperature variability at almost all frequencies, which will likely result in overestimated PRs. Power spectra for other model ensembles are shown for comparison, demonstrating that only the fully bias-corrected EURO-CORDEX ensemble has variability characteristics and magnitude that closely match the observations.

  % \clearpage
  \begin{figure}[h]
    \centering
    \includegraphics[width=\textwidth]{{Fig2.7}.pdf}
    \caption[Historical power spectrum of summer daily mean temperatures over the British Isles across a range of observations and model simulations]{\textbf{Historical power spectrum of summer daily mean temperatures over the British Isles across a range of observations and model simulations.} Power spectra are estimated as periodograms averaged over all historical years available for each data source, expressed as a fraction of the E-OBS power at each frequency. Colour indicates data source. Thick lines show ensemble mean power for each source. Thin lines show individual ensemble members for each source.}
    \label{fig2.7}
  \end{figure}
  % \clearpage

\section{Discussion}\label{ch2:discussion}

  Our analysis highlights a key property of extreme weather attribution: the variance of the event definition used, both in terms of the statistic itself and its representation within any models used. The use of longer temporal event scales in general increases both the spatial uniformity and magnitude of the probability ratios found, consistent with \citet{kirchmeieryoung_importance_2019}, due to a decrease in variance compared to shorter scales. The difference in temporal scale between two reports concerning the 2018 summer heat is sufficient to explain the large discrepancy in attribution result between them. We find that several European regions experienced season-long heat events with a present-day return period greater than 10 years. The present-day likelihood of such events occurring is approximately 10 to 100 times greater than a ``natural'' climate without human influence. The attribution results also show that the extreme daily temperatures experienced in parts of Scandinavia, the Netherlands, and the Iberian Peninsula would have been exceptionally unlikely without anthropogenic warming. The prescribed-SST model used primarily here tends to underestimate the variability of temperature extremes near the coast, which may lead to the attribution results overstating the increase in likelihood of such extremes due to anthropogenic climate change \citep{bellprat_towards_2019}. We aim to properly quantify the impact of the underrepresented variability in further work. Although here we have used an unconditional temperature definition for consistency with the studies we aimed to reconcile, we plan to further investigate the effect of including both the atmospheric flow context and other impact-related variables such as precipitation in the event definition, and address issues models might have with realistically simulating the physical drivers of heatwaves.

\section{Chapter close}\label{ch2:close}

  At the start of this chapter, I set out that I had two main goals for the study: to understand how two seemingly contradictory attribution results arose; and to gain practical experience in carrying out probabilistic attribution analyses myself. I believe that the contents of this chapter demonstrate that I achieved the first, and (hopefully) that the contents of this thesis demonstrate that I achieved the second to the degree required by my work. However, although both of these outcomes were realised, this study left me with a number of outstanding questions. Can we claim to be attributing a specific event if we simply consider all weather events that happened to be the hottest during a particular year, without considering anything else about the meteorology? Is a purely statistical model validation sufficient? How important really is ocean-atmosphere coupling? How do you decide how to define your event given that this definition can have huge implications for the quantitative result? Should we include more information about the event in the event definition (such as the atmospheric flow, or other variables like soil moisture), in order to provide a more event-specific analysis? 

  Several of these questions are highly related (especially those concerning event specificity and model validation). Following this work, I spent some time looking into the biases in variability in the models I had used. Although my work into these biases did not lead to any concrete and publishable results, I have attempted to address a number of the other questions raised here in the remainder of the thesis. In particular, the questions surrounding event specificity might be answered by using conditioned model simulations. This conditioning could be applied in many different ways, but one particularly attractive one would be to use successful weather forecasts. In this way you could reasonably claim to be attributing the \emph{specific} event in question, but also have a much higher degree of confidence that the model being used was actually able to simulate such an event in the way that it had unfolded in reality. As a result, much of the rest of this thesis, and its contribution to the field, is concerned with the idea of forecast-based attribution.