\begin{savequote}[8cm]
    Quote
      \qauthor{--- author}
\end{savequote}
    
\chapter{\label{ch3}Partial forecast-based attribution}

This chapter contains much of the conceptual description of, and motivation for, forecast-based attribution. Using the well-predicted February 2019 heatwave as a case study, I carry out forecasts with the operational medium-range ECMWF model in which I have instantaneously perturbed the CO$_2$ concentration at initialisation. These perturbed forecasts allow me to estimate the direct contribution of diabatic heating due to CO$_2$ to the heatwave. This partial attribution provides a proof-of-concept of the forecast-based approach, and I close with a discussussion of how I could perform a more complete estimate of anthropogenic influence on a specific extreme event in following work.
\small\paragraph{Author contributions:} This chapter is based on the the following publication \footnote{with the author contributing as follows. Conceptualisation, Data curation, Formal analysis, Investigation, Methodology, Resources, Visualisation and Writing -- original draft} \par\vspace{1em}
\formatchref{Leach, N. J., Weisheimer, A., Allen, M. R., \& Palmer, T.}{2021}{Forecast-based attribution of a winter heatwave within the limit of predictability}{Proceedings of the National Academy of Sciences}{118}{49}{}{https://doi.org/10.1073/pnas.2112087118}

\minitoc

\clearpage

\section{Section}

    \blindtext